\documentclass[a4paper]{article}

\usepackage{amsbsy,amsmath,amssymb,psfig}
\usepackage{times,mathpi,mathptm}
\usepackage{graphicx,psfrag,rotating,subfigure,supertabular}

%\usepackage[german]{babel}

\begin{document}

\title{grid-to-grid interpolation}

\author{Gerard J. Gorman\\ 
Earth Science and Engineering\\
Imperial College, Prince Consort Road\\ 
London SW7 2BP, UK\\
g.gorman@imperial.ac.uk}
\maketitle
\thispagestyle{empty}

%\noindent
%{\bf Abstract {\small\em }}
%\vspace{0.5cm}
%\noindent

\section{Introduction}

% The problems with having a static mesh
This library implements a grid-to-grid interpolation as described by
Rainald L\"{o}hner~\cite{lohner_grid2grid}.

\section{Fortran Interface}

In bath cases, mesh 1 refers to the source mesh while mesh 2 refers to
the destination mesh that's being interpolated to.

\subsection{Tri3 mesh Interpolation}
SUBROUTINE FLTri3toTri3(NNode1, NElems1, X1, Y1, 
                        ENLIST1, RMEM1, FIELDS1, NFIELDS,
                        NNode2, NElems1, X2, Y2,
                        ENLIST2, RMEM2, FIELDS2, IERROR)\\
\\
\noindent INTEGER NNode1, NElems1, NFIELDS         \\
REAL X1(NNode1), Y1(NNode1), RMEM1(*),             \\
INTEGER ENLIST1(NElems1*4)                         \\
INTEGER FIELDS1(NFIELDS)                           \\
INTEGER NNode2, NElems1,                           \\
REAL X2(NNode2), Y2(NNode2), RMEM2(*)              \\
INTEGER ENLIST2(NElems1*4)                         \\
INTEGER FIELDS2(NFIELDS)                           \\
INTEGER IERROR

\vspace{5mm}

\begin{supertabular}{lll}
IN  & NNode1   & number of nodes in mesh 1\\ 
IN  & NElems1  & number of elements in mesh 1\\ 
IN  & X1       & array of x positions of nodes in mesh 1\\ 
IN  & Y1       & array of y positions of nodes in mesh 1\\
IN  & ENLIST1  & element-node connectivity in mesh 1\\ 
IN  & RMEM1    & real memory array that contains the field vectors\\
IN  & FIELDS1  & array containing base pointers to fields in RMEM1\\ 
IN  & NFIELDS  & number of fields to be interpolated\\ 
IN  & NNode2   & number of nodes in mesh 2\\ 
IN  & NElems2  & number of elements in mesh 2\\ 
IN  & X2       & array of x positions of nodes in mesh 2\\ 
IN  & Y2       & array of x positions of nodes in mesh 2\\
IN  & ENLIST2  & element-node connectivity in mesh 2\\ 
OUT & RMEM2    & real memory array that contains the new field vectors\\
IN  & FIELDS2  & array containing base pointers to fields in RMEM2\\ 
OUT & IERROR   & return code for subroutine\\
\end{supertabular}

\subsection{Tetra4 mesh Interpolation}
SUBROUTINE FLTetra4toTetra4(NNode1, NElems1, X1, Y1, Z1, ENLIST1, RMEM1, FIELDS1, NFIELDS,
NNode2, NElems1, X2, Y2, Z2, ENLIST2, RMEM2, FIELDS2, IERROR)\\
\\
\noindent INTEGER NNode1, NElems1, NFIELDS         \\
REAL X1(NNode1), Y1(NNode1), Z1(NNode1), RMEM1(*), \\
INTEGER ENLIST1(NElems1*4)                         \\
INTEGER FIELDS1(NFIELDS)                           \\
INTEGER NNode2, NElems1,                           \\
REAL X2(NNode2), Y2(NNode2), Z2(NNode2), RMEM2(*)  \\
INTEGER ENLIST2(NElems1*4)                         \\
INTEGER FIELDS2(NFIELDS)                           \\
INTEGER IERROR

\vspace{5mm}

\begin{supertabular}{lll}
IN  & NNode1   & number of nodes in mesh 1\\ 
IN  & NElems1  & number of elements in mesh 1\\ 
IN  & X1       & array of x positions of nodes in mesh 1\\ 
IN  & Y1       & array of y positions of nodes in mesh 1\\ 
IN  & Z1       & array of z positions of nodes in mesh 1\\ 
IN  & ENLIST1  & element-node connectivity in mesh 1\\ 
IN  & RMEM1    & real memory array that contains the field vectors\\
IN  & FIELDS1  & array containing base pointers to fields in RMEM1\\ 
IN  & NFIELDS  & number of fields to be interpolated\\ 
IN  & NNode2   & number of nodes in mesh 2\\ 
IN  & NElems2  & number of elements in mesh 2\\ 
IN  & X2       & array of x positions of nodes in mesh 2\\ 
IN  & Y2       & array of x positions of nodes in mesh 2\\ 
IN  & Z2       & array of x positions of nodes in mesh 2\\ 
IN  & ENLIST2  & element-node connectivity in mesh 2\\ 
OUT & RMEM2    & real memory array that contains the new field vectors\\
IN  & FIELDS2  & array containing base pointers to fields in RMEM2\\ 
OUT & IERROR   & return code for subroutine\\
\end{supertabular}

\section{Conclusions}
It works.

\bibliographystyle{plain} 
\bibliography{references}

\end{document}
