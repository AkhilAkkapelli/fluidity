
\chapter{Adaptive remeshing}\label{chap:Adaptivity}

\section{Motivation}
Historically, numerical analysts concerned themselves
with \emph{a priori} error bounds of particular numerical
schemes, i.e.\@ \ asymptotic analyses of the order of convergence
of a discretisation with respect to some discretisation parameter
such as mesh sizing $h$ or polynomial order $p$. However, such
\emph{a priori} error bounds do not provide useful estimates
of the simulation error of a particular physical system on
a particular mesh for a specified norm: they merely describe
how that error behaves as the discretisation is modified. Since
such \emph{a priori} error bounds involve the unknown exact solution,
they are, in general, not computable.

In the late 1970s, the pioneering work of Babu\v ska
and Rheinboldt laid the foundations for \emph{a posteriori}
error estimates \citep{babuska1978,babuska1978b}.
In contrast to \emph{a priori} bounds,
\emph{a posteriori} error estimates involve only
the approximate computed solution and data from the problem,
and are thus computable (or approximately so, if they involve the
solution of an auxiliary problem). These error
estimates can then be used in an adaptive loop, modifying
the discretisation until some user-specified error criterion
is reached. Most \emph{a posteriori} error estimation literature deals with estimating
the error in the natural norm induced by the bilinear form
of the problem, the energy norm.
For a review of \emph{a posteriori} error estimation
with emphasis on energy norm estimation
see the books of Verf\"urth \citep{verfurth1996} and
Ainsworth and Oden \citep{ainsworth2000}. The goal-oriented
adaptive framework of Rannacher and co-workers, which estimates
the error in the computation of a given goal functional, is detailed
in \citet{becker2001} and \citet{bangerth2003}.

Once \emph{a posteriori} estimates have been computed,
there are many possible ways of modifying the discretisation
to achieve some error target. These include $h$-adaptivity,
which changes the connectivity of the mesh \citep{berger1989}; $p$-adaptivity,
which increases the polynomial order of the approximation
\citep{babuska1994}; and $r$-adaptivity, which relocates
the vertices of the mesh while retaining the same connectivity
\citep{budd2009}. Combinations of these methods are also
possible (e.g., \citet{houston2001,ledger2003}).

This review focuses on adaptive remeshing in multiple dimensions, as this is
the technology available in \fluidity.
This approach is the most powerful of all $hr$-adaptive methods, since the meshes produced are not constrained
by the previous mesh; therefore, this approach allows for maximum flexibility in
adapting to solution features. However, this flexibility comes
at a cost: guiding the adaptive remeshing procedure (choosing
what mesh to construct), executing the adaptation (constructing
the chosen mesh) and data transfer of solution fields (from the previous mesh to
the newly adapted mesh) become more complicated than with hierarchical refinement.

This chapter is divided as follows. Section \ref{sec:meshes_and_metrics} describes how a metric tensor field
may be used to encode the geometric properties (length scales and orientations) of the
elements of a mesh. Section \ref{sec:adaptive_remeshing_technology} describes how the
mesh optimisation procedure in \fluidity\ generates an adapted mesh, given a metric
describing its desired geometric properties. Section \ref{sec:metric_formation}
describes how the desired metric is generated from the computed solution. Section
\ref{sec:interpolation} then discusses how fields are interpolated between the
previous mesh and the newly adapted mesh. Section \ref{sec:adaptivity_related_topics}
discusses some related topics.

\section{Representing meshes as metric tensors} \label{sec:meshes_and_metrics}
The process of adaptive remeshing divides naturally into two parts. The first is to
decide what mesh is desired; the second is to actually generate that mesh. The form
of communication between these two stages is a symmetric positive-definite tensor field
which encodes the desired geometric properties of the mesh. This representation was first introduced
in \citet{vallet1990}.

Let $M$ be a symmetric positive-definite tensor (not a field, for now).
$M$ induces an inner product by
\begin{equation}
\langle x, y\rangle_M = x^T M y
\end{equation}
and so $M$ also induces a notion of distance between two points in the usual manner:
\begin{equation}
d_M(x, y) = ||x - y||_M = \sqrt{\langle x - y, x - y \rangle}.
\end{equation}

Now let $\metric$ be a symmetric positive-definite tensor field. As long as
$\metric$ is sufficiently smooth, then it also induces an inner product
(see \citet{simpson1994} for details). We say that a mesh $\cal{T}$ satisfies
$\metric$ if every edge has unit edge length with respect to its inner product;
$\cal{T}$ is a unit mesh with respect to this metric.

In order to see how the metric representing a mesh may be derived, consider the transformation from the reference element to the physical
element inherent in the finite element assembly procedure \citep{femtools}.
Let $K$ be an element of a linear simplicial mesh. $K$ is geometrically characterised by the affine map
$T_K: \hat{K} \rightarrow K$, where $\hat{K}$ is the reference isotropic element. Since
$T_K$ is assumed affine, we can write this transformation as
\begin{equation}
x = T_K(\hat{x}) = J_K\hat{x} + t_K,
\end{equation}
where $J_K$ is the Jacobian of the transformation. The metric for this element is derived
from the polar decomposition of the Jacobian, which yields
\begin{equation}
J_K = M_K Z_K
\end{equation}
where $M_K$ is a symmetric positive-definite matrix referred to as the metric,
and $Z_K$ is orthogonal \citep{micheletti2006}. The geometric properties of a simplicial mesh can be represented
as a piecewise constant (\Pzero) field over it. The same argument applies to
nonlinear or non-simplicial elements, but here the Jacobian varies over the element,
and so the metric does too.

For more background information and discussion on representing meshes in this manner,
see \citet{george1998}.

\section{Adaptive remeshing technology} \label{sec:adaptive_remeshing_technology}
Given a metric $\metric$, there are three main ways to generate a mesh which satisfies
that metric: global remeshing, local remeshing, and mesh optimisation. For a thorough discussion
of the relevant literature, see \citet{farrell2009i}.

Global remeshing is the act of generating an entirely new mesh of the same domain satisfying the sizing
specification, by means of an automatic mesh generator. This was first introduced in \citet{peraire1987} by incorporating
sizing and directionality inputs to an advancing front mesh generator.

By contrast, local remeshing is a mechanism of adaptive remeshing in which cavities of elements are removed
and the hole remeshed \citep{hassan1998}. These cavities are identified by measuring
their conformity to the given sizing specification.

Mesh optimisation is a mechanism of adaptive remeshing which deforms the previous mesh
to the adapted mesh by a sequence of local operations.
The mesh quality is measured by a functional measuring how close the mesh is to
the mesh encoded in the sizing specification. The local mesh modification
operations are applied successively to optimise this functional. This is the choice
adopted in \fluidity.

Mesh optimisation is available by default in three dimensions using the algorithm
of \citet{pain2001}. It is available in two dimensions using the \texttt{mba2d} algorithm of
\citet{vasilevskii1999} if \fluidity\ was configured with
the \configureflag{--enable-2d-adaptivity} flag; this is not enabled by default
as \texttt{mba2d} is licensed under the GPL, while \fluidity\ is licensed under the LGPL,
and so its default inclusion would cause licensing complications.

\section{Metric formation} \label{sec:metric_formation}
The flexibility of adaptive remeshing comes at a cost.
Adaptive remeshing is more complicated to guide than hierarchical or $p$-refinement.
To guide hierarchical or $p$-refinement, one needs element-level indicators
measuring the contribution of the element to some quantification of
the error. By contrast, the input to the adaptive remeshing algorithm
is a metric specifying the sizing and orientation of the desired output
mesh. This extra step of computing what mesh would (approximately)
give a desired target error is the characteristic challenge of guiding
the adaptive remeshing algorithm.

Most of the work to date has been guided by considerations of
interpolation error; this is the approach used in \fluidity. In one dimension, for interpolation
using $p^{th}$ order Lagrange polynomials, this is related to the
$(p+1)^{th}$ derivative of the function being interpolated; in multiple
dimensions, this extends to the tensor of $(p+1)^{th}$ order partial
derivatives of the function being interpolated. For linear interpolation
($p=1$), the interpolation error depends on the Hessian (the matrix
of second-order partial derivatives), which naturally induces
a metric in which to form a unit mesh.

Consider a field $u$ and its \Pone projection onto a mesh $\cal{T}$,
$\Pi_h u$. It can be shown \citep{frey2005} that
\begin{equation}
\epsilon = ||u - \Pi_h u||_{{\infty},K} \le c \cdot \max_e e^T \left| H \right| e,
\end{equation}
where $c$ is a constant independent of mesh size, $K$ is an element, $e$ ranges
over the edges of $K$, and $\left| H \right|$ is the absolute Hessian (the Hessian
with the absolute value function applied to its eigenvalues).
Therefore, one can write
\begin{equation}
\epsilon \approx c \cdot \max_e e^T \left| H \right| e,
\end{equation}
and so if $\epsilon$ is the target error in the $L_{\infty}$ norm,
\begin{equation}
1 \approx \max_e e^T \left( \frac{c}{\epsilon} \left| H \right| \right) e,
\end{equation}
i.e. the edges are of unit length with respect to the metric
$\frac{c}{\epsilon} \left| H \right|$.

Thus, in order to achieve a user-specified target interpolation error of $\epsilon$,
the metric $\frac{c}{\epsilon} \left| H \right|$ is used.

Metrics for norms other than the $L_{\infty}$ norm and interpolation functions
other than \Pone are given in \citet{huang2005}.

\subsection{Guidelines for choosing parameters which affect the metric construction}
A number of numerical options/parameters affect the construction of the metric 
which in turn greatly affect the adapted mesh produced by the optimisation procedure.

\subsubsection{Weights}
As described above a user-defined interpolation error must be chosen. This can also
be interpreted as an inverse weight, so a lower value means that one will obtain a
finer mesh. When adapting to a single field the value of this weight is the most
important factor in the mesh that will be returned. In the case when multiple
fields are being adapted to, and hence contribute to the single metric guiding
adaptivity, then the relative values of these weights will also impact on the 
mesh returned --- if one wants to place extra emphasis on adapting to one
field over another then simply set a lower value for the weight associated with
the first field.

It is possible to define the metric to be \emph{absolute} in which case the
weight described above will have the dimensions of the corresponding solution 
field. A useful rule-of-thumb it to start with a weighting of approximately 5\%
the magnitude of the solution field, if known --- a degree of trial and error
is somewhat inevitable here.

It is also possible for the metric to be \emph{relative} in which case the
weight should be chosen between zero and one. The 5\% rule would then involve
setting a value of $\epsilon=0.05$.

\subsubsection{Maximum and minimum element sizes}
For robustness of the mesh adaptivity procedure, and to limit refinement/coarsening of the
mesh it is possible to set maximum and minimum allowed element sizes. The input to
these quantities are tensors allowing one to impose different limits in different directions.
Assuming that these directions are aligned with the coordinate axes allows one to define
diagonal tensors.

There are both good and bad reasons that one may need to impose these constraints. The good
reasons are based on physics: a maximum size may be based on the size of the domain, or that
resolves a spatial scale that allows an instability to develop (the latter case would be better
handled with a more advanced a posteriori error measure of course), or simply a consequence of
the time/memory constraints of the machine the problem is running on. The bad reason is to control
the mesh when the above weights have been chosen poorly. Having said that it is often 
unavoidable that the weights and max/min element sizes will be chosen in tandem to achieve 
an appropriate mesh, especially for experienced users --- new users should be wary whenever maximum
and particularly minimum size constraints are actually hit in the mesh and as a first stage should
look to vary the weights to achieve the mesh they desire.

Finally, note that these constraints are achieved through manipulations to the metric, which in turn
controls an optimisation procedure. They are therefore not hard constraints and one may observe the
constraints being broken (slightly) in places.

\subsubsection{Maximum number of nodes}
Similar to the max/min element size constraint above, it is possible to limit the maximum number of nodes
that the mesh optimisation procedure returns. For reasons very similar to above this is dangerous
but somewhat inevitable. For new users altering the weights should be the primary way to control the
size of the adapted mesh.


\section{Interpolation} \label{sec:interpolation}
As the meshes produced by adaptive remeshing are in general entirely
unrelated, data must be interpolated from the old mesh to the new mesh
to proceed with the simulation.

There are two main approaches available in fluidity: consistent interpolation
and Galerkin projection. Consistent interpolation evaluates the old function
at the nodes of the new mesh. This is fast, but not conservative. It can be inaccurate,
and this approach is not well-defined for discontinuous fields.
Galerkin projection is more expensive (but not much more), is optimally accurate,
conservative, and well-defined for discontinuous fields \citep{farrell2009i}.
However, Galerkin projection is not bounded; a variant is available for \Pone fields
which approximately bounds the resulting interpolant \citep{farrell2009a}.

\section{Related topics} \label{sec:adaptivity_related_topics}
For a discussion of related topics such as Hessian recovery, gradation, and parallelisation,
see \citep{farrell2009i}.
