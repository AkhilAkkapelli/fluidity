\chapter{Parameterisations}\label{chap:parameterisations}

Although \fluidity is capable of resolving a range of scales dynamically using 
and adaptive mesh, it is not always feasible to resolve all process that are
required for a simulation. This chapter introduces the subgrid-scale paramateristations
that are available in \fluidity.

\section{General length scale subgridscale parameterisation}\label{Sect:GLS}
\index{viscosity!eddy}
\index{diffusivity!eddy}
\index{Reynolds stress}
\index{turbulence model}
\index{GLS|see{generic length scale model}}
\index{generic length scale model}

The general length scale (GLS) subgridscale parameterisation is capable of simulating vertical
turbulence at a scale lower than that of the mesh. There is no dependency on the mesh resolution, 
so is ideal for adaptive ocean-scale problems. GLS can be set-up to behave as a number of classical
turbulence models. \fluidity has four implementations of GLS available: $k-\epsilon$, $k-kl$, $k-\omega$, 
and an additional model based on \citet{umlauf2003}, the \emph{gen} model. 

Briefly, all implementations rely on a local, temporally varying, kinematic eddy 
viscosity $K_M$ that parametrises turbulence (local Reynolds stresses) in terms of mean-flow 
quantities (vertical shear) as, along 
with a buoyancy term that parameterises the kinematic eddy diffusivity, $K_H$:

\begin{equation}
\overline{u'w'} = -K_M\frac{\partial u}{\partial z},\;\;\;
\overline{v'w'} = -K_M\frac{\partial u}{\partial z},\;\;\;
\overline{w'\rho'} = -K_H\frac{\partial\rho}{\partial z}
\end{equation} 
with
\begin{equation}
K_M = \sqrt{k}lS_M+\nu_M^0, \;\;\;
K_H = \sqrt{k}lS_H+\nu_H^0
\label{eq:diff}
\end{equation}
where $u$ and $v$ are the horizontal components of Reynolds-averaged velocity along the $x$- and $y$-axes, $w$ is 
the vertical velocity along the vertical $z$-axis, positive upwards and $u'$, $v'$ and $w'$ are the components 
of the turbulent fluctuations about mean velocity. $\nu_H^0$ is the background diffusivity, $\nu_M^0$ 
is the background viscosity, $S_M$ and $S_H$ are stability functions, 
$k$ is the turbulent kinetic energy, and $l$ is a general length-scale.

The generic length scale turbulence closure model \citep{umlauf2003} is based in two equations, 
for the transport of turbulent kinetic energy (TKE, $k$) and a generic second quantity, $\Psi$. The TKE equation is:
\begin{equation}
\frac{\partial k}{\partial t} + U_i\frac{\partial k}{\partial x_i} = \frac{\partial}{\partial z}\left(\frac{K_M}{\sigma_k}\frac{\partial k}{\partial z}\right) + P + B - \epsilon 
\label{tke_one}
\end{equation}
\index{Schmidt number}
where $\sigma_k$ is the turbulence Schmidt number for $k$, $P$ and $B$ represent production by shear and buoyancy as:
\begin{equation}
P=-\langle u'w'\rangle\frac{\partial u}{\partial z}-\langle v'w' \rangle\frac{\partial v}{\partial z}=K_MM^2, \;\;\; M^2=\left(\frac{\partial U}{\partial z}\right)^2 + \left(\frac{\partial V}{\partial z}\right)^2 
\end{equation}
\begin{equation}
B=-\frac{g}{\rho_0}\langle\rho'w'\rangle=-K_HN^2, \;\;\; N^2= -\frac{g}{\rho_0}\frac{\partial\rho}{\partial z}
\end{equation}
where $N$ is the buoyancy frequency. The dissipation is modelled as:
\begin{equation}
\epsilon=\left(c_\mu^0\right)^{3+\frac{p}{n}}k^{\frac{3}{2}+\frac{m}{n}}\Psi^{-\frac{1}{n}}
\end{equation}
where $c_\mu^0$ is the stability coefficient based on experimental data for unstratified channel flow with a
log-layer solution. It takes on a specific value when used with a stability function and other model
parameters.

The second equation is 
\begin{equation}
\frac{\partial \Psi}{\partial t} + U_i\frac{\partial \Psi}{\partial x_i} = \frac{\partial}{\partial z}\left(\frac{K_M}{\sigma_\Psi}\frac{\partial \Psi}{\partial z}\right) + \frac{\Psi}{k}(c_1P + c_3B - c_2\epsilon F_{\mathrm{wall}})
\label{psi_one}
\end{equation}

The parameter $\sigma_\Psi$ is the Schmidt number for $\Psi$ and $c_i$ are constants based on experimental data. The value of $c_3$ depends on if the flow is stably stratified (in which case $c_3=c_3^-$) or unstable ($c_3=c_3^+$). Here,
\begin{equation}
\Psi=\left(c_\mu^0\right)^pk^ml^n
\end{equation}
and
\begin{equation}
l=\left(c_\mu^0\right)^3k^{\frac{3}{2}}\epsilon^{-1}
\end{equation}

The equation for $l$ can be derived from previous equations.

Choosing values for the parameters $p$, $m$, $n$, $\sigma_k$, $\sigma_\Psi$, $c_1$, $c_2$, $c_3$, $c_\mu^0$,
and $F_{\mathrm{wall}}$ one can recover the exact formulation of four GLS models incorporated into
\fluidity; the three standard GLS models,
$k-\epsilon$, $k-kl$, $k-\omega$, and the additional model based on \citet{umlauf2003}, 
the \emph{gen} model. Table \ref{tab:glsparams} shows the values of the pertinent parameters for each model.

\begin{table}[b]
\begin{center} 
\begin{tabular}{lllll}\hline
Model: & $k-kl$   & $k-\epsilon$                  & $k-\omega$                       & \emph{gen}       \\ 
$\Psi=$& $k^1l^1$ & $\left(c_\mu^0\right)^3k^{\frac{3}{2}}l^1$ & $\left(c_\mu^0\right)^{-1}k^{\frac{1}{2}}l^1$ & $\left(c_\mu^0\right)^2k^1l^{\frac{2}{3}}$  \\ \hline
$p$                   & 0.0            & 3.0          & -1.0         & 2.0          \\
$m$                   & 1.0            & 1.5          &  0.5         & 1.0          \\
$n$                   & 1.0            & -1.0         & -1.0         & -0.67        \\
$\sigma_k$            & 2.44           & 1.0          & 2.0          & 0.8          \\
$\sigma_\Psi$         & 2.44           & 1.3          & 2.0          & 1.07         \\
$c_1$                 & 0.9            & 1.44         & 0.555        & 1.0          \\
$c_2$                 & 0.5            & 1.92         & 0.833        & 1.22         \\
$c_3^+$               & 1.0            & 1.0          & 1.0          & 1.0          \\
$k_{\mathrm{min}}$    & $5.0\times10^{-6}$    & $7.6\times10^{-6}$  & $7.6\times10^{-6}$  & $7.6\times10^{-6}$  \\
$\Psi_{\mathrm{min}}$ & $1.0\times10^{-8}$    & $1.0\times10^{-12}$ & $1.0\times10^{-12}$ & $1.0\times10^{-12}$ \\
\end{tabular}
\end{center}
\caption{Generic length scale parameters}
\label{tab:glsparams}
\end{table}


\subsection{Stability functions}

Setting the parameters described above, \ie selecting which GLS model to use, closes the second-order 
moments, bar the definition of the stability functions, $S_M$ and $S_H$, 
which are a function of $\alpha_M$ and $\alpha_N$, defined as:
\begin{equation*}
\alpha_M=\frac{k^2}{\epsilon^2}M^2, \;\;\;
\alpha_N=\frac{k^2}{\epsilon^2}N^2,
\end{equation*}

The two stability can be defined as:
\begin{equation*}
S_M(\alpha_M,\alpha_N) = \frac{n_0+n_1\alpha_N+n_2\alpha_M}{d_0+d_1\alpha_N+d_2\alpha_M+d_3\alpha_N\alpha_M+d_4\alpha_N^2+d_5\alpha_M^2}
\end{equation*}
and
\begin{equation*}
S_H(\alpha_M,\alpha_N) = \frac{n_{b0}+n_{b1}\alpha_N+n_{b2}\alpha_M}{d_0+d_1\alpha_N+d_2\alpha_M+d_3\alpha_N\alpha_M+d_4\alpha_N^2+d_5\alpha_M^2}
\end{equation*}

However, using the equilibrium condition for the turbulent kinetic energy as $\frac{P+B}{\epsilon}=1$, one can write
$\alpha_M$ and a function of $\alpha_N$, allowing elimination of $\alpha_M$ in the above equations \citep{umlauf2005}:
\begin{equation*}
S_M(\alpha_M,\alpha_N)\alpha_M - S_N(\alpha_M,\alpha_N)\alpha_N=1
\end{equation*}
elementing some of the terms. A limit on negative values of $\alpha_N$ needs to applied to ensure $\alpha_M$ does not also become negative. 

The parameters $n_0$, $n_1$, $n_2$, $d_0$, $d_2$, $d_3$, $d_4$, $n_{b0}$, $n_{b1}$, $n_{b2}$ depend on
the model parameters chosen and can be related to traditional stability functions \citep{umlauf2005}.

\fluidity\ contains four choices of stability functions, GibsonLauder78 \citep{gibson1978}, KanthaClayson94 \citep{kantha1994}, CanutoA and CanutoB \citep{canuto2001}, each of which can be used in conjection with any of the four
GLS models implemented.



\section{Standard $k-\epsilon$ Turbulence Model}\label{Sect:kepsilon}
\index{viscosity!eddy}
\index{diffusivity!eddy}
\index{Reynolds stress}
\index{turbulence model}
\index{k-$\epsilon$ model}

Available under \option{\ldots/subgridscale\_parameterisations/k-epsilon}.

The widely-used $k-\epsilon$ turbulence model has been implemented in Fluidity based on the description given in
\citet{mathieu2000}. It is distinct from the $k-\epsilon$ option in the generic length scale (GLS) model (see
Section \ref{Sect:GLS}), in that it uses a 3D isotropic eddy-viscosity tensor, and it can be applied to any geometry.
The model is based on the unsteady Reynolds-averaged Navier-Stokes (RANS) equations, in which the velocity is decomposed into quasi-steady (moving average) and fluctuating (turbulent) components:

\begin{equation}\label{RANS}
\ppt[t]{u_i} + u_j \ppx[x_j]{u_i} = - \frac{1}{\rho } \ppx[x_j]{p} + \nu \frac{\partial^2 u_i}{\partial x_i \partial x_j} - \ppx[x_j]{u'_i u'_j},
\end{equation}
where $u_i$ are the components of the steady velocity, $u'_i$ are the fluctuating components, and $p$ is the steady pressure.
The third term on the right, containing the Reynolds stress tensor $u'_i u'_j$, represents the effect of turbulent fluctuations on the steady flow and is modelled as:

\begin{equation}\label{retens}
u'_i u'_j = - \frac 2 3 k \delta_{ij} + \nu_T \left (\ppx[x_j]{u_i} \ppx[x_i]{u_j} \right ),
\end{equation}
where $k=u'_i u'_j/2$ is the turbulent kinetic energy and $\nu(\bmx,t)$ is the eddy viscosity.
The equations are closed by solving transport equations for $k$ and the turbulent energy dissipation $\epsilon$:

\begin{equation}\label{k}
\ppt[t]{k} + u_i \ppx[x_i]{k} = \Pi - \epsilon + \ppx{} \left [ \frac{\nu_T}{\sigma_k} \ppx[x_i]{k} \right ],
\end{equation}

\begin{equation}\label{eps}
\ppt[t]{\epsilon} + u_i \ppx[x_i]{\epsilon} = C_{\epsilon1} \frac {\epsilon}{k} \Pi - C_{\epsilon2} \frac{\epsilon^2}{k} + \ppx{} \left [ \frac{\nu_T}{\sigma_\epsilon} \ppx[x_i]{\epsilon} \right ],
\end{equation}
where the isotropic tensor eddy viscosity $\nu_T$ and turbulent kinetic energy production $\Pi$ are approximated by the eddy viscosity hypothesis as:

\begin{equation}\label{nut}
\nu_T = C_\mu \frac{k^2}{\epsilon},
\end{equation}

\begin{equation}\label{Pi2}
\Pi = \nu_T \ppx[x_j]{u_i} : \left ( \ppx[x_j]{u_i} + \ppx[x_i]{u_j} \right ),
\end{equation}
where $:$ denotes the double-dot product, defined for two tensors $t, s$ as:

\begin{equation}\label{doubledot}
t : s = t_{11} s_{11} + t_{12} s_{21} + t_{13} s_{31} + t_{21} s_{12} + t_{22} s_{22} + t_{23} s_{32} + t_{31} s_{13} + t_{32} s_{23} + t_{33} s_{33}.
\end{equation}

A lengthscale is associated with the dissipation of turbulent kinetic energy by molecular viscosity. It is defined as:

\begin{equation}\label{lengthscale}
l = k^{1.5} / \epsilon.
\end{equation}

To ensure the correct behaviour near walls, a lengthscale limiter has been included in the model (see \citet{yap1987, craft1996}):

\begin{equation}\label{yap}
l = \text{max} \left ( 0.83 \frac{\epsilon^2}{k} \left [ \frac{k^{1.5}}{2.5 \epsilon y} -1 \right ] \left [ \frac{k^{1.5}}{2.5 \epsilon y} \right ]^2, 0 \right ),
\end{equation}
where $y$ is the wall-normal distance. The limiter, if switched on, operates only in the wall-adjacent elements.

The five model coefficients are in Table \ref{tab:kepsco}. These are the default values but they can be changed in Diamond.

\begin{table}[hb]
\begin{center} 
\begin{tabular}{ll}\hline
Coefficient & Standard value \\ \hline
$C_\mu$ & 0.09 \\
$C_{\epsilon1}$ & 1.44 \\
$C_{\epsilon2}$ & 1.92 \\
$\sigma_\epsilon$ & 1.3 \\
$\sigma_k$ & 1.0 \\ \hline
\end{tabular}
\end{center}
\caption{$k-\epsilon$ model coefficients}
\label{tab:kepsco}
\end{table}

\section{Large-Eddy Simulation (LES)}
\index{large eddy simulation}
\index{turbulence model}

Large-eddy simulation filters the velocity field into resolved and sub-filter-scale components. The eddy viscosity assumption is that the unresolved scales drain energy from the resolved scales in the manner of an extra turbulent viscosity $\nu_\tau$. Then \eqref{mtm} is written:

\begin{equation}
\frac{\partial \tilde u_i}{\partial t} + \tilde u_j \frac{\partial \tilde u_i}{\partial x_j}
 = -\frac 1 \rho \frac{\partial \tilde p}{\partial x_i}
 + \frac{\partial}{\partial x_j} \left [ (\nu + \nu_\tau) \left ( \frac{\partial \tilde u_i}{\partial x_j} + \frac{\partial \tilde u_j}{\partial x_i} \right ) \right ],
\end{equation}

Two LES models are available in Fluidity.

\subsection{Second-Order Dissipation}

The following model developed by \citet{bentham2003} is similar to the original Smagorinsky model \citep{smagorinsky1963general}, but allows for an anisotropic eddy viscosity that gives better results for flow simulations on unstructured grids:

\begin{equation}
\frac{\partial \tau_{ij}}{\partial x_j} = \frac{\partial}{\partial x_i} \left [ \nu_{jk}\frac{\partial u_j}{\partial x_k} \right ],
\end{equation}
with the anisotropic tensorial eddy viscosity:

\begin{equation}
\nu_{jk} = 4C_S^2 \left | \mathcal{\tilde S} \right | \mathcal{M}^{-1},
\end{equation}
where $\mathcal{M}$ is the length-scale metric from the adaptivity process (see \citet{pain2001}), used here to relate eddy viscosity to the local grid size. The factor of 4 arises because the filter width separating resolved and unresolved scales is twice the local element size, which is squared in the viscosity model. $C_S \approx 0.1$ is the Smagorinsky coefficient, and $\left | \mathcal{\tilde S} \right |$ is the local strain rate:

\begin{equation}
\left | \mathcal{\tilde S} \right | = (2 \tilde S_{ij} \tilde S_{ij})^{1/2},
\end{equation}
where:

\begin{equation}
\tilde S_{ij} = \frac 1 2 \left ( \frac{\partial \tilde u_i}{\partial x_j} + \frac{\partial \tilde u_j}{\partial x_i} \right ).
\end{equation}

\subsection{Fourth-Order Dissipation}

The fourth-order method is designed as an improvement to the second-order eddy viscosity method, which can be too dissipative \citep{adam}. The fourth-order term is taken as the difference of two second-order eddy viscosity discretisations, where one is larger than the other. Usually a smaller time-step and finer grid are necessary to make fourth-order worthwhile.
