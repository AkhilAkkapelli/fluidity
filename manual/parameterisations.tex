\chapter{Parameterisations}\label{chap:parameterisations}

Although \fluidity\ is capable of resolving a range of scales dynamically using
an adaptive mesh, it is not always feasible to resolve all processes and spatial scales that are
required for a simulation, and therefore some form of parameterisation is required.
This chapter introduces the paramateristations that are available in \fluidity.

\section{Turbulent flow modelling and simulation}
\label{sec:turbulence_parametrisations}

\subsection{Reynolds Averaged Navier Stokes (RANS) Modelling}
\label{sec:RANS}

\subsubsection{Generic length scale turbulence parameterisation}\label{sec:GLS}
\index{viscosity!eddy}
\index{diffusivity!eddy}
\index{Reynolds stress}
\index{turbulence model}
\index{GLS|see{generic length scale model}}
\index{generic length scale model}

The generic length scale (GLS) turbulence parameterisation is capable of simulating \emph{vertical}
turbulence at a scale lower than that of the mesh. There is no dependency on the mesh resolution,
so is ideal for adaptive ocean-scale problems. GLS has the additional advantage that it
can be set-up to behave as a number of classical turbulence models:
$k-\epsilon$, $k-kl$, $k-\omega$,
and an additional model based on \citet{umlauf2003}, the $gen$ model. Further details are
available in \citet{hill2012}, along with test cases and a demonstration of how GLS can be
used in conjuction with adaptive remeshing.

Briefly, all implementations rely on a local, temporally varying, kinematic eddy
viscosity $K_M$ that parametrises turbulence (local Reynolds stresses) in terms of mean-flow
quantities (vertical shear) as, along
with a buoyancy term that parameterises the kinematic eddy diffusivity, $K_H$:

\begin{equation}
\overline{u'w'} = -\nu_M\frac{\partial u}{\partial z},\quad
\overline{v'w'} = -\nu_M\frac{\partial v}{\partial z},\quad
\overline{w'\rho'} = -\nu_H\frac{\partial\rho}{\partial z},
\end{equation}
with
\begin{equation}
\nu_M = \sqrt{k}lS_{M}+\nu_M^0, \quad
\nu_H = \sqrt{k}lS_{H}+\nu_H^0,
\label{eq:diff}
\end{equation}
Here, we follow the notation of \citet{umlauf2003}, where $u$ and $v$ are the 
horizontal components of the Reynolds-averaged velocity along the $x$- and $y$-axes, 
$w$ is the vertical velocity along the vertical $z$-axis, positive upwards, and 
$u'$, $v'$ and $w'$ are the components of the turbulent fluctuations about the 
mean velocity. $\nu_H^0$ is the background diffusivity, $\nu_M^0$ is the background viscosity, 
$S_{M}$ and $S_{H}$ are often referred to as stability functions, $k$ is the turbulent kinetic energy,
and $l$ is a length-scale. When using GLS the values of $\nu_M$ and $\nu_H$ become the 
vertical components of the tensors $\tautens$ and $\kaptens_T$ 
in equation \ref{boussinesq} respectively. Other tracer fields, such as 
salinity use the same diffusivity as temperature, i.e. $\kaptens_T = \kaptens_S$.

The generic length scale turbulence closure model \citep{umlauf2003} is based
on two equations, for the transport of turbulent kinetic energy (TKE) and a
generic second quantity, $\Psi$. The TKE equation is:
\begin{equation}
\frac{\partial k}{\partial t} + \bmu_i\frac{\partial k}{\partial x_i} =
\frac{\partial}{\partial z}\left(\frac{\nu_M}{\sigma_k}\frac{\partial k}{\partial z}\right) + P + B - \epsilon,
\label{eq:tke_one}
\end{equation}

\noindent 
where $\sigma_k$ is the turbulence Schmidt number for $k$, $u_i$ are the
velocity components ($u$, $v$ and $w$ in the $x$, $y$ and $z$ directions 
respectively), and $P$ and $B$ represent production by shear and buoyancy
which are defined as:
\begin{equation}
P=-\overline{u'w'}\frac{\partial u}{\partial z}-\overline{v'w'}
\frac{\partial v}{\partial z}=\nu_MM^2, \quad M^2=
\left(\frac{\partial u}{\partial z}\right)^2 + \left(\frac{\partial v}{\partial z}\right)^2,
\end{equation}
\begin{equation}
B=-\frac{g}{\rho_0}\overline{\rho'w'}=-\nu_HN^2, \quad N^2= -\frac{g}{\rho_0}\frac{\partial\rho}{\partial z}
\end{equation}

Here $N$ is the buoyancy frequency; $u$ and $v$ are the horizontal velocity 
components. The dissipation is modelled using a rate of dissipation term:
\begin{equation}
\epsilon=\left(c_\mu^0\right)^{3+\frac{p}{n}}k^{\frac{3}{2}+\frac{m}{n}}\Psi^{-\frac{1}{n}}
\end{equation}
where $c_\mu^0$ is a model constant used to make $\Psi$ identifiable with any of the transitional
models, e.g. $kl$, $\epsilon$, and $\omega$, by adopting the values shown in Table \ref{tab:glsparams} \citep{umlauf2003}.

There is also the option to add an extra term to account for various
oceanic parameters, such an internal waves breaking. This is based on the NEMO
ocean model and takes a user-defined percentage of the surface $k$ and adds it down-depth
using an exponential profile:
\begin{equation}
    k_z = {k_z}_o + p*k_{\mathrm{sur}} * \exp{\left( -z / l_k \right)}
\end{equation}
where $k_z$ is the new turbulent kinetic energy value at depth, $z$, ${k_z}_o$ is the original TKE, $k_{\mathrm{sur}}$
is the surface TKE, $p$ is a constant, and $l_k$ is a lengthscale. The options for this can be found in
\option{\ldots/subgridscale\_parameterisations/gls/ocean\_parameterisation}.

The second equation is:
\begin{equation}
\frac{\partial \Psi}{\partial t} + \bmu_i\frac{\partial \Psi}{\partial x_i} =
\frac{\partial}{\partial z}\left(\frac{\nu_M}{\sigma_\Psi}\frac{\partial \Psi}{\partial z}\right) +
\frac{\Psi}{k}(c_1P + c_3B - c_2\epsilon F_{wall}),
\label{eq:psi_one}
\end{equation}
\noindent
The parameter $\sigma_\Psi$ is the Schmidt number for $\Psi$ and $c_i$ are
constants based on experimental data. The value of $c_3$ depends on whether the flow
is stably stratified (in which case $c_3=c_3^-$) or unstable ($c_3=c_3^+$).
Here,
\begin{equation}
\Psi=\left(c_\mu^0\right)^pk^ml^n,
\label{eq:psi}
\end{equation}
and
\begin{equation}
l=\left(c_\mu^0\right)^3k^{\frac{3}{2}}\epsilon^{-1},
\end{equation}

By choosing values for the parameters $p$, $m$, $n$, $\sigma_k$, $\sigma_\Psi$,
$c_1$, $c_2$, $c_3$, and $c_\mu^0$ one can recover the
exact formulation of three standard GLS models, $k-\epsilon$, $k-kl$ (equivalent of the Mellor-Yamada formulation),
$k-\omega$, and an additional model based on \citet{umlauf2003}, the \emph{gen}
model (see Tables \ref{tab:glsparams} and \ref{tab:c3minus} for values).

\begin{table}[b]
\begin{center}
\begin{tabular}{lllll}\hline
Model: & $k-kl$   & $k-\epsilon$                  & $k-\omega$                       & \emph{gen}       \\
$\Psi=$& $k^1l^1$ & $\left(c_\mu^0\right)^3k^\frac{3}{2}l^1$ & $\left(c_\mu^0\right)^{-1}k^\frac{1}{2}l^1$ & $\left(c_\mu^0\right)^2k^1l^\frac{2}{3}$  \\ \hline
$p$                   & 0.0            & 3.0          & -1.0         & 2.0          \\
$m$                   & 1.0            & 1.5          &  0.5         & 1.0          \\
$n$                   & 1.0            & -1.0         & -1.0         & -0.67        \\
$\sigma_k$            & 2.44           & 1.0          & 2.0          & 0.8          \\
$\sigma_\Psi$         & 2.44           & 1.3          & 2.0          & 1.07         \\
$c_1$                 & 0.9            & 1.44         & 0.555        & 1.0          \\
$c_2$                 & 0.5            & 1.92         & 0.833        & 1.22         \\
$c_3^-$               & See Table \ref{tab:c3minus}\\
$c_3^+$               & 1.0            & 1.0          & 1.0          & 1.0          \\
$k_{\mathrm{min}}$    & $5.0\times10^{-6}$    & $7.6\times10^{-6}$  & $7.6\times10^{-6}$  & $7.6\times10^{-6}$  \\
$\Psi_{\mathrm{min}}$ & $1.0\times10^{-8}$    & $1.0\times10^{-12}$ & $1.0\times10^{-12}$ & $1.0\times10^{-12}$ \\
$F_{wall}$            & See sec \ref{sec:wall_functions} & 1.0 & 1.0 & 1.0 \\
\end{tabular}
\end{center}
\caption{Generic length scale parameters}
\label{tab:glsparams}
\end{table}

\begin{table}[b]
\begin{center}
\begin{tabular}{lllll}\hline
Model: & $k-kl$   & $k-\epsilon$                  & $k-\omega$                       & \emph{gen}       \\
$\Psi=$& $k^1l^1$ & $\left(c_\mu^0\right)^3k^\frac{3}{2}l^1$ & $\left(c_\mu^0\right)^{-1}k^\frac{1}{2}l^1$ & $\left(c_\mu^0\right)^2k^1l^\frac{2}{3}$  \\ \hline
KC                   & 2.53            & -0.41          & -0.58         & 0.1         \\
CA                   & 2.68            & -0.63          & -0.64         & 0.05        \\
CB                   &  -              & -0.57          & -0.61         & 0.08        \\
GL                   &  -              & -0.37          & -0.492        & 0.1704      \\
\end{tabular}
\end{center}
\caption{Values for the $c_3^+$ parameter for each combination of closure scheme and stability function}
\label{tab:c3minus}
\end{table}


\par{\textbf{Wall functions}\\}
\label{sec:wall_functions}

The $k-kl$ closure scheme requires that a wall function as the value of $n$ is positive (see \citet{umlauf2003}). There
are four different wall functions enabled in \fluidity. In standard Mellor-Yamada models \citep{mellor1982}, $F_{wall}$ is defined as:
\begin{equation}
F_{wall} = \left(1+E_2 \left(\frac{l}{\kappa} \frac{d_b + d_s}{d_b d_s}\right)^2\right)
\end{equation}

\noindent
where $E_2=1.33$, and $d_s$ and $d_b$ are the distance to the surface and bottom respectively.

An alternative suggestion by \citet{burchard1998} suggests a symmetric linear shape:
\begin{equation}
F_{wall} = \left(1+E_2 \left(\frac{l}{\kappa} \frac{1}{MIN\left(d_b,d_s\right)}\right)^2\right)
\end{equation}

\citet{burchard2001} used numerical experiments to define a wall function simulating an infinitely deep
basin:
\begin{equation}
F_{wall} = \left(1+E_2 \left(\frac{l}{\kappa} \frac{1}{d_s}\right)^2\right)
\end{equation}

Finally, \citet{blumberg1992} suggested a correction to the wall function for open channel flow:
\begin{equation}
F_{wall} = \left(1+E_2 \left(\frac{l}{\kappa d_b}\right)^2 + E_4 \left(\frac{l}{\kappa d_s}\right)^2\right)
\end{equation}

\noindent
where $E_4=0.25$.

\par{\textbf{Stability functions}\\}
Setting the parameters described above, \ie selecting which GLS model to use, closes the second-order
moments, bar the definition of the stability functions, $S_M$ and $S_H$,
which are a function of $\alpha_M$ and $\alpha_N$, defined as:
\begin{equation*}
\alpha_M=\frac{k^2}{\epsilon^2}M^2, \quad
\alpha_N=\frac{k^2}{\epsilon^2}N^2.
\end{equation*}

The two stability can be defined as:
\begin{equation*}
S_M(\alpha_M,\alpha_N) = \frac{n_0+n_1\alpha_N+n_2\alpha_M}{d_0+d_1\alpha_N+d_2\alpha_M+d_3\alpha_N\alpha_M+d_4\alpha_N^2+d_5\alpha_M^2},
\end{equation*}
and
\begin{equation*}
S_H(\alpha_M,\alpha_N) = \frac{n_{b0}+n_{b1}\alpha_N+n_{b2}\alpha_M}{d_0+d_1\alpha_N+d_2\alpha_M+d_3\alpha_N\alpha_M+d_4\alpha_N^2+d_5\alpha_M^2}.
\end{equation*}

However, using the equilibrium condition for the turbulent kinetic energy as $(P+B)/\epsilon=1$, one can write
$\alpha_M$ and a function of $\alpha_N$, allowing elimination of $\alpha_M$ in the above equations \citep{umlauf2005}:
\begin{equation*}
S_M(\alpha_M,\alpha_N)\alpha_M - S_N(\alpha_M,\alpha_N)\alpha_N=1
\end{equation*}
eliminating some of the terms. A limit on negative values of $\alpha_N$ needs to applied to ensure $\alpha_M$ does not also become negative.

The parameters $n_0$, $n_1$, $n_2$, $d_0$, $d_2$, $d_3$, $d_4$, $n_{b0}$, $n_{b1}$, $n_{b2}$ depend on
the model parameters chosen and can be related to traditional stability functions \citep{umlauf2005}.

\fluidity\ contains four choices of stability functions, GibsonLauder78 \citep{gibson1978}, KanthaClayson94 \citep{kantha1994}, CanutoA and CanutoB \citep{canuto2001},
each of which can be used in conjunction with any of $gen$, $k-\epsilon$, and $k-\omega$ closures; and CanutoA and KanthaClayson94 available with the $k-kl$ 
closure scheme.

\par{\textbf{Boundary conditions}\\}
The boundary conditions for the two GLS equations can be either of Dirichlet or Neumann
type. For the turbulent kinetic energy, the Dirichlet condition can be written as:
\begin{equation}
k=\frac{\left(u^*\right)^2}{\left(c_\mu^0\right)^2},
\label{eq:k_dir_bc}
\end{equation}
where $u^*$ is the friction velocity. However, as the viscous sublayer is not
resolved, the Dirichlet condition can be unstable unless the resolution at the
boundary is very high \citep{burchard1999}. It is therefore advisable to use
the Neumann condition:
\begin{equation}
\frac{\nu_M}{\sigma_k}\frac{\partial k}{\partial z} = 0.
\end{equation}

Similarly for the generic quantity, $\Psi$, the Dirichlet condition is written
as:
\begin{equation}
\Psi=\left(c_\mu^0\right)^{p}l^nk^m
\end{equation}

At the top of the viscous sublayer the value of $\Psi$ can be determined from equation \ref{eq:psi}, 
specifying $l=\kappa z$ and $k$ from equation \ref{eq:k_dir_bc}, giving:
\begin{equation}
\Psi=\left(c_\mu^0\right)^{p-2m}\kappa^n\left(u_s^*\right)^{2m}\left(\kappa z_s\right)^n
\end{equation}
where $z_s$ is the distance from the boundary surface (either surface or bottom) and $u_s^*$ is the 
friction at the surface or bottom respectively.

Calculating the corresponding Neumann  conditions by differentiating with respect to $z$, yields:
\begin{equation}
\left(\frac{K_M}{\sigma_{\Psi}}\frac{\partial\Psi}{\partial z}\right) = n\frac{K_M}{\sigma_{\Psi}} \left(c_\mu^0\right)^p k^m \kappa^n z_s^{n-1}
\label{eq:psi-flux-bc}
\end{equation}

Note that it is also an option to express the Neumann condition above in terms
of the friction velocity, $u^*$. Previous work has shown this causes numerical
difficulties in the case of stress-free surface boundary layers \citep{burchard1999}.


\subsubsection{Standard $k-\epsilon$ Turbulence Model}\label{sec:kepsilon}
\index{viscosity!eddy} \index{diffusivity!eddy} \index{Reynolds stress} \index{turbulence
  model} \index{k-$\epsilon$ model}

Available under \option{\ldots/subgridscale\_parameterisations/k-epsilon}.

The widely-used $k-\epsilon$ turbulence model has been implemented in \fluidity\ based on
the descriptions given in \citet{wilcox1998turbulence} and \citet{Rodi1993}. It is
distinct from the $k-\epsilon$ option in the generic length scale (GLS) model (see Section
\ref{sec:GLS}), in that it uses a 3D eddy-viscosity tensor and can be applied to any
geometry. The eddy viscosity is added to the user-specified molecular (background)
viscosity when solving for velocity, and if solving for additional prognostic scalar
fields, it is scaled by a user-specified Prandtl number to obtain the field eddy
diffusivity.

The model is based on the unsteady Reynolds-averaged Navier-Stokes (RANS) equations, in
which the velocity is decomposed into quasi-steady (moving average) and fluctuating
(turbulent) components:

%% \begin{equation}\label{RANS}
%%  \rho\left(\frac{\partial\bmu}{\partial{t}} + \bmu\cdot\nabla\bmu\right) = \nabla\cdot\tautens - \nabla p +\rho\bmF,
%% \end{equation}

\begin{equation}\label{eq:RANS}
  \rho\frac{\partial \mathbf{u}}{\partial t} + \rho\mathbf{u}\cdot\nabla\mathbf{u} =
  -\nabla p + \rho\mathbf{g} + \nabla\cdot\tautens + \nabla\cdot\left(-\rho\mathbf{u}'\mathbf{u}'\right),
\end{equation}
where $\mathbf{u}$ is the steady velocity, $\mathbf{u}'$ is the fluctuating velocity, and
$p$ is the steady pressure.  The fourth term on the right, containing the Reynolds stress
tensor $-\rho\mathbf{u}'\mathbf{u}'$, represents the effect of turbulent fluctuations on
the steady flow and is modelled as:

\begin{equation}\label{eq:R_tensor_compressible}
  -\rho\mathbf{u}'\mathbf{u}' = \tautens_R = \mu_T \left( \nabla\mathbf{u} + \left(\nabla\mathbf{u}\right)^{\mathrm{T}} - \frac{1}{3}\left(\nabla\cdot\mathbf{u}\right)\mathbf{I} \right) - \frac{2}{3} k\rho\mathbf{I},
\end{equation}
where $k=(\mathbf{u}'\cdot\mathbf{u}')/2$ is the turbulent kinetic energy and
$\mu_T(\mathbf{x},t)$ is the dynamic eddy viscosity. For incompressible flow this becomes:
\begin{equation}\label{eq:R_tensor_incompressible}
  -\rho\mathbf{u}'\mathbf{u}' = \tautens_R = \mu_T \left( \nabla\mathbf{u} + \left(\nabla\mathbf{u}\right)^{\mathrm{T}} \right) - \frac{2}{3} k\rho\mathbf{I},
\end{equation}
The eddy viscosity is estimated as:
\begin{equation}\label{eq:nut}
  \mu_T = \rho C_\mu \frac{k^2}{\epsilon},
\end{equation}
where $\epsilon$ is the turbulent dissipation. The equations are closed by solving
transport equations for $k$ and the turbulent energy dissipation $\epsilon$:
\begin{equation}\label{eq:k}
  \rho\frac{\partial k}{\partial t} + \rho\mathbf{u}\cdot\nabla k = \nabla\cdot\left(\left(\mu + \frac{\mu_T}{\sigma_k}\right) \nabla k \right) + \tautens_R \cdot \nabla\mathbf{u} - \frac{\mu_T}{\rho\sigma_T} \, \mathbf{g}\cdot\nabla\rho - \rho \epsilon ,
\end{equation}
\begin{equation}\label{eq:eps}
  \rho\frac{\partial \epsilon}{\partial t} + \rho\mathbf{u}\cdot\nabla \epsilon = \nabla\cdot\left(\left(\mu + \frac{\mu_T}{\sigma_\epsilon}\right) \nabla{\epsilon} \right) + C_{\epsilon1} \left( \frac{\epsilon}{k} \right) \left( \tautens_R \cdot \nabla\mathbf{u} - C_{\epsilon3} \, \frac{\mu_T}{\rho\sigma_T} \, \mathbf{g}\cdot\nabla\rho \right)- C_{\epsilon2} \rho \frac{\epsilon^2}{k} ,
\end{equation}
The right hand side terms in the $k$ and $\epsilon$ equations relate to the diffusion,
production, production/destruction due to buoyancy, and destruction of $k$ and $\epsilon$.
Within the buoyancy term, $\mathbf{g}$ is the gravitational vector, $\sigma_T$ is the
Prandtl or Schmidt number for the Density field, and $C_{\epsilon3}$ varies depending on
the direction of the flow with respect to gravity.  This is approximated to be:
\begin{equation}\label{eq:Cepsilon3}
  C_{\epsilon3} = \mathrm{tanh}\left ( \frac{u_z}{u_{xy}}  \right ),
\end{equation}
where $u_z$ is the magnitude of the velocity in the same direction as gravity and $u_{xy}$
is the magnitude of the velocity in all other directions.

The last term in equations \ref{eq:R_tensor_compressible} and \ref{eq:R_tensor_incompressible} is not added explicitly. This term is determined during the conservation equation solve and is absorbed into the calculated pressure gradient. This means that the pressure gradient calculated by the model is actually:
\begin{equation}\label{eq:Modified_pressure}
  \nabla p' = \nabla \left( p + \frac{2}{3} k \right).
\end{equation}
The pressure field will therefore no longer be the pressure, but instead a modified pressure $p'$. The real pressure can be obtained by subtracting $\frac{2}{3} k$.

A turbulence length scale is associated with the dissipation of turbulent kinetic
energy by the subgrid scale motions:

\begin{equation}\label{lengthscale}
l = \frac{k^{3/2}}{\epsilon}.
\end{equation}

The five model coefficients are in Table \ref{tab:kepsco}. These are the default values
but they can be changed in Diamond.

\begin{table}[hb]
  \begin{center}
    \begin{tabular}{ll}\hline
      Coefficient & Standard value \\ \hline
      $C_\mu$ & 0.09 \\
      $C_{\epsilon1}$ & 1.44 \\
      $C_{\epsilon2}$ & 1.92 \\
      $\sigma_\epsilon$ & 1.3 \\
      $\sigma_k$ & 1.0 \\
      $\sigma_T$ & 1.0 \\ \hline
    \end{tabular}
  \end{center}
  \caption{$k-\epsilon$ model coefficients}
  \label{tab:kepsco}
\end{table}

\par{\textbf{Important notes on applying the model in Fluidity}}
\begin{itemize}
\item The background viscosity must be set as an \option{anisotropic\_symmetric} tensor,
  with all values set equal to the isotropic viscosity.
\item The velocity stress terms should be in partial-stress form, which is set under
  \option{../vector\_field::Velocity/prognostic/spatial\_discretisation/../stress\_terms/}.
\item If using anything other than a P1 Velocity mesh, mass-lumping does not work when
  calculating the source and absorbtion terms for the
  model. \option{../k-epsilon/mass\_lumping\_in\_diagnostics/solve\_using\_mass\_matrix}
  must be selected.
\end{itemize}

For more notes on usage see \ref{sec:kepsilon_usage}

\par{\textbf{Low Reynolds number model}\\}
When simulating low Reynolds numbers ($Re<10^4$) it is recommended that the low-Re
k-epsilon model is used. \option{\ldots/boundary\_conditions/type::k\_epsilon/Low\_Re/}
boundary conditions should be selected for both the $k$ and $\epsilon$ fields, for all
solid boundaries, and the \option{DistanceToWall} field must be set. The distance to the
closest solid boundary is a prescribed field and can be described using a python function
for simple geometries. For more complex geometries where this is not possible an estimate
of the distance to the closest wall can be generated using a Poisson's, or Eikonal,
equation, as described in \citet{Tucker2011}.

The Lam and Bremhorst low-reynolds RANS model is implemented in Fluidity, as detailed in
\citet{wilcox1998turbulence}. Damping functions are applied to the equations \ref{eq:eps}
and \ref{eq:nut} as:
\begin{equation}\label{eq:nut_lowRe}
  \mu_T = \rho C_\mu f_\mu \frac{k^2}{\epsilon},
\end{equation}
\begin{equation}\label{eq:eps_lowRe}
  \rho\frac{\partial \epsilon}{\partial t} + \rho\mathbf{u}\cdot\nabla \epsilon = \nabla\cdot\left(\frac{\mu_T}{\sigma_\epsilon} \nabla{\epsilon} \right) + C_{\epsilon1} f_1 \frac{\epsilon}{k} \left( \tautens_R \cdot \nabla\mathbf{u} + C_{\epsilon3} \, \frac{\mu_T}{\sigma_T} \, \mathbf{g}\cdot \beta \nabla c \right)- C_{\epsilon2} \rho f_2 \frac{\epsilon^2}{k},
\end{equation}
where:

\begin{equation}\label{eq:f_mu}
  f_\mu = \left( 1 - e^{-0.0165 \, R_y} \right)^2 \left( 1 + 20.5/Re_T \right),
\end{equation}
\begin{equation}\label{eq:f_1}
  f_1 = 1 + \left( 0.05/f_\mu \right)^3,
\end{equation}
\begin{equation}\label{eq:f_2}
  f_2 = 1 - e^{-Re_T^2},
\end{equation}
\begin{equation}\label{eq:R_y}
  R_y = \frac{\rho k^{1/2}y}{\mu},
\end{equation}
\begin{equation}\label{eq:Re_T}
  Re_T = \frac{\rho k^2}{\epsilon \mu},
\end{equation}
Additionally:
\begin{itemize}
\item $f_\mu$ is limited to a maximum value of 1.0.
\item For stability, $f_1$ and $f_2$ are limited to a value set under
  \option{\ldots/max\_damping\_value/}.
\end{itemize}

The associated boundary conditions are:
\begin{align}
  \bmu &= 0, \label{eq:ubc1}\\
  k &= 0, \label{eq:kbc1}\\
  \epsilon &= \frac{\mu}{\rho} \ppxx[y]{k} = \frac{2\mu}{\rho} \left ( \ppx[n]{k^{1/2}}
  \right ). \label{eq:epsbc1}
\end{align}

\par{\textbf{Turbulent diffusivity of scalar fields}\\}
When using scalar fields, such as a temperature or sediment field, The k-$\epsilon$ model
simulates subgridscale eddies by increasing the diffusivity of the scalar fields such that
the diffusivity tensor, $\kaptens$, in the advection diffusion
equation~\ref{eq:general_scalar_eqn} becomes:

\begin{equation}\label{eq:keps_diff}
  \kaptens = D + \mu_T/\sigma_T
\end{equation}

where $D$ is the background diffusivity tensor for the fluid or the scalar field, and
$\sigma_T$ is the Prandtl number, $Pr_T$, for temperature fields or the Schmidt number,
$Sc_T$, for massive fields.

The Prandtl number is the ratio of momentum diffusivity (eddy viscosity) to the eddy
thermal diffusivity. The Schmidt number is the ratio of momentum diffusivity to the
diffusivity of mass.

\par{\textbf{Time discretisation and coupling}\\}
The $k$ and $\epsilon$ equations are coupled and are highly non-linear. In Fluidity, the
equations are linearised and decoupled using available values from previous iterations
and time steps as follows.  

Being consistent with all advection diffusion equations in Fluidity, the density and
velocity values used are defined by non-linear relaxation of the available values for
these variables as defined in section \ref{sec:relax}, which are denoted as
$\hat{\mathbf{u}} = \mathbf{u}^{n+\theta_{nl}}$ and $\hat{\rho} = \rho^{n+\theta_{nl}}$
for the remainder of this section.

Two discretisations are used for $k$ and $\epsilon$. % The time-discretisation of the
% advection diffusion equations for $k$ and $\epsilon$ differs from the discretisation
% described in section\ref{sec:ND_time_disct_adv_diff} due to 
The coupled, non-linear source terms and eddy viscosity are calculated using a
discretisation of $k$ and $\epsilon$ similar to the non-linear relaxation of $\mathbf{u}$
and $\rho$. i.e. based upon values from the previous non-linear iteration,
$\tilde{\gamma}^{n+1}$ (where $\gamma$ implies either $k$ or $\epsilon$), and values from
the previous time step $\gamma^n$. The choice between these values is made using the
$k-\epsilon$ nonlinear relaxation parameter, $\theta_{k\epsilon}$, which must lie in the
interval [0, 1]. This allows us to define:

\begin{equation}
  \hat{\gamma} = \gamma^{n+\theta_{k\epsilon}} =  \theta_{k\epsilon} \tilde{\gamma}^{n+1} + \left( 1.0 - \theta_{k\epsilon} \right) \gamma^n .
\end{equation}

Other terms in these equations are discretised using the $\theta$ scheme described in
section~\ref{sec:ND_time_theta_scheme}, as in
section~\ref{sec:ND_time_disct_adv_diff}. These values are denoted as $\bar{\gamma}$
throughout the remainder of this section.

Equations \ref{eq:R_tensor_compressible} to \ref{eq:Cepsilon3} therefore become:

\begin{equation}\label{eq:R_tensor_compressible_td}
  \hat{\tautens_{R}} = \hat{\mu_T} \left( \nabla\hat{\mathbf{u}} + \left( \nabla\hat{\mathbf{u}} \right)^{\mathrm{T}} - \frac{1}{3}\left( \nabla\cdot\hat{\mathbf{u}} \right)\mathbf{I} \right) - \frac{2}{3} \hat{k} \hat{\rho} \mathbf{I},
\end{equation}

\begin{equation}\label{eq:nut_td}
  \hat{\mu_T} = \hat{\rho} C_\mu \frac{\hat{k}^2}{\hat{\epsilon}}
\end{equation}
\noindent $\hat{\mu_T}$ is evaluated both at the beginning of each non-linear iteration
and also before the momentum solve so that the most up to date value is used for all
equation solves.

\begin{equation}\label{eq:k_td}
  \hat{\rho} \frac{\partial k}{\partial t} + \hat{\rho}
  \hat{\mathbf{u}} \cdot \nabla \bar{k} =
  \nabla\cdot\left(\left(\mu + \frac{\hat{\mu_T}}{\sigma_k}\right) \nabla
    \bar{k} \right) + \hat{\tautens_R} \cdot
  \nabla\hat{\mathbf{u}} - \hat{G_k} - \hat{\rho}
  \epsilon^{n+\theta_{k\epsilon}} ,
\end{equation}
\noindent where:
\begin{equation}\label{eq:g_k_td}
  \hat{G_k} = \frac{\hat{\mu_T}}{\hat{\rho}\sigma_T} \, \mathbf{g}\cdot\nabla\hat{\rho},
\end{equation}


\begin{equation}\label{eq:eps_td}
  \hat{\rho}\frac{\partial \epsilon}{\partial t} +
  \hat{\rho}\hat{\mathbf{u}}\cdot \nabla
  \bar{\epsilon} =
  \nabla\cdot\left(\left(\mu + \frac{\hat{\mu_T}}{\sigma_\epsilon}\right)
    \nabla{\bar{\epsilon}} \right) + C_{\epsilon1} \left(
    \frac{\hat{\epsilon}}{\hat{k}} \right) \left(
    \hat{\tautens_R} \cdot \nabla\hat{\mathbf{u}} -
    \hat{G_\epsilon} \right) - C_{\epsilon2} \hat{\rho}
  \frac{\hat{\epsilon}^2}{\hat{k}} ,
\end{equation}
\noindent where:
\begin{equation}\label{eq:g_eps_td}
  \hat{G_\epsilon} = C_{\epsilon3} \, \frac{\hat{\mu_T}}{\hat{\rho}\sigma_T} \, \mathbf{g}\cdot\nabla \hat{\rho},
\end{equation}

\begin{equation}\label{eq:Cepsilon3_td}
  C_{\epsilon3} = \mathrm{tanh}\left ( \frac{\hat{u_z}}{\hat{u_{xy}}}  \right ),
\end{equation}

The low-Reynolds number model follows the same convention with equations \ref{eq:R_y} and
\ref{eq:Re_T} becoming:
\begin{equation}\label{eq:R_y_td}
  \hat{R_y} = \frac{\hat{\rho} \hat{k}^{1/2} y}{\mu},
\end{equation}
\begin{equation}\label{eq:Re_T_td}
  \hat{Re_T} = \frac{\hat{\rho} \hat{k}^2}{\hat{\epsilon} \mu},
\end{equation}

Additionally, it is possible to implement each of the source terms that are specific to the $k-\epsilon$ model as absorbtion terms. When this is done, the chosen source term, denoted as $\hat{S}$, becomes:
\begin{equation}
  \hat{S}_{absorbtion} = \frac{\hat{S}}{\hat{\gamma}}\bar{\gamma}
\end{equation}
\noindent where $\gamma$ is $k$ or $\epsilon$ depending upon the equation the source term is from. This can help stability in some cases and is the recommended approach for the destruction terms.

\subsection{Large-Eddy Simulation (LES)}\label{sec:LES}
\index{large eddy simulation}
\index{turbulence model}
In Large Eddy Simulations the large scales in the flow are captured while the effect of the small scales
is modelled. Formally, a filtering operator is defined and a decomposition, similar to the Reynolds
decomposition, is introduced:
\begin{equation}
\overline{u}_i = \int_{-\infty}^\infty G_m\left( \overrightarrow{r} \right) u_i \left( \overrightarrow{x} - \overrightarrow{r} \right) \text{d} \overrightarrow{r}
\label{eqn:les_filtering}
\end{equation}
and
\begin{equation}
u_i = \overline{u}_i + u_i^\prime
\label{eqn:les_decomposition}
\end{equation}
where $u_i^\prime$ denotes the subgrid-scale fluctuation. Applying
the filtering operator to the continuity and momentum equations for constant-property, incompressible
flow and introducing the decomposition \eqref{eqn:les_decomposition} to the non-linear terms gives,
\begin{equation}
\frac{\partial \overline{u}_i}{\partial x_i} = 0
\label{eqn:filtered_continuity}
\end{equation}
\begin{equation}
\frac{\partial \overline{u}_i}{\partial t} + \frac{\partial \overline{u}_i \ \overline{u}_j}{\partial x_j} =
 -\frac 1 \rho \frac{\partial \overline{p}}{\partial x_i}
 + 2 \nu \frac{\partial \overline{S}_{ij}}{\partial x_j} - \frac{\partial \tau_{ij}}{\partial x_j}
\label{eqn:filtered_momentum}
\end{equation}
where $\overline{S}_{ij}$ denotes the strain-rate tensor of the filtered velocity field
and $\tau_{ij}$ is usually termed the residual stress tensor,
\begin{equation}
\overline{S}_{ij} = \frac 1 2 \left ( \frac{\partial \overline{u}_i}{\partial x_j} + \frac{\partial \overline{u}_j}{\partial x_i} \right )
\label{eqn:defin_strain_rate_tensor}
\end{equation}
%Currently avoiding the introduction of triple decompositions if the residual tensor.
\begin{equation}
\tau_{ij} = \overline{u_i u_j} - \overline{u}_i \ \overline{u}_j
\label{eqn:defin_sgs_stress_tensor}
\end{equation}
\par
The above, formal, definition of LES involves an explicit filtering operation. Various options are available
for the filtering kernel $G_m\left( \overrightarrow{r} \right)$ in the convolution \eqref{eqn:les_filtering},
see \cite{pope2000} and \cite{sagaut1998}. However, in most implementations the filtering kernel is tied
to the mesh and the numerical approximation.%Must check and find out what is the filtering kernel used in Fluidity, this issue is currenlty evaded here.
\par
The residual stress tensor is an unknown and a subgrid-scale model is used to close the Navier-Stokes
equations. All three subgrid-scale models implemented in \fluidity\ are based on the eddy-viscosity concept: The small scales in the flow act as a diffusive agent, so the subgrid stress can be expressed in a way
similar to the viscous stress:
\begin{equation}
\tau_{{ij}_a} = \tau_{ij} - \delta_{ij} \frac 1 3 \tau_{kk} = -2 \nu_\tau \overline{S}_{ij}
\label{eqn:eddy_viscosity_concept} 
\end{equation}
where only the anisotropic part of $\tau_{ij}$ is treated explicitly. This is usual practise, as the isotropic
part can be added to the pressure. In addition, this allows for the diagonal components of $\tau_{ij}$
to be non-zero when $S_{ij}=0$. Equation \eqref{eqn:filtered_momentum} becomes:
\begin{equation}
\frac{\partial \overline{u}_i}{\partial t} + \overline{u}_j \frac{\partial \overline{u}_i}{\partial x_j}
 = -\frac 1 \rho \frac{\partial \overline{p}}{\partial x_i}
 + \frac{\partial}{\partial x_j} \left [ (\nu + \nu_\tau) \left ( \frac{\partial \overline{u}_i}{\partial x_j} + \frac{\partial \overline{u}_j}{\partial x_i} \right ) \right ],
\end{equation}
\par
In comparison with the unfiltered Navier-Stokes equations, equation \eqref{eqn:filtered_momentum}
features an additional term. This is reflected in the options that must be adjusted in order to
perform large eddy simulations using \fluidity. In particular, the user must activate the option
with path \lstinline+...Velocity/prognostic/spatial_discretisation/continuous_galerkin/les_model+
(See chapter \ref{chap:configuration} for how to configure \fluidity\ options and interact with diamond).
Note however, currently \fluidity\ supports ``explicit'' LES only with continuous Galerking discretisations.
Once the aforementioned option is activated, the user can select and configure one of the three models
outlined in the next section. The available models (and option paths) are:
\begin{itemize}
  \item Second order dissipation model (\lstinline+.../les_model/second_order+)
  \item Fourth order dissipation model (\lstinline+.../les_model/fourth_order+)
  \item Dynamic Smagorinsky model (\lstinline+.../les_model/dynamic_les+)
\end{itemize}

\subsubsection{Subgrid-scale modelling}
\label{sec:LES_sgs_modelling}
The subgrid-scale models available in \fluidity\ are based on the Smagorinsky model
\citep{smagorinsky1963general}. The Smagorinsky model itself is a mixing length model, where by
dimensional analysis \citep{deardorff1970, germano1992} the eddy viscosity is expressed as:
\begin{equation}
\nu_\tau = c \ l^{\frac{4}{3}} \ \epsilon^{\frac 1 3}
\label{eqn:mixing_length}
\end{equation}
where $c$ is a dimensionless constant, $l$ is the \emph{Smagorinsky length-scale} (the mixing length)
and $\epsilon$ is the rate of dissipation. The assumption that $\epsilon$ is equal to the production
rate $\mathcal{P}$ is then used, correct only if the cut-off frequency of the filter is placed in the
inertial sub-range of the spectrum:
\begin{equation}
\epsilon = \mathcal{P} = \tau_{{ij}_a} \overline{S}_{ij}
\label{eqn:production_equals_dissipation}
\end{equation}
Equations \eqref{eqn:eddy_viscosity_concept}, \eqref{eqn:mixing_length} and
\eqref{eqn:production_equals_dissipation} give: 
\begin{equation}
\nu_\tau = C_s^2 \ l^2 \ \left| \mathcal{\overline{S}} \right| 
\end{equation}
where $C_s$ is the \emph{Smagorinsky coefficient} and $\left | \mathcal{\overline{S}} \right |$ is the local strain
rate (the second invariant of the strain-rate tensor):
\begin{equation}
\left| \mathcal{\overline{S}} \right| = \left( 2 \overline{S}_{ij} \overline{S}_{ij} \right)^{1/2}
\end{equation}

\par{\textbf{Second-Order Dissipation model}\\}
\label{sec:second_order_diss_LES}

The following model developed by \citet{bentham2003} is similar to the original Smagorinsky model, but allows for an anisotropic eddy viscosity that gives better results for flow simulations on unstructured grids:

\begin{equation}
\frac{\partial \tau_{ij}}{\partial x_j} = \frac{\partial}{\partial x_i} \left [ \nu_{jk}\frac{\partial \overline{u}_j}{\partial x_k} \right ],
\end{equation}
with the anisotropic tensorial eddy viscosity:

\begin{equation}
\nu_{jk} = 4C_s^2 \left | \mathcal{\overline{S}} \right | \mathcal{M}^{-1},
\end{equation}
where $\mathcal{M}$ is the length-scale metric from the adaptivity process (see \citet{pain2001}),
used here to relate eddy viscosity to the local grid size. The factor of 4 arises in both of the above formula because the filter width separating resolved and unresolved scales is assumed to be twice the local element size, which is squared in the viscosity model.

There is also a scalar (rather than tensorial) length scale metric available, defined by
\begin{equation}
 l = 2V^{\frac{1}{n}},
\end{equation}
where $n$ is the dimension of the problem, and $V$ is the volume (if $n=3$) or area (if $n=2$) of the element. Using this yields an isotropic eddy viscosity tensor:
\begin{equation}
 \nu_{jk} = 4C_s^2 \left | \mathcal{\overline{S}} \right |V^{\frac{1}{n}}.
\end{equation}

\par
The options for this model are available under the spud path 
\lstinline+.../les_model/second_order+. The following options are available to the user:
\begin{itemize}
\item \lstinline+.../les_model/second_order/smagorinsky_coefficient+\ : (Compulsory) The user can here
      specify the value of the Smagorinsky coefficient $C_s$. A value of $0.1$ is recommended for most flows.
      However, many researchers have carried out calibration studies for particular flows and mesh
      resolutions, for example see \cite{deardorff1971}, \cite{nicoud1999}, \cite{germano1991} and
      \cite{canuto_cheng1997}.
\item \lstinline+.../les_model/second_order/length_scale_type+\ : (Compulsory) The user can select between
      the scalar or tensor length scales, described above.
\item \lstinline+.../les_model/second_order/tensor_field::EddyViscosity+\ : (Optional) When this
      option is active the eddy viscocity is calculated as a diagnostic field. Further sub-options
      allow the user to store this field for later visualisation and post-processing.
\end{itemize}

\par{\textbf{Fourth-Order Dissipation model}\\}
\label{sec:fourth_order_diss_LES}

The fourth-order method is designed as an improvement to the second-order eddy viscosity method, which can be too dissipative \citep{adam}. The fourth-order term is taken as the difference of two second-order eddy viscosity discretisations, where one is larger than the other. Usually a smaller time-step and finer grid are necessary to make fourth-order worthwhile.
\par
Currently the only option for this model allows the specification of the Smagorinsky coefficient. This option 
is available at the spud path \lstinline+.../les_model/fourth_order/smagorinsky_coefficient+\ , see the
corresponding option of the second-order dissipation model above for more information.

\par{\textbf{Dynamic Smagorinsky model}\\}
\label{sec:dynamic_smag_LES}

The most important disadvantage of Smagorinsky-type models such as discussed in sections 
\ref{sec:second_order_diss_LES} and \ref{sec:fourth_order_diss_LES} is the behaviour of the
eddy viscosity near walls and non-turbulent regions. As shown in \citet{pope2000}, in a fully developed
channel flow the sub-grid eddy viscosity $\nu_\tau$ should diminish as $\nu_\tau \propto z^3$, $z$ the
wall-normal coordinate. Conversely, it is shown in \citet{nicoud1999} that the Smagorinsky model gives
$\sqrt{2 \tilde{S}_{ij} \tilde{S}_{ij}} \sim O(1)$ near walls leading to an incorrect subgrid-scale
viscosity. In addition, the Smagorinsky model is absolutely dissipative: energy transfer is only allowed
from resolved scales to subgrid scales. The opposite (commonly termed as backscatter), commonly occurs in
transitional flows, and absolutely dissipative models have been shown in \citet{piomelli1990} to
under-predict the growth rate of perturbations in the flow, leading to delayed transition to turbulence.%check
\par
Dynamical models were designed to overcome the aforementioned disadvantages by using arguments
with better physical grounding as a starting point towards the evaluation of the subgrid viscosity.
In particular, at the core of dynamical Smagorinsky type models lies the idea of scale similarity
\citep{bardina1980}, which states that in the inertial subrange, the statistical properties of the
fluctuations at a given wave-number are similar to the statistical properties of fluctuations of
near-by wave-numbers.
\par
Formally, a second filtering of equations \eqref{eqn:filtered_continuity} and
\eqref{eqn:filtered_momentum} is introduced:
\begin{equation}
\tilde{\overline{u}}_i = \int_{-\infty}^\infty G_t\left( \overrightarrow{r} \right) \overline{u}_i \left( \overrightarrow{x} - \overrightarrow{r} \right) \text{d} \overrightarrow{r}
\label{eqn:les_test_filtering}
\end{equation}
\begin{equation}
\frac{\partial \tilde{\overline{u}}_i}{\partial x_i} = 0
\label{eqn:twice_filtered_continuity}
\end{equation}
\begin{equation}
\frac{\partial \tilde{\overline{u}}_i}{\partial t} + \frac{\partial \tilde{\overline{u}}_i \ \tilde{\overline{u}}_j}{\partial x_j} =
 -\frac 1 \rho \frac{\partial \tilde{\overline{p}}}{\partial x_i}
 + 2 \nu \frac{\partial \tilde{\overline{S}}_{ij}}{\partial x_j} - \frac{\partial T_{ij}}{\partial x_j}
\label{eqn:twice_filtered_momentum}
\end{equation}

The first filter is the mesh filter $G_m(x)$ (see equation \eqref{eqn:les_filtering}) and has a
characteristic filter width $\overline \bigtriangleup$ in physical space. $\overline{u}_i$ is simply
the velocity represented on the computational mesh. The second filter, $G_t(x)$, is called the test filter
and has a characteristic filter width $\widetilde \bigtriangleup$.
%It is the convolution of a wider filter $\widetilde G(x)$ with $\overline G(x)$: $\widetilde{\overline G}^2 = \widetilde G^2 + \overline G^2$.
The filter widths are related by,
\begin{equation}
\widetilde \bigtriangleup/\overline \bigtriangleup = \alpha
\label{eqn:les_filter_ratio}
\end{equation}
where $\alpha=2$ for best results \citep{germano1991}. The residual stress resulting from mesh-filtering,
$\tau_{ij}$ is given in equation \eqref{eqn:defin_sgs_stress_tensor}. The residual stress tensor resulting
from test-filering, $T_{ij}$ is:
\begin{equation}\label{testtau}
T_{ij} = \widetilde{\overline{u_i u_j}} - \widetilde{\overline u}_i \widetilde{\overline u}_j,
\end{equation}
The Germano identity postulates the relation bewteen the test-filtered $\tau_{ij}$ and $T_{ij}$:

\begin{equation}\label{leonard}
\mathcal L_{ij} = T_{ij} - \widetilde{\tau}_{ij} = \widetilde{\overline u_i \overline u_j} - \widetilde{\overline u}_i \widetilde{\overline u}_j.
\end{equation}

The Smagorinsky model is used to obtain an expression for the stress tensors on the right-hand-side of
the germano identity above. The Smagorinsky model is here written as:

\begin{equation}\label{eqn:smag_model}
\tau_{ij} = -2 \nu_\tau \overline S_{ij} \text{\ \ where,\ \ } \nu_\tau = C \overline \bigtriangleup^2 \left | \overline S \right |
\end{equation}

Where $C=C_s^2$ is used, for convenience. The Germano identity is used in conjuction with equation
\eqref{eqn:smag_model} and after contraction with the strain rate tensor \eqref{eqn:defin_strain_rate_tensor} 
an expression for the model coefficient $C$ is derived (see \cite{germano1991,germano1992} for details):

\begin{equation}\label{germanocoeff}
C(\mathbf x, t) = - \frac{1}{2} \frac{\mathcal L_{ij} \overline S_{ij}}
{\widetilde{\overline \bigtriangleup}^2 | \widetilde{\overline S} | \widetilde{\overline S}_{ij} \overline S -
\overline \bigtriangleup^2 \widetilde{ | \overline S | \overline S_{ij}} \overline S}.
\end{equation}
Then the eddy viscosity is given by

\begin{equation}\label{germanovisc}
\nu_\tau(\mathbf x, t) = - \frac{1}{2} \frac{\mathcal L_{ij} \overline S_{ij}}
{(1+\alpha^2) | \widetilde{\overline S} | \widetilde{\overline S}_{ij} \overline S -
\widetilde{ | \overline S | \overline S_{ij}} \overline S}
\left | \overline S \right |.
\end{equation}

Unlike the Smagorinsky-based models of \citet{bentham2003}, the eddy viscosity is isotropic in equation \eqref{germanovisc}, since the anisotropic filter widths have cancelled top and bottom. The anisotropy of the mesh is accounted for in the definition of the test filter $G_t(x)$.

A major drawback of the Germano variant of the dynamic Smagorinsky model is that the denominator in
equations \eqref{germanocoeff} and \eqref{germanovisc} can become very small. In addition, equation
\eqref{germanovisc} can give large fluctuations in the subgrid scale viscosity, which can also lead to
instability. Planar averaging of the terms in the numerator and denominator of equations
\eqref{germanocoeff} and \eqref{germanovisc} is used in \cite{germano1991} to stabilise turbulent
channel flow simulations. In \fluidity\ the Germano variant is implemented as well as the Lilly
variant. In the latter, in order to remove the instability caused by the denominator of
\eqref{germanocoeff} becoming small, a modified expression for $C$ is proposed in \cite{lilly1991}:

\begin{equation}\label{lillycoeff}
C(\mathbf x, t) = - \frac{1}{2} \frac{\mathcal L_{ij} M_{ij}}{M^2_{ij}},
\end{equation}
where
\begin{equation}\label{eqn:lilly_m_tensor}
M_{ij} = {\widetilde{\overline \bigtriangleup} | \widetilde{\overline S} | \widetilde{\overline S}_{ij} -
\widetilde{\overline \bigtriangleup} \widetilde{ | \overline S | \overline S_{ij}}}.
\end{equation}
This represents a least-squares error minimisation of the equation $\mathcal L_{ij} - \frac 1 3 \delta_{ij} \mathcal L_{kk} = 2CM_{ij}$. The modification is reported to remove the need for planar averaging.

The numerators of \eqref{germanocoeff} and \eqref{lillycoeff} can become negative, resulting in negative eddy viscosity. In this case, energy is transferred from the subgrid scales to the resolved scales. \citet{germano1991} report that the stresses were closer to DNS as a result. This is an option in Fluidity.%, with a lower limit set on the eddy viscosity to prevent $(\nu + \nu_T) < 0$. <---It appears, from the options that no such bound is currently implemented.

Equations \eqref{germanocoeff} and \eqref{lillycoeff} require the calculation of the test-filtered velocity field $\tilde{\overline{u}}_i$. The inverse Helmholtz filter is implemented in \fluidity\ for use with the dynamic model:
% It was introduced by \citet{germano1986a} in the context of LES, and is defined by <---had to drop the citation, No metion of Helmholtz filter found therein, could locate another suitable citation.

\begin{equation}\label{invhelm}
\overline{u}_i = \left( 1 - \alpha^2 \nabla^2 \right) \tilde{\overline{u}}_i
\end{equation}
where $\alpha$ is the filter width ratio defined above. It is related to the local mesh size by $\alpha^2 = \overline \bigtriangleup^2/24$ \citep{pope2000}. In \fluidity\ equation \eqref{invhelm} is constructed in weak finite element form with a strong Dirichlet boundary condition on the filtered field: $\tilde{\overline{u}}_i = \overline{u}_i$. Similar problems must be solved for calculation of
$\widetilde{ | \overline S | \overline S_{ij}}$ in equations \eqref{germanocoeff}, \eqref{germanovisc} and \eqref{eqn:lilly_m_tensor} 

\par
The options available to the user are located under path \lstinline+.../les_model/dynamic_les+.
\begin{itemize}
  \item \lstinline+.../les_model/dynamic_les/alpha+\ : (Compulsory) The test-to-mesh filter ratio,
        see equation \eqref{eqn:les_filter_ratio}.
  \item \lstinline+.../les_model/dynamic_les/solver+\ : (Compulsory) Sub-options allow the user to
        select the matrix solver used in solving equation \eqref{invhelm}, during the calculation
        of test-filtered fields. See section \ref{ND_Linear_solvers} for available
        linear solvers and their options.
  \item \lstinline+.../les_model/dynamic_les/enable_lilly+\ : (Optional) When active, the Lilly
        variant of the model is used.
  \item \lstinline+.../les_model/dynamic_les/enable_backscatter+\ : (Optional) When inactive the
        subgrid eddy viscocity is constrained to be positive, $\nu_\tau \geq 0$.
  \item \lstinline+.../les_model/dynamic_les/vector_field::FilteredVelocity+\ : (Optional)
        Sub-options allow the user to store realisations of the twice-filtered velocity field,
        $\tilde{\overline{u}}_i\ i=1,2,3$, for post-processing.
  \item \lstinline+.../les_model/dynamic_les/tensor_field::FilterWidth+\ : (Optional)
        Sub-options allow the user to store realisations of the mesh filter width field,
        $\overline \bigtriangleup$, for post-processing.
  \item \lstinline+.../les_model/dynamic_les/tensor_field::StrainRate+\ : (Optional)
        Sub-options allow the user to store realisations of the strain rate of the mesh-filtered
        velocity, for post-processing.
  \item \lstinline+.../les_model/dynamic_les/tensor_field::FilteredStrainRate+\ : (Optional)
        Sub-options allow the user to store realisations of the strain rate of the twice-filtered
        velocity, for post-processing.
  \item \lstinline+.../les_model/dynamic_les/tensor_field::EddyViscosity+\ : (Optional)
        Sub-options allow the user to store realisations of the eddy viscosity, $\nu_\tau$,
        for post-processing.
\end{itemize}

\section{Ice shelf parameterisation}
Exchange of heat and salt between the ice and ocean drives the circulation.
The temperature $T_b$ and salinity $S_b$ at the ice-ocean interface are determined by the balance of heat and salt fluxes between the ice and ocean \cite[e.g.][]{mcpheebook,jenkins95}:
\begin{equation}
mL + mc_I(T_b-T_I)=c_0 \gamma_T |u| (T-T_b); \quad mS_b=\gamma_S |u| (S-S_b),
\label{three}
\end{equation}
where $c_o=3974~J~kg^{-1}~^\circ C^{-1}$ and $c_I=2009~J~kg^{-1}~^\circ C^{-1}$ are the specific heat capacity of seawater and ice, respectively. The variable $m$ and $L=3.35 \times 10^{5}~J~kg^{-1}$ represent the melt rate and the latent heat of ice fusion.
We assume the temperature of ice to be $T_I=-25~^\circ C $. The velocity, temperature, and salinity of the ocean are $u$, $T$, and $S$, respectively.These two flux balances are linked with constraining the interface to be at the local freezing temperature:
\begin{equation}
T_b = aS_b+b + cP,
\label{last}
\end{equation}
where $a=-0.0573 ^\circ C$, $b=0.0832 ^\circ C$ and $c=-7.53e^{-8}~^\circ C~Pa^{-1} $.
The three unknowns $T_b$, $S_b$, and $m$ are solved by the three equations (\ref{three})-(\ref{last}). 

\subsection{Boundary condition at ice surface}
At the ice surface, heat and salt fluxes to the ocean based on the melt rate are
\begin{equation}
F_H = c_0(\gamma_T |u|+ m)(T-T_b); \quad F_S = (\gamma_S|u|+ m)(S-S_b).
\end{equation}

These boundary conditions are applied to the surface field specified in the flml file.
