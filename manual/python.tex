\chapter{The \fluidity\ Python state interface}\label{chap:python}
\index{field!Python function}
\index{Python!state interface}

\fluidity\ incorporates the Python interpreted programming language as a
mechanism for users to customise the model without editing the main Fortran
source of the model. There are, in fact, two distinct Python interfaces
presented by \fluidity. The first allows users to specify prescribed fields
and the initial conditions of prognostic fields by providing a Python
function of space and time. This interface is documented in section
\ref{Sect:setting_with_python}. The present chapter documents the much more
comprehensive \emph{Python state}\ interface which gives the user access to
the complete current system state. This may be used to specify diagnostic
fields as a function of the values of other fields. In particular, this is
used by embedded models such as the biology model to specify the coupling
between different model variables.

\section{System requirements}

The \fluidity\ Python interface requires Python to be installed as well as
NumPy, the fundamental Python numerical package. To check that \fluidity\
has been build with Python, run:
\begin{lstlisting}[language=Bash]
  fluidity -h
\end{lstlisting}
and check for the lines:
\begin{lstlisting}[language=Bash]
Python support			yes
Numpy support			yes
\end{lstlisting}
Python will be installed on any modern Unix machine but NumPy may need to be
specially installed. Ubuntu and Debian users can do so with the
\lstinline[language=Bash]+python-numpy+ package.

\fluidity\ also requires access to its own Python modules which are stored in
the python directory in the \fluidity\ source tree. To ensure that these are
visible to \fluidity\ at runtime, users should add this directory to the
\lstinline[language=Bash]+PYTHONPATH+ environment variable. For example:\
\begin{lstlisting}[language=Bash]
export PYTHONPATH=<my_fluidity>/python/:$PYTHONPATH
\end{lstlisting}%$
where \lstinline[language=Bash]+<my_fluidity>+ is the location of the
\fluidity\ source tree.

\section{Data types}

The data classes of most importance to users are
\lstinline[language=Python]+State+ and
\lstinline[language=Python]+Field+. Between them, these present all of the
field data in \fluidity\ in a readily accessible way.

\subsection{\lstinline[language=Python]+Field+ objects}

The \lstinline[language=Python]+Field+ class defines data types for
\fluidity\ fields. The fields are implemented as wrappers around the
internal data structures in \fluidity\ so the field values are always
current and changes to field values are seen by the whole model. Field
objects are actually of an appropriate subclass:
\lstinline[language=Python]+ScalarField+,
\lstinline[language=Python]+VectorField+ or
\lstinline[language=Python]+TensorField+, however, these classes differ only
in the shape of arguments to their data routines.

\lstinline+Field+ objects support the following methods and attributes:

\begin{description}
\item[node\_count] The number of nodes in the field.
\item[element\_count] The number of elements in the field.
\item[dimension] The data dimension (not for
  \lstinline[language=Python]+ScalarField+ objects).
\item[node\_val(node)] Return the value of the field at node(s). If
  \lstinline[language=Python]+node+ is a scalar then the result is the value
  of the field at that one node. If \lstinline[language=Python]+node+ is a
  sequence then the result is the value of the field at each of those
  nodes. The shape of the result is given for each case below.
\item[set(node, val)] Set the value(s) of the field at the node(s) specified. If
  \lstinline[language=Python]+node+ is a scalar then the value
  of the field at that one node is set. If \lstinline[language=Python]+node+ is a
  sequence then the value of the field at each of those nodes is set. The
  shape of \lstinline[language=Python]+val+ must be as given below.
\item[addto(node, val)] Add value(s) to the field at the node(s) specified.
  If \lstinline[language=Python]+node+ is a scalar then the value of the
  field at that one node is modified. If \lstinline[language=Python]+node+
  is a sequence then the value of the field at each of those nodes is
  modified. The shape of \lstinline[language=Python]+val+ must be as given
  below.
\item[ele\_loc(ele\_number)] Return the number of nodes in element
  \lstinline[language=Python]+ele_number+.
\item[ele\_nodes(ele\_number)] Return the indices of the nodes in element
  \lstinline[language=Python]+ele_number+. 
\item[ele\_val(ele\_number)] Return the value of the field at the nodes in element
  \lstinline[language=Python]+ele_number+. This is equivalent to calling
  \lstinline[language=Python]+field.node_val(field.ele_nodes(ele_number))+. 
\item[ele\_region\_id(ele\_number)] Return the region\_id of element
  \lstinline[language=Python]+ele_number+. This can be used to specify
  diagnostics which only apply over some portion of the domain. 
\end{description}

\begin{table}
\centering

\begin{tabular}{lccc}
  \textbf{\lstinline[language=Python]+node+ shape}
  &\textbf{\lstinline[language=Python]+ScalarField+ values} & 
  \textbf{\lstinline[language=Python]+VectorField+ values} & 
  \textbf{\lstinline[language=Python]+TensorField+ values}\\
  \textbf{scalar} &
scalar& \lstinline[language=Python]+(dim)+ &\lstinline[language=Python]+(dim, dim)+ \\
  \textbf{sequence} &
\lstinline[language=Python]+(len(node))+&
\lstinline[language=Python]+(len(node), dim)+&
\lstinline[language=Python]+(len(node), dim, dim)+\\
\end{tabular}

\caption{The shapes of the return value of \lstinline[language=Python]+node_val+ and the
\lstinline[language=Python]+val+ argument to
\lstinline[language=Python]+set+ and
\lstinline[language=Python]+addto+. \lstinline[language=Python]+dim+ is the
data dimension of the field.
}
\end{table}

\subsection{\lstinline[language=Python]+State+ objects}

A \lstinline[language=Python]+State+ is a container for all of the fields in
a single material phase. The fields in the material phase are accessed by
the names given in the \fluidity\ options file.
\lstinline[language=Python]+State+ objects contain the following data. In
each case it is assumed that \lstinline[language=Python]+s+ is an object of
class \lstinline[language=Python]+State+.

\begin{description}
\item[scalar\_fields] This is a dictionary of all the scalar fields in the
  material phase. For example, if the state contains a field named
  ``Temperature'' then it can be accessed with\\
  \lstinline+s.scalar_fields['Temperature']+.
\item[vector\_fields] This is a dictionary of all the vector fields in the
  state. For example, the coordinate field can be accessed using
  \lstinline[language=Python]+s.vector_fields['Coordinate']+.
\item[tensor\_fields] This is a dictionary of all the tensor fields in the
  state. For example, if there is a field called ``Viscosity'', it can be
  accessed as \lstinline[language=Python]+s.tensor_fields['Viscosity']+.
\end{description}

A useful debugging facility is that a \lstinline[language=Python]+State+ can
be printed from within python (\lstinline[language=Python]+print `s`+) which
results in a list of the fields and meshes in that state.

\section{Predefined data}

 There will always be a variable named
\lstinline[language=Python]+state+ of type
\lstinline[language=Python]+states+ which will contain all of the fields in
the material phase of diagnostic field currently being calculated.

There will also always be a dictionary called \lstinline[language=Python]+states+
which will contain at least the current material phase. If the interface is
being used to specify a diagnostic field and the\\
\option{\ldots/diagnostic/material\_phase\_support}\ attribute is set to
``multiple'', then all of the material phases will be present in the
states dictionary. For example, if there are two material phases, ``Air''
and ``Water'', then the air velocity will be present as the field\\
\lstinline[language=Python]+states['Air'].vector_fields['Velocity']+. 

In addition to these states, the variable \lstinline[language=Python]+field+
is preset to the current diagnostic to be set (this does not apply in
situations such as the biology model in which multiple fields are to be set
in a single Python calculation). 

The variables \lstinline[language=Python]+time+ and
\lstinline[language=Python]+dt+ are pre-set to the current simulation time
and the timestep respectively.

\section{Importing modules and accessing external data}

The Python interpreter run by \fluidity\ is exactly the same as that used by
the Python command line. Therefore essentially anything which is legal in
Python is legal in the \fluidity\ Python interface. In particular, standard
and user-defined modules can be imported including modules for reading
external data sources. The current directory is automatically added to the
Python search path when \fluidity\ is run so user defined modules placed in
the directory from which \fluidity\ is launched will be found.

\section{The persistent dictionary}

All of the data in the \fluidity\ interpreter is wiped after every field
calculation. Usually this is desirable as it prevents the code for different
diagnostics interfering. If it is necessary to store data in the Python
interpreter after flow passes back to the main model, these data items can
be stored in the dictionary named \lstinline[language=Python]+persistent+
under any key the user chooses.

It is \emph{not}\ safe to store fields extracted from the states in the
persistent dictionary. 

\section{Debugging with an interactive Python session}

It is possible to launch an interactive Python session at runtime from
within \fluidity. This depends on iPython being installed on the system and
is achieved by placing the following code in the flml file at the point
within Python at which you wish to stop:
\begin{lstlisting}[language=Python]
import fluidity_tools
fluidity_tools.shell()()
\end{lstlisting}
Note the double brackets \lstinline[language=Python]+()()+ after
\lstinline[language=Python]+fluidity_tools.shell+! From within this Python
session, you can examine or set any variables which are currently
visible. As well as using this to debug straightforward syntax errors, you
can trap much more complex errors by placing the commands inside an
\lstinline[language=Python]+if+ statement or a
\lstinline[language=Python]+try+\ldots\lstinline[language=Python]+except+ clause.
As soon as you leave the Python shell, \fluidity\ execution will continue.

It is also possible to launch an interactive Python session from within the
simple Python interface for prescribed fields, however resuming execution
afterwards currently causes a crash.

\section{Limitations}

The Python state interface is essentially driven at fields which are
pointwise functions of other fields. This is only straightforward in the
situation where all the fields concerned are on the same mesh. Where this is
not the case, there are two different work-arounds which can be used.

If the field to be calculated is discontinuous or periodic and it is desired
to use continuous and/or non-periodic data in the calculation (for example
because the field is a function of position), this can be achieved by
looping over the elements and setting the value at each node of each
element rather than directly looping over all the nodes in the field. This
is possible because the element numbering and the local numbering of nodes within
elements is common between meshes of different continuity. 

In other cases, such as the case in which a diagnostic field is a function
of a field of different polynomial order, it may be necessary to introduce
an additional diagnostic field which is the Galerkin projection of the
second field onto the same mesh as the diagnostic field. 
