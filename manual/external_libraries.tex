\chapter{External libraries}\label{chap:external}

\section{Introduction}

This appendix gives an overview of the external libraries that are required to
build and run \fluidity.

\fluidity's development strategy has taken a conscious decision to employ
external libraries wherever it is possible and beneficial to do so. This
industry-standard approach both short-cuts the development process by making
use of the work and expertise of external projects, and in most cases provides
a better solution than could be implemented by the \fluidity development team.

\section{List of external libraries and software}
\label{sec:required_ḻibraries_list}

\fluidity requires the following libraries and supporting software to build and
run:

\begin{itemize}
\item Fortran 90 and C++ compiler (tested gfortran 4.5-4.7, Intel 11.1 for versions above 11.1.073)
\item BLAS (tested netlib, ATLAS, MKL)
\item LAPACK (tested netlib, ATLAS, MKL)
\item XML2 (tested version 2.6)
\item MPI2 implementation (tested OpenMPI version 1.4.3)
\item PETSc (tested versions 3.0 and 3.1)
\item ParMetis (tested version 3.2)
\item ARPACK (tested version 96)
\item Python (tested version 2.7) 
\item NumPy (tested version 1.6.1)
\item SciPy (tested version 0.9.0)
\item Trang (tested version 20030619, any recent version should work)
\item VTK (tested version 5.6)
\item Zoltan (tested version 3.501)
\end{itemize}

\fluidity recommends also making available the following:

\begin{itemize}
\item CGNS (tested version 2.5)
\item NetCDF (tested version 4.0)
\item UDUnits (tested version 2.1.23)
\item Bazaar (tested version 2.4.0)
\end{itemize}

\section{Installing required libraries on Debian or Ubuntu}
\label{sec:required_ḻibraries_debian}\index{libraries!installing on debian and ubuntu}

By far the easiest way to obtain all the supporting libraries and other
software required to build and run \fluidity is to make use of the \fluidity
packages available from the Launchpad fluidity-core PPA. These are available
for as many Ubuntu versions as \fluidity currently supports; at the time of
writing, 10.10 (Maverick) and newer, though support for 10.10 will end as of
April 2012 when central Ubuntu support ends.

\textbf{BE AWARE:} AMCG packages are provided for use at your own risk and
without warranty. You should ensure that any packages installed from external
repositories are not going to adversely affect your system before installing
them!

To access the repository containing the \fluidity support packages, you will
need to run:

\begin{lstlisting}[language=bash]
sudo apt-add-repository -y ppa:fluidity-core/ppa
\end{lstlisting}

You will then need to update your system and install the fluidity-dev package,
which depends on all the other software required for building \fluidity:

\begin{lstlisting}[language=bash]
sudo apt-get update
sudo apt-get install fluidity-dev
\end{lstlisting}

To benefit from the environment modules supplied from AMCG you may want to add
the following lines to your \lstinline[language=bash]+/etc/bash.bashrc+ file:

\begin{lstlisting}[language=bash]
if [ -f /usr/share/Modules/init/bash ]; then
    . /usr/share/Modules/init/bash
fi
\end{lstlisting}

New bash shells should automatically inherit the modules environment now, and
you should be able to type:

\begin{lstlisting}[language=bash]
module load petsc-gcc4
\end{lstlisting}

to enable use of PETSc, required for building \fluidity.
 
\section{Manual install of external libraries and software}
\label{sec:required_ḻibraries_manual_install}\index{libraries!installing externals from source}

Competent systems administrators should find it relatively straightforward to
install the supporting software and external libraries required by \fluidity on
most modern UNIX systems and compute clusters. The following instructions are
intended to help with this process, offering hints and tips to speed up the
deployment process.

In most cases, modern Linux systems will supply some if not most of the
required packages without needing to resort to compiling them from source.

\subsection{Supported compilers}
\label{sec:required_libraries_supported_compilers}

Before starting to build the supporting libraries for \fluidity it is strongly
recommended that you ensure that your builds will use a compiler that is also
able to build \fluidity. If you do not, you may encounter problems when you try
to interface \fluidity with the libraries. At present, \fluidity is tested with
gcc/g++/gfortran versions 4.5-4.7 and the Intel compiler with version 11.1.073
and later versions of 11.1. Versions of 11.1 earlier than 11.1.073 contain bugs
that prevent building \fluidity, as do all other major and minor versions of
the Intel compiler.

\subsection{Build environment}
\label{sec:required_libraries_build_environment}

The following compile instructions assume that you have set up a basic bash
environment containing a few key environment variables. Set WORKING to be the
root of your working area which the subsequent variables will refer to:

\begin{lstlisting}[language=bash]
export WORKING="/path/to/my/data"

export PATH="$WORKING/fluidity/bin:$PATH"
export LD_LIBRARY_PATH="$LD_LIBRARY_PATH:$WORKING/fluidity/lib"
export CFLAGS="-L$WORKING/fluidity/lib"
export FFLAGS="-L$WORKING/fluidity/lib"
export CPPFLAGS="-I$WORKING/fluidity/include"
export LDFLAGS="-L$WORKING/fluidity/lib"
\end{lstlisting}

csh users will want to alter all export commands to the corresponding setenv
syntax throughout this appendix.

Throughout this section where standard configure, make, and install is referred
to it is assumed to mean running the following commands:

\begin{lstlisting}[language=bash]
./configure --prefix=$WORKING/fluidity
make
make install
\end{lstlisting}

Where the source directory for a package is referred to it is assumed to mean
the root directory created when the package is uncompressed.

\subsection{Compilers}
\label{sec:required_libraries_compilers}\index{compilers}

The \fluidity build process requires working Fortran 90 and C++ compilers.

\fluidity has been tested with gfortran $\geq$ 4.4 and Intel 11.1 for versions
$\geq$ 11.1.073. It is not supported for gfortran $\leq$ 4.3 or Intel $\leq$
11.1.073, and if using gfortran some features are not available except for
gfortran $\geq$ 4.5. Current testing is with gfortran 4.6, so whilst it is
expected that 4.4 and 4.5 will function correctly, this is not guaranteed.

Unsupported compilers generally have incorrect Fortran 90 implementations for
which bug reports have been submitted and implemented in later versions where
applicable. Bug reports have been submitted for Portland group compilers but
not yet implemented. \fluidity is not yet tested against Intel 12.x compilers.

\fluidity has been tested with both GNU and Intel C++ compilers at corresponding
versions to the tested and known-good Fortran 90 compilers.

If you do not already have compilers suitable for building \fluidity, GCC is
freely available and is possible to build from source with sufficient disk
space and time.

\subsubsection{GMP, MPFR, and MPC}
\label{sec:required_libraries_compilers_gmp_mpfr_mpc}

GMP (tested for \fluidity with gmp-5.0.5) and MPFR (tested for \fluidity with
mpfr-3.1.0) are needed for the GCC 4.6 build if you do not already have them.
Download GMP from \url{http://gmplib.org/} and build it in the source
directory, appending \lstinline[language=bash]+--enable-cxx+ to the standard
configure, then running the standard make and install.

Once GMP has been installed, download MPFR from
\url{http://www.mpfr.org/mpfr-current/} and build it in the source directory,
appending \lstinline[language=bash]+--with-gmp=\$WORKING/fluidity+ to the
standard configure, then running the standard make and install.

Once MPFR has been installed, download MPC from 
\url{http://www.multiprecision.org/index.php?prog=mpc&page=download} and build
it in the source directory, appending
\lstinline[language=bash]+--with-gmp=$WORKING/fluidity --with-mpfr=$WORKING/fluidity+ 
to the standard configure, then running the standard make and install.

If you have no compilers at all, you may need to download GMP, MPFR, and MPC as
pre-built binaries. If you have any compilers, even if not ones which support
building \fluidity, you should be able to build GMP, MPFR, and MPC and then go
on to build GCC. Finding a binary compiler distribution from the start will
make matters a great deal simpler for you if that option is available.


\subsubsection{GCC}
\label{sec:required_libraries_compilers_gcc}

GCC (tested for \fluidity with gcc-4.6.x Ubuntu-built compilers, assumed
working with most 4.4.x and 4.5.x, not known yet to work with 4.7.x) can be
downloaded from the UK mirror at \url{http://gcc-uk.internet.bs/} Before the
build, make sure that the GMP, MPFR, and MPC libraries are on
LD{\textunderscore}LIBRARY{\textunderscore}PATH or the stage 1 configure will
fail even if \lstinline[language=bash]+--with-[gmp|mpfr|mpc]+ is supplied.

Also note that the build needs to be in a target build directory, NOT in the
source directory, or again the build will fail with definition conflicts
against the system includes.

Finally, the java build appears to be buggy in (at least) 4.4 and is enabled by
default, but \fluidity only needs c, c++, and fortran, so just specify those.

Make a target build directory which is OUTSIDE the source tree, then build
with the following configure:

\begin{lstlisting}[language=bash]
/path/to/gcc/source/configure --prefix=$WORKING/fluidity 
--with-gmp=$WORKING/fluidity --with-mpfr=$WORKING/fluidity 
--with-mpc=$WORKING/fluidity --enable-languages=c,c++,fortran
\end{lstlisting}

followed by the standard make and install.

The GCC4 build has not been tested by the \fluidity builders in an environment
with no compilers at all and thus the ability or otherwise of the build to
bootstrap itself is not known.

\subsubsection{OpenMPI}
\label{sec:required_libraries_compilers_openmpi}\index{OpenMPI}

Finally, you'll need an MPI implementation to wrap your compiler for the
\fluidity build, which lets you spawn parallel runs from the compiled \fluidity
binary. Any full MPI implementation should be sufficient, though \fluidity is
generally tested using OpenMPI. Please note that Clustervision-supplied
clusters generally ship with broken MPI C++ support and will need attention
before \fluidity can be compiled. \fluidity is no longer generally supported as
non-MPI code as it is assumed that serial runs will be precursors to large
parallel runs and will be built with MPI enabled for later use.

OpenMPI (tested for \fluidity with openmpi-1.4.3) can be downloaded from
\url{http://www.open-mpi.org/software/ompi/v1.4/}. It is built in the source
directory with the standard configure, make, and install.

\subsection{Python}
\label{sec:required_libraries_python}\index{Python!installing}

Python is widely used within \fluidity for user-defined functions and for
diagnostic tools and problem setup, and currently tested up to Python version
2.7. Earlier Python version may be suitable for use but may lack later
functionality. Python extensions required are: setuptools for \fluidity builds,
Python-4suite and Python-XML for options file parsing, and NumPy for custom
function use within \fluidity.

If you do not have a working version of Python it can be built from source. 

\subsubsection{Readline}
\label{sec:required_libraries_python_readline}

Readline (tested version 6.2) is not strictly needed for Python to build but is
very handy if you want to make use of things like Python command history.
Download readline from \url{http://ftp.gnu.org/pub/gnu/readline/} and then add
-fPIC to CFLAGS and FFLAGS for the duration of this build with:

\begin{lstlisting}[language=bash]
export CFLAGS="$CFLAGS -fPIC"
export FFLAGS="$FFLAGS -fPIC"
\end{lstlisting}

These can be returned to their default values after the readline build.
Building with -fPIC shouldn't be necessary but seems to be required by the
later Python build.

Build readline in the source directory, appending
\lstinline[language=bash]+--disable-shared+ to the standard configure and then
running the standard make and install.

\subsubsection{Python}
\label{sec:required_libraries_python_python}

Python (tested version 2.7) can be downloaded from
\url{http://www.python.org/download/} and built in the source directory with
the standard configure, make, and install, adding
\lstinline[language=bash]+--enable-shared+ to the configure flags.

\subsubsection{Setuptools}
\label{sec:required_libraries_python_setuptools}

Setuptools (tested version 0.6c11) can be downloaded from
\url{http://pypi.python.org/pypi/setuptools} and installed from the appropriate
egg file with:

\begin{lstlisting}[language=bash]
sh ./setuptools-0.6c11-py2.6.egg --prefix=$WORKING/fluidity
\end{lstlisting}

Use the appropriate filename for the Python egg you have downloaded.

\subsubsection{NumPy}
\label{sec:required_libraries_python_numpy}\index{numpy}

NumPy (tested version 1.6.0) can be installed through
easy{\textunderscore}install which was provided by setuptools, using:

\begin{lstlisting}[language=bash]
easy_install numpy
\end{lstlisting}

\subsection{Numerical Libraries}
\label{sec:required_libraries_numerical}

BLAS and LAPACK are required for efficient linear algebra methods within
\fluidity, and are tested with the netlib, ATLAS, and MKL implementations, though
any standard BLAS or LAPACK implementation should be sufficient for \fluidity.
PETSc is required to provide matrix solvers, and ParMetis is required for mesh
partitioning and sparse matrix operations.

\subsubsection{BLAS}
\label{sec:required_libraries_numerical_blas}\index{BLAS}

BLAS can be dowloaded from \url{http://www.netlib.org/blas/} (for netlib BLAS),
\url{http://sourceforge.net/projects/math-atlas/files/} (for ATLAS), or
combined with commercially available compilers such as MKL from Intel.

BLAS is built in the source directory after editing the following entries in
the make.inc file:

\begin{lstlisting}[language=bash]
FORTRAN  = gfortran
LOADER   = gfortran
OPTS     = -O3 -fPIC
\end{lstlisting}

Then run:

\begin{lstlisting}[language=bash]
make
cp blas_LINUX.a $WORKING/fluidity/lib/libblas.a
\end{lstlisting}

\subsubsection{LAPACK}
\label{sec:required_libraries_numerical_lapack}\index{LAPACK}

LAPACK can be downloaded from \url{http://www.netlib.org/lapack/} (for netlib
LAPACK), \url{http://sourceforge.net/projects/math-atlas/files/} (for ATLAS),
or combined with commercially available compilers such as MKL from Intel. 

LAPACK is built in the source directory. First, make a copy of
make.inc.example:

\begin{lstlisting}[language=bash]
cp make.inc.example make.inc
\end{lstlisting}

Edit it to set:

\begin{lstlisting}[language=bash]
FORTRAN  = gfortran -fimplicit-none -g -fPIC
BLASLIB  = /path/to/your/libraries/libblas.a
\end{lstlisting}

Then:

\begin{lstlisting}[language=bash]
make
cp lapack_LINUX.a $WORKING/fluidity/lib/liblapack.a
\end{lstlisting}

\subsubsection{ParMetis}
\label{sec:required_libraries_numerical_parmetis}\index{ParMetis}

ParMetis (tested for \fluidity with version 3.2.0) is required for mesh
partitioning and sparse matrix operations, and can be downloaded from
\url{http://glaros.dtc.umn.edu/gkhome/fsroot/sw/parmetis/OLD}

ParMetis is built in the source directory with:

\begin{lstlisting}[language=bash]
make
cp lib*.a $WORKING/fluidity/lib
cp parmetis.h $WORKING/fluidity/include
\end{lstlisting}

Note that \fluidity is NOT currently tested with ParMETIS 4.0.0, and further
note that ParMETIS is subject to licensing conditions for commercial users.
Commercial users should contact the University of Minnesota's Office for
Technology Commercialization directly. 

\fluidity is in the process of transitioning away from using ParMETIS to avoid
this restrictive commercial licensing, and by release 4.2 should have an
alternative partitioner which is free to use for commercial partners.

\subsubsection{PETSc}
\label{sec:required_libraries_numerical_petsc}\index{PETSc}

PETSc (currently tested for \fluidity with version 3.1p8) is required for
efficient solver methods within \fluidity. Some work has been carried out
to ensure that Fluidity builds correctly with PETSc 3.2 but this has not
been extensively tested, so іs used at your own risk. Work is ongoing to
support the latest development branch of PETSc but this is not currently
supported as an option.

PETSc can be downloaded from
\url{http://www.mcs.anl.gov/petsc/petsc-2/download/index.html} and built in the
source directory. First, set PETSC{\textunderscore}DIR in the source directory:

\begin{lstlisting}[language=bash]
export PETSC_DIR=$PWD
\end{lstlisting}

Then configure with the following all on one line:

\begin{lstlisting}[language=bash]
./configure --prefix=$WORKING/fluidity --with-mpi-shared=1 --with-shared=1
   --with-debugging=0 --with-parmetis=1 --download-parmetis=1 --with-hypre=1
   --download-hypre=1 --with-prometheus=1 --download-prometheus=1
   --with-fortran-interfaces=1
\end{lstlisting}

When configure completes, it should supply you with a 'make all' command line,
including various configuration variables. Copy and paste this into your
terminal and run it. Once it completes, it will supply you with a further 'make
install' command line containing other variables; do the same with this, copy
and pasting it into your terminal and running it. 

Finally, reset the environment variables back to normal:

\begin{lstlisting}[language=bash]
export PETSC_DIR=$WORKING/fluidity
\end{lstlisting}

NOTE: If you see problems with shared libraries not building correctly, make
sure you have built BLAS and LAPACK with \lstinline[language=bash]+-fPIC+.

\subsection{VTK and supporting software}
\label{sec:required_libraries_vtk}\index{vtk!installing}

VTK is required for \fluidity data output, and currently tested to version 5.6.
If you do not already have them installed, you will need Cmake (tested for
version 2.8.5, download from
\url{http://www.cmake.org/cmake/resources/software.html}), as well as Tcl and
Tk (tested for version 8.5.10, download from
\url{http://www.tcl.tk/software/tcltk/download.html}. All three packages are
built with the standard configure, make, and install, Cmake first in its source
directory, then Tcl followed by Tk in each package's respective unix/
subdirectory of their main source directory.

VTK (tested version 5.8 from the Ubuntu build) can be download from
\url{http://www.vtk.org/VTK/resources/software.html}

The following build is for a non-graphical install of \fluidity - ie, one for a
cluster, not one for a workstation expected to run diamond. In the situation
that diamond is required, VTK{\textunderscore}USE{\textunderscore}RENDERING
must be enabled and dependencies on GTK+ satisfied which are provided on the
vast majority of modern Linux systems. A description of how to satisfy these
dependencies from scratch is beyond the scope of this appendix. When building
VTK, it is recommended that shared libraries are enabled, and that VTKpython is
enabled. The \fluidity configure script should be tolerant of local variations
in terms of VTK libraries either being supported internally by VTK or supported
through system libraries.

At runtime, the environment variables VTK{\textunderscore}INCLUDE and
VTK{\textunderscore}LIBS will need to be set to point at your VTK install, and
the library directory added to LD{\textunderscore}LIBRARY{\textunderscore}PATH.

For the build described here, these would be set to:

\begin{lstlisting}[language=bash]
export VTK_INCLUDE="$WORKING/fluidity/include/vtk-5.6"
export VTK_LIBS="$WORKING/fluidity/lib/vtk-5.6"
export LD_LIBRARY_PATH="$LD_LIBRARY_PATH:$WORKING/fluidity/lib/vtk-5.6"
\end{lstlisting}

If you are building from source, VTK should be built in a separate build
directory which is not inside the source hierarchy. In the source directory
run:

\begin{lstlisting}[language=bash]
mkdir ../VTK-build
cd ../VTK-build
ccmake ../VTK/
\end{lstlisting}

Then type 'c' and edit the resulting rules screen to:

\begin{lstlisting}[language=bash]
BUILD_EXAMPLES                   OFF
BUILD_SHARED_LIBS                ON                                     
BUILD_TESTING                    ON
CMAKE_BACKWARDS_COMPATIBILITY    2.4
CMAKE_BUILD_TYPE
CMAKE_INSTALL_PREFIX             /path/to/WORKING/fluidity          
VTK_DATA_ROOT                    VTK_DATA_ROOT-NOTFOUND
VTK_USE_GEOVIS                   OFF                                    
VTK_USE_INFOVIS                  OFF                                    
VTK_USE_N_WAY_ARRAYS             OFF
VTK_USE_PARALLEL                 OFF
VTK_USE_RENDERING                OFF                                    
VTK_USE_VIEWS                    OFF                                    
VTK_WRAP_JAVA                    OFF
VTK_WRAP_PYTHON                  ON                                     
VTK_WRAP_TCL                     OFF
\end{lstlisting}

Then:

\begin{itemize}
  \item Type 'c' a first time to configure
  \item Type 'c' a second time to configure
  \item Type 'g' to generate and quit
\end{itemize}

Finally, run the standard make and install.

\subsection{Supporting Libraries}
\label{sec:required_libraries_supporting}

The above libraries should be sufficient for the most basic \fluidity builds,
but, depending on local requirements, other external libraries may be required.
Brief details and suggestions for avoiding common problems are given here, but
package instructions should be referred to for full build details.

\subsubsection{XML2}
\label{sec:required_libraries_supporting_xml2}\index{XML2}

XML2 is required for parsing \fluidity's flml parameter file format, and is
tested with version 2.7.8. It can be downloaded from \url{ftp://xmlsoft.org/libxml2/}

\subsubsection{ARPACK}
\label{sec:required_libraries_supporting_arpack}\index{ARPACK}

ARPACK is required for solving large eigenvalue problems, and is tested with
version 96 with paths configured for the local site. It can be downloaded from
\url{http://www.caam.rice.edu/software/ARPACK/}

\subsubsection{NetCDF}
\label{sec:required_libraries_supporting_netcdf}\index{NetCDF}

NetCDF is required for reading datafiles in NetCDF format, and is tested with
version 4.1.3. It is recommended to be configured with f77, f90, c, cxx, and
utilities enabled, and with \lstinline[language=bash]+--enable-shared+ added to
the standard configure. 

NetCDF can be downloaded from \url{http://www.unidata.ucar.edu/downloads/netcdf/}

Experimental NetCDF 5 is not yet tested for use with Fluidity.

\subsubsection{UDUnits}
\label{sec:required_libraries_supporting_udunits}\index{UDUnits}

UDUnits is required for physical unit conversions, and is tested with version
2.1.23. Note there is a common issue with hand-building this package where
\lstinline[language=bash]+CPPFLAGS+ needs to be correctly set with a -D
option for the relevant Fortran environment. This commonly leads to an error
during the build when not set. See for example
\url{http://www.unidata.ucar.edu/downloads/udunits/}.

Setting

\begin{lstlisting}[language=bash]
export CPPFLAGS=-Df2cFortran
\end{lstlisting}

should be sufficient for a GCC-based build on Linux.

UDUnits can be downloaded from \url{http://www.unidata.ucar.edu/downloads/udunits/index.jsp}

Legacy support should be present for UDUnits 1.x but this is no longer tested 
hence is used at your own risk.

\subsubsection{Trang}
\label{sec:required_libraries_supporting_trang}\index{Trang}

Trang is required for parsing \fluidity's flml schema, and is tested with
20030619 but any recent version should be sufficient. Trang can be downloaded
from \url{http://www.thaiopensource.com/relaxng/trang.html}

\subsubsection{Bazaar (bzr)}
\label{sec:required_libraries_supporting_bzr}\index{bzr!installing}

Bazaar is recommended as a general tool for accessing the \fluidity code
repository, and tested to version 2.5.0. Bazaar can be
downloaded from \url{https://launchpad.net/bzr/+download}

\subsubsection{CGNS}
\label{sec:required_libraries_supporting_cgns}\index{CGNS}

CGNS is required when any CGNS-format datafiles are being read by \fluidity, and
is currently tested to version 2.5.4. It can be downloaded from 
\url{http://cgns.sourceforge.net/download.html}.
