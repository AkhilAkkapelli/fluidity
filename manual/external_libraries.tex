\chapter{External libraries}\label{chap:external}

\section{Introduction}

This appendix gives an overview of the external libraries that are required to
build and run \fluidity.

\fluidity's development strategy has taken a conscious decision to employ
external libraries wherever it is possible and beneficial to do so. This
industry-standard approach both short-cuts the development process by making
use of the work and expertise of external projects, and in most cases provides
a better solution than could be implemented by the \fluidity development team.

\section{List of external libraries and software}
\label{sec:required_libraries_list}

\fluidity requires the following libraries and supporting software to build and
run:

\begin{itemize}
\item Fortran 90 and C++ compiler (gcc tested version 4.8.2, expected working for gcc 4.5.x and higher; other compilers not currently supported but work is ongoing with Intel to support their 2014 releases)
\item BLAS (tested netlib, ATLAS, MKL)
\item LAPACK (tested netlib, ATLAS, MKL)
\item MPI2 implementation (tested OpenMPI version 1.6.5)
\item PETSc (tested version 3.4.3)
\item ParMetis (tested version 3.2, 4.x not supported)
\item Python (build tested version 2.7.5) 
\item NumPy (build tested version 1.8.1)
\item VTK (tested version 5.10.1, 6.x not supported)
\item Zoltan (build tested version 3.8)
\item GMSH (tested version 2.8.4)
\end{itemize}

\fluidity recommends also making available the following:

\begin{itemize}
\item Trang (tested version 20030619, any recent version should work)
\item NetCDF (tested version 4.1.3)
\item UDUnits (tested version 2.2.0)
\item Bazaar (tested version 2.6.0)
\item SciPy (tested version 0.13.3)
\item ARPACK (tested version 3.1.5)
\item XML2 (tested version 2.9.1)
\end{itemize}

\section{Installing required libraries on major Linux distributions}
\label{sec:required_libraries_linux}\index{libraries!installing on linux}

\fluidity comes pre-packages for many major Linux distributions, both as a set
of libraries for building the code and as a pre-built \fluidity binary package
for immediate use.

\textbf{BE AWARE:} All packages from \fluidity repositories are provided for
use at your own risk and without warranty. You should ensure that any packages
installed from external repositories are not going to adversely affect your
system before installing them!

\subsection{Installing on Ubuntu}
\label{sec:required_libraries_Ubuntu}\index{libraries!installing on Ubuntu}

\fluidity aims to fully support the latest Ubuntu LTS distribution at any given
time, to offer support for previous LTS distributions for as long as practical
within the official five-year Ubuntu support period, and to offer packages for
building and installing \fluidity on all other main Ubuntu releases still
listed as current on
\href{https://wiki.ubuntu.com/Releases}{https://wiki.ubuntu.com/Releases}
. A full testing program will be carried out on the latest LTS
distribution as a matter of \fluidity general policy, whilst testing on other
Ubuntu versions will be carried out as available resources permit.

At the time of writing (April 2014), all current releases are supported bar
Ubuntu 10.10 LTS (lucid).

Going forward, timely support for six-monthly Ubuntu releases is intended,
although the faster-moving nature of this target may mean that from time to
time there is a delay in releasing a full \fluidity package set. Users move
from the LTS to six-month releases at the price of being on a lower-priorty
platform as far as central \fluidity support and testing is concerned.

To access the repository containing the \fluidity packages, you will
need to run:

\begin{lstlisting}[language=bash]
sudo apt-add-repository -y ppa:fluidity-core/ppa
\end{lstlisting}

To develop or locally build \fluidity, you will then need to update your system
and install the fluidity-dev package, which depends on all the other software
required for building \fluidity:


\begin{lstlisting}[language=bash]
sudo apt-get update
sudo apt-get install fluidity-dev
\end{lstlisting}

If you wish to install the \fluidity binary packages you can also run:

\begin{lstlisting}[language=bash]
sudo apt-get install fluidity
\end{lstlisting}

Note that the above is tested on a clean Ubuntu LTS install with no other PPAs
or other external repositories enabled; if you have other software sources
which may conflict with the core repositories and the \fluidity PPA then you
will need to contact your local technical support team in the case of package
conflicts or version mismatches.

\subsection{Installing on Red Hat Enterprise and derivatives}
\label{sec:required_libraries_redhat}\index{libraries!installing on redhat}

\fluidity aims to fully support the latest Red Hat Enterprise Linux release at any given
time, and to offer support for previous distributions for as long as practical
within the official Red Hat support period. Development and testing is carried
out using the CentOS community rebuild of Red Hat Enterprise. Packages are also
sourced from the Extra Packages for Enterprise Linux (EPEL) repository,
maintained by Fedora.

It is expected that the community rebuilds of Red Hat Enterprise Linux,
including CentOS and Scientific Linux, should be able to use the \fluidity Red
Hat package repositories.

At the time of writing, Red Hat Enterprise Linux (Desktop) version 6.5 is the
supported and tested release, though it is expected that packages provided from
the \fluidity repository will work on most 6.x systems. Red Hat Enterprise 5.x
systems are not supported, as the provided compiler and python versions are too
old for building and testing \fluidity. When version 7 is released, support from
the \fluidity repositories is expected to be promptly provided as Fedora 19, on
which Red Hat Enterprise 7 is based, is already fully supported.

The following methods assume that the EPEL repository is enabled on your
system. If this is not the case, follow instructions at
\url{https://fedoraproject.org/wiki/EPEL} to enable use of EPEL by installing
an 'epel-release' package. Note that this assumes no other third-party
repositories are enabled; if, for example, the rpmforge repository is enabled,
you may well encounter package versioning issues against EPEL and the Fluidity
repository.

To access the repository containing the \fluidity packages, you will
need to run the following, typed all on one line:

\begin{lstlisting}[language=bash]
sudo yum-config-manager --add-repo
                   http://amcg.ese.ic.ac.uk/yum/rhel/6/fluidity.repo
\end{lstlisting}

To develop or locally build \fluidity you will then need to install the
fluidity-dev package, which depends on all the other software required for
building \fluidity:

\begin{lstlisting}[language=bash]
sudo yum install fluidity-dev
\end{lstlisting}

If you wish to install the \fluidity binary packages you can also run:

\begin{lstlisting}[language=bash]
sudo yum install fluidity
\end{lstlisting}

Note that the above is tested on a clean CentOS 6.5 install with no other
external repositories enabled; if you have other software sources which may
conflict with the \fluidity, EPEL, or core repositories then you will need to
contact your local technical support team in the case of package conflicts or
version mismatches. Problems have been reported using the rpmforge repositories
alongside the \fluidity and EPEL repositories, for example.

\subsection{Installing on Fedora}
\label{sec:required_libraries_fedora}\index{libraries!installing on fedora}

\fluidity aims to fully support all Fedora releases that are supported
centrally by the Fedora team, although the faster-moving target presented by
Fedora means that from time to time latest release may not be immediately or
fully supported.

At the time of writing, Fedora 19 is fully supported and tested for \fluidity,
but Fedora 20 is not yet supported as \fluidity does not yet support building
with VTK6.

To access the repository containing the \fluidity packages for Fedora 19, you
will need to run the following, typed all on one line:

\begin{lstlisting}[language=bash]
sudo yum-config-manager --add-repo http://amcg.ese.ic.ac.uk/yum/fedora/19/fluidity.repo
\end{lstlisting}

To develop or locally build \fluidity you will then need to install the
fluidity-dev package, which depends on all the other software required for
building \fluidity:

\begin{lstlisting}[language=bash]
sudo yum install fluidity-dev
\end{lstlisting}

If you wish to install the \fluidity binary packages you can also run:

\begin{lstlisting}[language=bash]
sudo yum install fluidity
\end{lstlisting}

Note that the above is tested on a clean Fedora 19 install with no other
external repositories enabled; if you have other software sources which may
conflict with the \fluidity or Fedora repositories then you will need to
contact your local technical support team in the case of package conflicts or
version mismatches.

\subsection{Installing on OpenSuSE}
\label{sec:required_libraries_opensuse}\index{libraries!installing on opensuse}

\fluidity aims to fully support all OpenSuSE releases that are supported
centrally by the OpenSuSE team, for as long as they remain in support and are
realistic platforms on which \fluidity can be built.

At the time of writing, OpenSuSE 12.3 is fully supported and tested for \fluidity,
but OpenSuSE 13.1 is not yet supported as \fluidity does not yet support building
with VTK6.

To access the repository containing the \fluidity packages for OpenSuSE 12.3, you
will need to run:

\begin{lstlisting}[language=bash]
sudo zypper ar -f http://amcg.ese.ic.ac.uk/yast/opensuse/12.3/
\end{lstlisting}

To develop or locally build \fluidity you will then need to install the
fluidity-dev package, which depends on all the other software required for
building \fluidity:

\begin{lstlisting}[language=bash]
sudo zypper install fluidity-dev
\end{lstlisting}

If you wish to install the \fluidity binary packages you can also run:

\begin{lstlisting}[language=bash]
sudo zypper install fluidity
\end{lstlisting}

Note that the above is tested on a clean OpenSuSE 12.3 install with no other
external repositories enabled; if you have other software sources which may
conflict with the \fluidity or core OpenSuSE repositories then you will need to
contact your local technical support team in the case of package conflicts or
version mismatches.

\section{Manual install of external libraries and software}
\label{sec:required_libraries_manual_install}\index{libraries!installing externals from source}

Competent systems administrators should find it relatively straightforward to
install the supporting software and external libraries required by \fluidity on
most modern UNIX systems and compute clusters. The following instructions are
intended to help with this process, offering hints and tips to speed up the
deployment process.

In most cases, modern Linux systems will supply some if not most of the
required packages without needing to resort to compiling them from source.

These instructions are developed and tested with a base install of
ubuntu-server on top of which are installed gcc, gcc-multilib, g++, m4, gawk,
and make. For other systems, it is assumed that you have the GCC prerequisites
as described in \url{http://gcc.gnu.org/install/prerequisites.html}, as well as
a working C++ compiler and an m4 install. GCC and its supporting software
should be relatively tolerant of older versions, as long as they work and are
POSIX-compliant.

As a side-note, the gcc-multilib requirement for building gcc appears to be
the result of gcc failing to check for the existence of 32-bit headers at
configure time, and thus requiring the 32-bit compatibility libraries to be
installed on 64-bit systems. There may be a workaround for this of which the
\fluidity developers are not aware; in the interim, the gcc build appears to
need some degree of 32-bit support on 64-bit systems.

\subsection{Supported compilers}
\label{sec:required_libraries_supported_compilers}

Before starting to build the supporting libraries for \fluidity it is strongly
recommended that you ensure that your builds will use a compiler that is also
able to build \fluidity. If you do not, you may encounter problems when you try
to interface \fluidity with the libraries. At present, \fluidity is tested with
gcc/g++/gfortran versions 4.8 (and expected to behave reasonably with versions
4.6 and later) and the Intel compiler with version 11.1.073 and later versions
of 11.1. Versions of 11.1 earlier than 11.1.073 contain bugs that prevent
building \fluidity, as do all other major and minor versions of the Intel
compiler. Bug reports have been filed with Intel to remedy this but remain
outstanding as of this document date (March 2014).

\subsection{Build environment}
\label{sec:required_libraries_build_environment}

The following compile instructions assume that you have set up a basic bash
environment containing a few key environment variables. Set WORKING to be the
root of your working area which the subsequent variables will refer to:

\begin{lstlisting}[language=bash]
export WORKING="/path/to/my/installation"

export PATH="$WORKING/fluidity/bin:$PATH"
export LD_LIBRARY_PATH="$LD_LIBRARY_PATH:$WORKING/fluidity/lib:$WORKING/fluidity/lib64"
export CFLAGS="-L$WORKING/fluidity/lib -L$WORKING/fluidity/lib64"
export FFLAGS="-L$WORKING/fluidity/lib -L$WORKING/fluidity/lib64"
export CPPFLAGS="-I$WORKING/fluidity/include"
export LDFLAGS="-L$WORKING/fluidity/lib -L$WORKING/fluidity/lib64"
\end{lstlisting}

csh users will want to alter all export commands to the corresponding setenv
syntax throughout this appendix.

Throughout this section where standard configure, make, and install is referred
to it is assumed to mean running the following commands:

\begin{lstlisting}[language=bash]
./configure --prefix=$WORKING/fluidity
make
make install
\end{lstlisting}

Where the source directory for a package is referred to it is assumed to mean
the root directory created when the package is uncompressed.

\subsection{Compilers}
\label{sec:required_libraries_compilers}\index{compilers}

The \fluidity build process requires working Fortran 90 and C++ compilers.

\fluidity has been tested with gfortran $\geq$ 4.4 and Intel 11.1 for versions
$\geq$ 11.1.073. It is not supported for gfortran $\leq$ 4.3 or Intel $\leq$
11.1.073, and if using gfortran some features are not available except for
gfortran $\geq$ 4.5. Current testing is with gfortran 4.8, so whilst it is
expected that 4.4 and 4.5 will function correctly, this is not guaranteed.

Unsupported compilers generally have incorrect Fortran 90 implementations for
which bug reports have been submitted and implemented in later versions where
applicable. Bug reports have been submitted for Portland group compilers but
not yet implemented. \fluidity is not yet tested against Intel 12.x compilers.

\fluidity has been tested with both GNU and Intel C++ compilers at corresponding
versions to the tested and known-good Fortran 90 compilers.

If you do not already have compilers suitable for building \fluidity, GCC is
freely available and is possible to build from source with sufficient disk
space and time.

All the instructions below assume that you're building on a 64-bit Linux
system; if this is not the case, you will need to specify a different
architecture string for your --build parameter for gcc and the gcc
prerequisites. It is also assumed that you have some form of compilers for C
and C++ available to run the GCC and supporting software builds, as well as a
working M4 install and a working implementation of 'make'; full bootstrapping
instructions are outside the scope of this manual.

For each install step that follows, two software versions are listed, one being
the build-tested version as a 'known highest version building' and the other
being the 'fully tested' version as currently run through automated testing on
a daily basis. Instructions generally correspond to the former version as our
experience is that people building their own versions of packages frequently
want to use latest versions. It is expected that these latest versions will
perform well with \fluidity but some wrinkles may still remain to be ironed
out.

\subsubsection{GMP, MPFR, and MPC}
\label{sec:required_libraries_compilers_gmp_mpfr_mpc}

GMP (tested version 5.1.3), MPFR (tested version 3.1.2), and MPC (tested
version 1.0.1) are needed for the GCC 4.8 build if you do not already have
them. Download GMP from \url{http://gmplib.org/} and build it in the source
directory, appending
\lstinline[language=bash]+--build=x86_64-linux-gnu --enable-cxx+
to the standard configure, then running the standard make and
install.

Once GMP has been installed, download MPFR from
\url{http://www.mpfr.org/mpfr-current/} and build it in the source directory,
appending
\lstinline[language=bash]+--build=x86_64-linux-gnu --with-gmp=\$WORKING/fluidity+
to the standard configure, then running the
standard make and install.

Once MPFR has been installed, download MPC from
\url{http://www.multiprecision.org/index.php?prog=mpc&page=download} and build
it in the source directory, appending
\lstinline[language=bash]+--build=x86_64-linux-gnu --with-gmp=$WORKING/fluidity+
\lstinline[language=bash]+--with-mpfr=$WORKING/fluidity+
to the standard configure, then running the
standard make and install.

If you have no compilers at all, you may need to download GMP, MPFR, and MPC as
pre-built binaries. If you have any compilers, even if not ones which support
building \fluidity, you should be able to build GMP, MPFR, and MPC and then go
on to build GCC. Finding a binary compiler distribution from the start will
make matters a great deal simpler for you if that option is available.

\subsubsection{GCC}
\label{sec:required_libraries_compilers_gcc}

GCC (tested for \fluidity with version 4.8.2) can be downloaded from the UK mirror at
\url{http://gcc-uk.internet.bs/} Before the build, make sure that the GMP,
MPFR, and MPC libraries are on LD{\textunderscore}LIBRARY{\textunderscore}PATH
or the stage 1 configure will fail even if
\lstinline[language=bash]+--with-[gmp|mpfr|mpc]+ is supplied.

Also note that the build needs to be in a target build directory, NOT in the
source directory, or again the build will fail with definition conflicts
against the system includes.

First, set up the following environment variables:
\begin{lstlisting}[language=bash]
export LIBRARY_PATH=/usr/lib/x86_64-linux-gnu/
export C_INCLUDE_PATH=/usr/include/x86_64-linux-gnu
export CPLUS_INCLUDE_PATH=/usr/include/x86_64-linux-gnu
\end{lstlisting}

Then make a target build directory which is OUTSIDE the source tree, then build
with the following configure:

\begin{lstlisting}[language=bash]
/path/to/gcc/source/configure --build=x86_64-linux-gnu
--prefix=$WORKING/fluidity --with-gmp=$WORKING/fluidity
--with-mpfr=$WORKING/fluidity --with-mpc=$WORKING/fluidity
--enable-checking=release --enable-languages=c,c++,fortran 
--disable-multilib --program-suffix=-4.8.2
\end{lstlisting}

followed by the standard make and install. Note you may well want to change the
program suffix in the configure string if you are not building gcc 4.8.1.

Having built a new version of gcc, make sure that subsequent build steps
actually use it. Either explicitly specify the new version by setting
environment variables, ie:

\begin{lstlisting}[language=bash]
export CC=gcc-4.8.2
export CXX=g++-4.8.2
export FC=gfortran-4.8.2
export F90=gfortran-4.8.2
export CPP=cpp-4.8.2
\end{lstlisting}

(with suitable version changes if necessary), or if you prefer you could reset
the symlinks for gcc, gfortran, g++, etc. to point to your new version.

\subsubsection{OpenMPI}
\label{sec:required_libraries_compilers_openmpi}\index{OpenMPI}

Finally, you'll need an MPI implementation to wrap your compiler for the
\fluidity build, which lets you spawn parallel runs from the compiled \fluidity
binary. Any full MPI implementation should be sufficient, though \fluidity is
generally tested using OpenMPI. Please note that Clustervision-supplied
clusters generally ship with broken MPI C++ support and will need attention
before \fluidity can be compiled. \fluidity is no longer generally supported as
non-MPI code as it is assumed that serial runs will be precursors to large
parallel runs and will be built with MPI enabled for later use.

OpenMPI (tested version 1.6.5) can be downloaded from
\url{http://www.open-mpi.org/software/}. It is built in the source directory
with the standard configure, make, and install.

\subsection{Numerical Libraries}
\label{sec:required_libraries_numerical}

BLAS and LAPACK are required for efficient linear algebra methods within
\fluidity, and are tested with the netlib, ATLAS, and MKL implementations, though
any standard BLAS or LAPACK implementation should be sufficient for \fluidity.
PETSc is required to provide matrix solvers, and ParMetis is required for mesh
partitioning and sparse matrix operations.

\subsubsection{BLAS}
\label{sec:required_libraries_numerical_blas}\index{BLAS}

BLAS can be dowloaded from \url{http://www.netlib.org/blas/} (for netlib BLAS),
\url{http://sourceforge.net/projects/math-atlas/files/} (for ATLAS), or
combined with commercially available compilers such as MKL from Intel. The
following method assumes netlib BLAS.

BLAS is built in the source directory after editing the following entries in
the make.inc file:

\begin{lstlisting}[language=bash]
FORTRAN  = gfortran-4.8.1
LOADER   = gfortran-4.8.1
OPTS     = -O3 -fPIC
\end{lstlisting}

(Alter your compiler name and version number as required to suit your system)

Then run:

\begin{lstlisting}[language=bash]
make
cp blas_LINUX.a $WORKING/fluidity/lib/libblas.a
\end{lstlisting}

\subsubsection{LAPACK}
\label{sec:required_libraries_numerical_lapack}\index{LAPACK}

LAPACK can be downloaded from \url{http://www.netlib.org/lapack/} (for netlib
LAPACK), \url{http://sourceforge.net/projects/math-atlas/files/} (for ATLAS),
or combined with commercially available compilers such as MKL from Intel. The
following method assumes netlib LAPACK (tested with \fluidity for version
3.4.2).

LAPACK is built in the source directory. First, make a copy of
make.inc.example:

\begin{lstlisting}[language=bash]
cp make.inc.example make.inc
\end{lstlisting}

Edit it to set:

\begin{lstlisting}[language=bash]
FORTRAN  = gfortran-4.8.1 -fPIC
LOADER   = gfortran-4.8.1 -fPIC
CC = gcc-4.8.1 -fPIC
BLASLIB  = /path/to/your/libraries/libblas.a
\end{lstlisting}

(Alter your compiler names and version numbers as required to suit your system)

Then:

\begin{lstlisting}[language=bash]
make
cp liblapack.a $WORKING/fluidity/lib/liblapack.a
\end{lstlisting}

\subsubsection{ParMetis}
\label{sec:required_libraries_numerical_parmetis}\index{ParMetis}

ParMetis (tested for \fluidity build compatibility with version 3.2.0, fully
tested with version 3.1.1; \fluidity does NOT work correctly with version 4.x) is required for mesh
partitioning, and can be downloaded from
\url{http://glaros.dtc.umn.edu/gkhome/fsroot/sw/parmetis/OLD}

ParMetis is built in the source directory with:

\begin{lstlisting}[language=bash]
make
cp lib*.a $WORKING/fluidity/lib
cp parmetis.h $WORKING/fluidity/include
\end{lstlisting}

Note that \fluidity is NOT currently tested with ParMETIS 4.0.0 and it is
expected that \fluidity will not work correctly with versions higher than 3.2.

\subsubsection{PETSc}
\label{sec:required_libraries_numerical_petsc}\index{PETSc}

If you do not already have it installed, you will need Cmake (tested for
version 2.8.12, download from
\url{http://www.cmake.org/cmake/resources/software.html}) for the PETSc Metis
build.  Cmake is built in its source directory with the standard configure,
make, and install.

PETSc is required for
efficient solver methods within \fluidity. Currently,
\fluidity supports PETSc versions 3.1 to 3.4, using version 3.4
is recommended. It can be downloaded from
\url{http://www.mcs.anl.gov/petsc/download/} and built in the source directory.
First, set PETSC{\textunderscore}DIR in the source directory:

\begin{lstlisting}[language=bash]
export PETSC_DIR=$PWD
\end{lstlisting}

Then configure with the following all on one line:

\begin{lstlisting}[language=bash]
./configure --prefix=$WORKING/fluidity 
   --download-fblaslapack=1 --download-blacs=1 --download-scalapack=1 
   --download-ptscotch=1 --download-mumps=1 --download-hypre=1
   --download-suitesparse=1 --download-ml=1 --with-fortran-interfaces=1
\end{lstlisting}

This set of configuration options provides you with the additional
multigrid preconditioners hypre and ml, and direct solvers umfpack 
from SuiteSparse and mumps. These are not strictly necessary to run
Fluidity, but have been found useful in different applications. NOTE:
you should NOT build PETSc with parmetis
(so no --download/with-parmetis=1) as the version used by PETSc 
(parmetis 4.x) leads to incorrect behaviour in Fluidity. Therefore, the above
configuration options uses ptscotch as an alternative.

When configure completes, it should supply you with a 'make all' command line,
including various configuration variables. Copy and paste this into your
terminal and run it. Once it completes, it will supply you with a further 'make
install' command line containing other variables; do the same with this, copy
and pasting it into your terminal and running it. 

Finally, reset the environment variables back to normal:

\begin{lstlisting}[language=bash]
export PETSC_DIR=$WORKING/fluidity
\end{lstlisting}

NOTE: If you see problems with shared libraries not building correctly, make
sure you have built BLAS and LAPACK with \lstinline[language=bash]+-fPIC+.

\subsubsection{Zoltan}
\label{sec:required_libraries_numerical_zoltan}\index{Zoltan}

Zoltan (fully tested with version 3.8) is available as a tarball from  
\url{http://www.cs.sandia.gov/~kddevin/Zoltan_Distributions/zoltan_distrib_v3.8.tar.gz}

Uncompress the tarball, then make a build directory and change into it:

\begin{lstlisting}[language=bash]
   mkdir Zoltan-build/
   cd Zoltan-build/
\end{lstlisting}

For configuration you will need to supply your system architecture. Type:

\begin{lstlisting}[language=bash]
   uname -i
\end{lstlisting}

A common return value is 'x86{\textunderscore}64' which will be used as the
example for the following configure. If you are on a 32-bit system the above
will return 'i386'. In that case, replace 'x86{\textunderscore}64' with 'i386' in
the following command so 'x86{\textunderscore}64-linux-gnu' reads
'i386-linux-gnu'.

Configure Zoltan in your build directory with the following typed all on one
line:

\begin{lstlisting}[language=bash]
   ../Zoltan_v3.8/configure x86_64-linux-gnu --prefix=$WORKING/fluidity
      --enable-mpi --with-mpi-compilers --with-parmetis
      --enable-f90interface --disable-examples
      --enable-zoltan-cppdriver --with-parmetis-libdir=$WORKING/fluidity/lib
\end{lstlisting}

(This assumes your build directory and the Zoltan source directory are in the
same place; if not, tinker with the path to 'configure' accordingly)

Then run the standard make and make install.

\subsection{Python}
\label{sec:required_libraries_python}\index{Python!installing}

Python is widely used within \fluidity for user-defined functions and for
diagnostic tools and problem setup, and currently tested up to Python version
2.7. Earlier Python version may be suitable for use but may lack later
functionality. \fluidity is not yet tested with Python3. Python extensions
required are: setuptools for \fluidity builds, Python-4suite and Python-XML for
options file parsing, and NumPy for custom function use within \fluidity.

If you do not have a working version of Python it can be built from source. 

\subsubsection{Readline}
\label{sec:required_libraries_python_readline}\index{Python!readline}

Readline (tested version 6.3) is not strictly needed for Python to build but is
very handy if you want to make use of things like Python command history.
Download readline from \url{http://ftp.gnu.org/pub/gnu/readline/} and then add
-fPIC to CFLAGS and FFLAGS for the duration of this build with:

\begin{lstlisting}[language=bash]
export CFLAGS="$CFLAGS -fPIC"
export FFLAGS="$FFLAGS -fPIC"
\end{lstlisting}

These can be returned to their default values after the readline build.
Building with -fPIC shouldn't be necessary but seems to be required by the
later Python build.

Build readline in the source directory, appending
\lstinline[language=bash]+--disable-shared+ to the standard configure and then
running the standard make and install.

\subsubsection{zLib}
\label{sec:required_libraries_python_zlib}\index{Python!zLib}

zLib (tested version 1.2.8) is not strictly needed for Python to build but is
useful for many automated package scripts. Download zLib from 
\url{http://zlib.net/} and build in the source directory with the standard
configure, make, and make install.

\subsubsection{Python}
\label{sec:required_libraries_python_python}

Python (tested version 2.7.5) can be downloaded from
\url{http://www.python.org/download/} and built in the source directory with
the standard configure, make, and install, adding
\lstinline[language=bash]+--enable-shared+ to the configure flags.

\subsubsection{NumPy}
\label{sec:required_libraries_python_numpy}\index{Python!numpy}

NumPy (tested version 1.8.1) can be downloaded as a compressed tarball from
\url{https://pypi.python.org/pypi/numpy} and after unsetting CFLAGS and
LDFLAGS:

\begin{lstlisting}[language=bash]
unset CFLAGS
unset LDFLAGS
\end{lstlisting}

is installed from the source directory with:

\begin{lstlisting}[language=bash]
python ./setup.py --install
\end{lstlisting}

Ensure that the python you run is the python which you have just installed, as
opposed to an alternative system version.

Once this step is complete, remember to reset CFLAGS and LDFLAGS to the
recommended environment versions; assuming you put the code earlier in this 
appendix into your .bashrc you should be able to simply do this by logging out
and in again, or restarting your terminal.

\subsection{VTK and supporting software}
\label{sec:required_libraries_vtk}\index{vtk!installing}

VTK is required for \fluidity data output, and currently tested to version
5.10.0 via Ubuntu's packaged version, although bugs in the 5.10.0 build mean
that for source builds from the original tarball 5.8.0 is recommended and will
be used as the example in this appendix. \fluidity does not currently support
VTK 6.x.

If you do not already have them installed, you will need Cmake (tested for
version 2.8.11.2, download from
\url{http://www.cmake.org/cmake/resources/software.html}), as well as Tcl and
Tk (tested for version 8.6.0, download from
\url{http://www.tcl.tk/software/tcltk/download.html}. All three packages are
built with the standard configure, make, and install, Cmake first in its source
directory, then Tcl followed by Tk in each package's respective unix/
subdirectory of their main source directory.

It should be noted that the Tk build requires X11 development headers to be
present on the system. The expected solutions if this is not the case are
outlined on the Tk wiki at \url{http://wiki.tcl.tk/1843}.

It's also assumed later in this document that you have ncurses on your system
so cmake also builds the ccmake curses interface. If you do not, you can
download ncurses from \url{http://ftp.gnu.org/pub/gnu/ncurses/} and build in
the source directory with the standard configure, make, and make install.

VTK (build tested version 5.8.0) can be downloaded from
\url{http://vtk.org/files/release/5.8/}

The following build is for a non-graphical install of \fluidity - ie, one for a
cluster, not one for a workstation expected to run diamond. In the situation
that diamond is required, VTK{\textunderscore}USE{\textunderscore}RENDERING
must be enabled and dependencies on GTK+ satisfied which are provided on the
vast majority of modern Linux systems. A description of how to satisfy these
dependencies from scratch is beyond the scope of this appendix. When building
VTK, it is recommended that shared libraries are enabled, and that VTKpython is
enabled. The \fluidity configure script should be tolerant of local variations
in terms of VTK libraries either being supported internally by VTK or supported
through system libraries.

At runtime, the environment variables VTK{\textunderscore}INCLUDE and
VTK{\textunderscore}LIBS will need to be set to point at your VTK install, and
the library directory added to LD{\textunderscore}LIBRARY{\textunderscore}PATH.

For the build described here, these would be set to:

\begin{lstlisting}[language=bash]
export VTK_INCLUDE="$WORKING/fluidity/include/vtk-5.8"
export VTK_LIBS="$WORKING/fluidity/lib/vtk-5.8"
export LD_LIBRARY_PATH="$LD_LIBRARY_PATH:$WORKING/fluidity/lib/vtk-5.8"
\end{lstlisting}

If you are building from source, VTK should be built in a separate build
directory which is not inside the source hierarchy. In the source directory
run:

\begin{lstlisting}[language=bash]
mkdir ../VTK-build
cd ../VTK-build
ccmake ../VTK
\end{lstlisting}

Then type 'c' and edit the resulting rules screen to:

\begin{lstlisting}[language=bash]
BUILD_EXAMPLES                  OFF
BUILD_SHARED_LIBS               ON
BUILD_TESTING                   ON
CMAKE_BACKWARDS_COMPATIBILITY   2.4
CMAKE_BUILD_TYPE
CMAKE_INSTALL_PREFIX            /path/to/WORKING/fluidity
VTK_DATA_ROOT                   VTK_DATA_ROOT-NOTFOUND
VTK_EXTRA_COMPILER_WARNINGS     OFF
VTK_LARGE_DATA_ROOT             VTK_LARGE_DATA_ROOT-NOTFOUND
VTK_USE_CHARTS                  OFF
VTK_USE_CHEMISTRY               OFF
VTK_USE_GEOVIS                  OFF
VTK_USE_INFOVIS                 OFF
VTK_USE_N_WAY_ARRAYS            OFF
VTK_USE_PARALLEL                OFF
VTK_USE_QT                      OFF 
VTK_USE_RENDERING               OFF
VTK_USE_TEXT_ANALYSIS           OFF
VTK_USE_VIEWS                   OFF
VTK_WRAP_JAVA                   OFF
VTK_WRAP_PYTHON                 ON
VTK_WRAP_PYTHON_SIP             OFF
VTK_WRAP_TCL                    OFF 
\end{lstlisting}

Then:

\begin{itemize}
  \item Type 'c' a first time to configure
  \item Type 'c' a second time to configure
  \item Type 'g' to generate and quit
\end{itemize}

Finally, run the standard make and install.

\subsection{GMSH}
\label{sec:required_libraries_gmsh}\index{gmsh!installing}

GMSH is strongly recommended for mesh conversion and creation and is tested
with version 2.8.4. It can be downloaded from the 'source' link at 
\url{http://geuz.org/gmsh/#Download}.

Uncompress the downloaded tarball, then make a build directory and change into
it:

\begin{lstlisting}[language=bash]
   mkdir gmsh-build/
   cd gmsh-build/
\end{lstlisting}

From this build directory (assuming it's in the same location as the
uncompressed source), run:

\begin{lstlisting}[language=bash]
   ccmake -i ../gmsh-2.8.4-source
\end{lstlisting}

Press 'c' to configure, then when that completes leave all options at defaults
and press 'c' a second time to configure again, then when that completes press
'g' to generate a configuration and exit. If at any point you end up in a help
screen, press 'e' to exit it.

Finally, run the standard make and install.

\subsection{Supporting Libraries}
\label{sec:required_libraries_supporting}

The above libraries should be sufficient for the most basic \fluidity builds,
but, depending on local requirements, other external libraries may be required.
Brief details and suggestions for avoiding common problems are given here, but
package instructions should be referred to for full build details.

\subsubsection{XML2}
\label{sec:required_libraries_supporting_xml2}\index{XML2}

XML2 is required for parsing \fluidity's flml parameter file format, and is
tested with version 2.9.0. It can be downloaded from \url{ftp://xmlsoft.org/libxml2/}

\subsubsection{ARPACK}
\label{sec:required_libraries_supporting_arpack}\index{ARPACK}

ARPACK is required for solving large eigenvalue problems, and is tested with
version 96 with paths configured for the local site. It can be downloaded from
\url{http://www.caam.rice.edu/software/ARPACK/}

\subsubsection{NetCDF}
\label{sec:required_libraries_supporting_netcdf}\index{NetCDF}

NetCDF and NetCDF-fortran are required for reading datafiles in NetCDF format,
and are tested with NetCDF version 4.2.1.1 and NetCDF-fortran version 4.2. They
is recommended to be configured with f77, f90, c, cxx, and utilities enabled,
and with \lstinline[language=bash]+--enable-shared+ added to the standard
configure. 

NetCDF can be downloaded from \url{http://www.unidata.ucar.edu/downloads/netcdf/}

Experimental NetCDF 5 is not yet tested for use with Fluidity.

\subsubsection{UDUnits}
\label{sec:required_libraries_supporting_udunits}\index{UDUnits}

UDUnits is required for physical unit conversions, and is tested with version
2.1.24. Note there is a common issue with hand-building this package where
\lstinline[language=bash]+CPPFLAGS+ needs to be correctly set with a -D
option for the relevant Fortran environment. This commonly leads to an error
during the build when not set. See for example
\url{http://www.unidata.ucar.edu/downloads/udunits/}.

Setting

\begin{lstlisting}[language=bash]
export CPPFLAGS=-Df2cFortran
\end{lstlisting}

should be sufficient for a GCC-based build on Linux.

UDUnits can be downloaded from \url{http://www.unidata.ucar.edu/downloads/udunits/index.jsp}

Legacy support should be present for UDUnits 1.x but this is no longer tested 
hence is used at your own risk.

\subsubsection{Trang}
\label{sec:required_libraries_supporting_trang}\index{Trang}

Trang is required for parsing \fluidity's flml schema, and is tested with
20030619 but any recent version should be sufficient. Trang can be downloaded
from \url{http://www.thaiopensource.com/relaxng/trang.html} and requires a
Java Runtime Environment to be present.

\subsubsection{Git}
\label{sec:required_libraries_supporting_git}\index{git!installing}

Git is recommended as a general tool for accessing the \fluidity code
repository, and generally expected to function well for versions later than
1.2. Git can be downloaded from \url{http://git-scm.com/download}
