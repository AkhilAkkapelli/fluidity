\chapter{Visualisation and Diagnostics}
\label{chap:visualisation_and_diagnostics}

%%%%%%%%%%%%%%%%%%%%%%%%%%%%%%%%%%%%%%%%%%%%%%%%%%%%%%%%%%%%%%%%%%%
%--------------------VISUALISATION--------------------------------%
%%%%%%%%%%%%%%%%%%%%%%%%%%%%%%%%%%%%%%%%%%%%%%%%%%%%%%%%%%%%%%%%%%%


\section{Visualisation: Mayavi}
\label{sect:mayavi}

Mayavi is a visualisation program that can be used to view the vtu files output by \fluidity. There are two versions, Mayavi 1.5 and Mayavi 2.   The following instructions are for Mayavi 1.5 because  most people still prefer to use it, but please note that mayavi 1.5 is no longer supported by Linux. The following instructions relate to Mayavi 1.5, but the use of Mayavi 2 is very similar. In the examples mentioned the only difference between the two is in the layout of the buttons in the interface. Further information about both of them is available at \url{http://mayavi.sourceforge.net/}.

To open a set of vtu files from the command line, type the command in Example \ref{ex:mayavi}.

\begin{example}
  \begin{lstlisting}[language=bash]
mayavi -d run_0.vtu
\end{lstlisting}
  \caption{This opens mayavi and opens the set of files specified.}
\label{ex:mayavi}
\end{example}

Once you have opened Mayavi, if you wish to visualise something in two dimensions, you should open a SurfaceMap. Click on the visualise menu, then modules, then SurfaceMap (its called Surface in Mayavi 2). Click on Configure Data at the top left of the window to change the field that is being viewed. This also allows you to change the time step that is being viewed.  The SurfaceMap can also be configured by altering the settings on the SurfaceMap window. It can be changed to show the mesh by selecting 'show wireframe', and it can also show contours. The SurfaceMap window normally appears automatically but you can open it by clicking on SurfaceMap at the bottom of the screen. 

In three dimensions it is helpful to use a ScalarCutPlane or VectorCutPlane instead of a SurfaceMap. These will show a cross-section of the domain. The surface that you view can be altered in the ScalarCutPlane or VectorCutPlane window.  The IsoSurface module is also useful; it will display the surface where a specified field has a constant value. This value is altered in the IsoSurfae window.  
 

\section{Visualisation: Paraview}
\label{sect:paraview}

Paraview (\url{http://www.paraview.org/paraviewindex.html}) is an open-source, multi-platform parallel visualization application, which makes use of scalable architecture to reproduce large scale simulations.

\subsection{Obtaining Paraview}
\subsubsection{Ubuntu}
Paraview is available through the Ubuntu package system and can be installed by typing: 

\begin{example}
  \begin{lstlisting}[language=bash]
wajig install paraview
\end{lstlisting}
\end{example}

Intrepid provides paraview 3.2.2, Jaunty provides paraview 3.2.3, and Karmic
provides paraview 3.4.0. The user should be aware that there are problems
with paraview 3.2. In particular, it restricts axis scaling to 100 times the
original, which may well not be sufficient for some oceanographic
applications.

When using an AMCG workstation, access to in-house packages for Paraview
3.6.1, which as of this documentation is the latest version, will be
available. To get access from other workstations, the repository lines for
the AMCG repository in the sources.list file are:

\begin{example}
  \begin{lstlisting}[language=bash]
deb http://amcg.ese.ic.ac.uk/debian intrepid main contrib non-free
deb-src http://amcg.ese.ic.ac.uk/debian intrepid main contrib non-free
\end{lstlisting}
\end{example}

Be aware they contain a number of other updated packages so if you only want
paraview you may want to comment out or remove those lines once you have
done the paraview install.

\subsubsection{Direct Download}

To either use a newer version of the software or to simply gain access to
paraview without installing the system's packages, a pre-compiled paraview
client can be downloaded from the paraview website at:

\url{http://www.paraview.org/paraview/resources/software.html}

This download page has pre-compiled client software for 32-bit and 64-bit
Windows, Linux, and Mac OS X. Administrative privileges may be needed on the
computer to install the software.

\subsection{Running Paraview}

On a Ubuntu system, paraview can be run by typing 'paraview' into a
terminal. NB, when running it entirely locally it can be heavy on both CPU
and graphics card which in turn means it is heavy on battery life if
intensively used on a laptop.

On a Mac the Paraview application will probably have been put in the
Applications directory during the install procedure, and it can be run from
there.

No information is available about how to run Paraview in Windows. 

\subsubsection{Starting up a serial visualisation session}

If you don't require the parallel capabilities of paraview, just run it from
the command line or from your applications folder and you should have the
same functionality as you would with a parallel instance, just less
capacity. You can skip directly to the data loading section, ignoring the
extra work to set up the parallel back-end.

\subsubsection{Starting up a parallel visualisation session}

You will need to have both the paraview program and the pvserver program
available to run a parallel visualisation, as well as a working mpiexec to
run the pvserver. Check they're all available to you before you start; an
easy way to check they're on your path is to run:

\begin{example}
  \begin{lstlisting}[language=bash]
which pvserver
which paraview
which mpiexec
\end{lstlisting}
\end{example}

which should return something like:

\begin{example}
  \begin{lstlisting}[language=bash]

$ which pvserver
/usr/bin/pvserver
$ which paraview
/usr/bin/paraview
$ which mpiexec
/usr/bin/mpiexec
$

\end{lstlisting}
\end{example}

Now start up your paraview session, and once you have the main window click on the server connection button, which looks like two computers connected by a line with a green dot on it. You will now have a window which allows you to choose a server, but you will not have any servers defined yet. By clicking 'add server', a new configure server window appears which needs to be filled in. Now, click on 'configure', and you will have a second configuration window which needs to be set by entering the command to run pvserver. These steps are all shown on fig. \ref{fig:paraview1} below.

\begin{figure}[h!]
 \centering
 \label{fig:paraview1}
 \subfigure[Screenshot highlighting the location of the server connection
 button on paraview main
 window.]{\fig[width=0.48\textwidth]{visualisation_and_diagnostics_images/Paraviewconnectbutton}}
 \subfigure[Screenshot of the choose server window as viewed upon initial
 opening.]{\fig[width=0.48\textwidth]{visualisation_and_diagnostics_images/Paraviewchooseserver}}
 \subfigure[Screenshot showing how to configure the server on the
 appropriate
 window.]{\fig[width=0.48\textwidth]{visualisation_and_diagnostics_images/Paraviewconfigureserverwindow}}
 \subfigure[Again, another window needs to be filled in as
 shown.]{\fig[width=0.48\textwidth]{visualisation_and_diagnostics_images/Paraviewconfigureserverwindow2}}
\caption{First steps when starting up the paraview parallel session.}
\end{figure}
 

Note that if pvserver is NOT on your path then you will need to give the full path to pvserver rather than just 'pvserver'. Also note that you can change the '2' to be the number of cores you want to run the parallel visualisation across on your machine. Now click on 'save' and you will be back to the 'choose server' window, with a server 'localhost' defined (see fig. \ref{fig:paraview5}). Click on 'connect' and pserver should run and connect. If there are any problems you'll get an output window telling you what went wrong.

\begin{figure}[h!]
\label{fig:paraview5}
  \centering
  \fig[width=.7\textwidth]{visualisation_and_diagnostics_images/Paraviewchooseserver2}
\caption{The localhost server can now be selected for connection.}
\end{figure}

 

\subsubsection{What are the strange windows with random patterns in?}

This is documented for running a parallel visualisation across the cores of your local system, using your graphics hardware for the rendering. It results in one X-window per pserver process being opened, which you can push to the back behind your paraview window. Don't close them, or pserver will fail! 

\subsection{Reading in data}

\subsubsection{Opening files}

Paraview will natively handle the majority of datafiles that you will generate in the process of your work. To load in a new data file, access the 'File' menu and choose 'Open', then select the file you wish to load and click 'OK'. Single files should appear in the file list immediately, but where a directory contains a time-series of files they will often be condensed into a single entry in the paraview file listing with an arrow to the left of the entry (see fig. \ref{paraview6}). Clicking on the arrow expands the time series to allow you to access individual files, whilst clicking on the series' entry (the generic name, with a '..' before the file type) will select the whole time-series for loading. 

\begin{figure}[h!]
\label{paraview6}
  \centering
  \fig[width=.7\textwidth]{visualisation_and_diagnostics_images/Paraview-filedialogue-timeseries}
\end{figure}

\subsubsection{Reading in fields}

So you have opened a file and an entry for it has appeared in the 'Pipeline Browser' window, generally upper-left of your Paraview session. At this point the file is open but not read, an intermediate state which allows you to select how much of the file should be read into memory. This is very important to be careful of where large files are concerned, as reading all fifty fields on a twenty five million node mesh could be prejudicial to the stability of the computer. 

\begin{figure}[h!]
\label{paraview7}
  \centering
  \fig[width=.7\textwidth]{visualisation_and_diagnostics_images/Paraview-subwindows}
\end{figure}

With the newly opened file highlighted in the Pipeline Browser, the Object Inspector window will show a list of available fields to be read. You can either select/deselect these individually, or select/deselect as a whole by using the tickbox above-left of the list. Once you have selected the fields you wish to read in, click on 'Apply' in the Object Inspector. 

\subsection{Basic Visualisation}

Having loaded in some data you may wish to examine it to check it is what you think it is. With the file highlighted in the pipeline browser you can look at aspects of the data by changing the drop-down menu entry selected in the 'Active Variable Controls' and 'Representation' toolbars. If either are not present you can re-enable them in the 'View' then 'Toolbars' menu. One drop-down menu should allow you to alter the colour of the visualised mesh from a default solid colour to either colour representations of properties of the mesh or colour representations of fields loaded in and defined for the mesh.

For the representation menu, 'points', 'wireframe', 'outline', 'surface', and 'surface with edges' should be familiar concepts to Mayavi users. However, 'volume' may not be. 'Volume' displays the mesh as a partially-transparent cloud of points which can be very useful to denote the extent and fields of the mesh within a domain whilst primarily visualising another field or a specific cut.

Controls for manipulating the object within the view window are the same as for the majority of visualisation packages - left click to turn the object, right click or use the mouse wheel to zoom, centre-click to drag the object. A useful trick after manipulating the object is to return to the Object Inspector's 'Display' page and use the 'Zoom to data' button to return the view zoom to a whole-object value. The 'Camera Controls' toolbar, as in Mayavi, allows you to reset the view to up or down an axis, but new to Mayavi users may be the three extra buttons which allow you to define a different rotation centre around which your mesh is rotated when you move it, and the subsequent ability to re-centre the rotation automatically around the centre of your mesh. 

\subsubsection{Stretching along an axis}

In many cases you may have data which has a very high aspect ratio in the original data that hinders interpretative visualisation, and want to scale the data along one axis. This is very simple to do on the x, y, and z axes within paraview by selecting the 'Display' tab in the 'Object Inspector' frame (make sure the object is selected above, otherwise you will not find any options under the Display tab) and scrolling down to the end of the frame where you'll find a 'Transformation' section, as shown on figure \ref{paraview8}.

\begin{figure}[h!]
\label{paraview8}
  \centering
  \fig[width=.7\textwidth]{visualisation_and_diagnostics_images/Paraview-rescaling-view}
\end{figure}

Of interest here is the row labelled 'Scale' which gives scaling factors for, in turn, the X, Y, and Z directions. Change the values in these boxes to rescale your data.

Note that if you have set up a pipeline of actions (loading a file, taking a slice, contouring the slice, etc) and you change the scaling on one object in the pipeline you will find that the other objects in the pipeline look out of scale and you have to change the scaling for each one in turn.

Additionally, there appears to be a bug when using the GL 'volume' rendering for a dataset where the scaling is not honoured. In this case, switching to something like near-transparent wireframe coloured by your field of interest may be the best option. 

\subsubsection{Changing Opacity}

You may want to be able to see multiple elements in the pipeline but have most of them sufficiently transparent that you can see through them to the main element of interest. To do this, reset the opacity of your object from the 'Style' panel part way down the 'Object Inspector' frame (see fig. \ref{paraview9} below).

\begin{figure}[h!]
\label{paraview9}
  \centering
  \fig[width=.7\textwidth]{visualisation_and_diagnostics_images/Paraview-opacity2}
\end{figure}

\subsubsection{Filters}

Paraview offers a wide range of filters which can be applied to active data objects. Upon selection of a filter, the user is asked to accept the usage of the filter, which is then directly applied to the previously highlighted data object in the pipeline browser. During usage, the filter parameters can be adjusted in its property sheet, again located immediately below the pipeline browser. Display settings can thus be improved and useful information on the filter characteristics are here presented in a user-friendly way. Some of the most used filters are available on a separate "filters toolbar" located immediately above the pipeline browser.

Table \ref{ParaviewFilters} gives a very brief overview of some of the filters which are usually the most useful when running a test case (see chapter \ref{chap:examples}). For a more in-depth account on using filters and a summary on the various display and 3D view properties supported by paraview, please refer to the online users guide:
\url{www.paraview.org/files/v1.6/ParaViewUsersGuide.PDF}

\begin{center}
\begin{table}[h!]
\begin{tabular}{cp{110mm}}
{\bfseries Filter name} & {\bfseries\centering Main function} \\
\hline
CONNECTIVITY & Extract the largest "connected region" in the mesh data set. \\
TRIANGULATE & Change the polygons in a mesh to triangles. \\ 
DECIMATE & Reduce the number of triangles in a triangle mesh, keeping a good approximation - this needs to be applied on top of TRIANGULATE. \\
QUADRIC CLUSTERING & See DECIMATE, but here the topology is affected. \\
FEATURE EDGES & Extract special edge types from the mesh input polygonal data. \\
LINEAR EXTRUSION & Create solid objects from 2D polygonal meshes. \\
MASK POINTS & Render vertices only (not points). \\
SHRINK & Disconnect the cells from one another. \\
SMOOTH & Laplacially smooth down the mesh, "relaxing" it, i.e. make the cell shapes nicer and distribute vertices more evenly. \\
STREAM TRACER & Generate stream traces in a vector field starting from a set of "seed points". \\
\end{tabular}
\label{ParaviewFilters}
\caption{Some useful filters and their respective function when applied to a selected active data object.}
\end{table}
\end{center}






\section{Visualisation: Making a movie}
\label{sect:making_a_movie}

The first step to making a movie is creating .png files of the frames that you wish to include in the movie. The easiest way to do this is to first set up the scene in Mayavi, ie open one of your .vtus in Mayavi and open the SurfaceMap or whichever filter you want, select the field that you want to visualise and add any axes or legends that you wish to include. You then save the visualisation by clicking on "save" and then "entire visualisation". A .mv file should be saved. The .png files can now be produced using the following command. The numbers at the end are the numbers of the .vtu files at the beginning and end of the movie, so this command would produce .pngs of all the .vtus numbered between 1 and 5.
\begin{lstlisting}[language=bash]
xvfb-run --auto-servernum --wait=4 --server-args="-screen 0 1024x1024x24" 
<<fluidity_source_path>>/scripts/fl_mayavi_animate.py -f <<name of your 
.vtu files without number or extension>> -z <<name of your .mv file with 
extension>> 1 5
\end{lstlisting}
Once you have the .pngs, you could use the convert command to make a .gif movie as follows.
\begin{lstlisting}[language=bash]
convert *.png movie.jpg
\end{lstlisting}
This may not play on some movie players including many windows packages, so please check it on the system you intend to play it on.

If you want to produce a .avi movie that will play on windows, you can use the script  encode.bash in the scripts directory. This requires the following four arguements.
\begin{enumerate}
\item image file type (e.g. jpg,png,etc...)   
\item path to folder containing images, and for movie output
\item name of movie output file (without the file extension)
\item OPTIONAL - enter characters here to limit the image files used as frames (e.g. 'image1' would only select images which satisfy 'image1*.jpg')
\end{enumerate}
So, for example, you could type this:
\begin{lstlisting}[language=bash]
<<fluidity source path>>/scripts/encode.bash jpg /data/movie_files my_movie 
image002
\end{lstlisting}
You can change the frames per second or bitrate in the encode.bash file. Please see the file for instructions.


%%%%%%%%%%%%%%%%%%%%%%%%%%%%%%%%%%%%%%%%%%%%%%%%%%%%%%%%%%%%%%%%%%%
%---------------------ONLINE DIAGNOSTICS--------------------------%
%%%%%%%%%%%%%%%%%%%%%%%%%%%%%%%%%%%%%%%%%%%%%%%%%%%%%%%%%%%%%%%%%%%

\section{Online diagnostics}
\label{sect:online_diagnostics}

\subsection{Fields}
\label{sect:diagnostics_fields}

\subsubsection{Internal diagnostic fields}
\fluidity\ has a set of predefined diagnostic fields called internal diagnostic fields. These diagnostic fields can be scalar, vector and tensor fields. Common used internal diagnostic field are the CFLNumber or the Gradient of a specified scalar\_field.


Each internal diagnostic field has a unique identifier and are classified by their field type: \option{scalar\_field}, \option{vector\_field}
or \option{tensor\_field}. To configure a internal diagnostic field, add a new field of the appropriate type and select the identifier. A description of the available diagnostic fields is given below. For example, to add the CFLNumber (which is a \option{scalar\_field}), one would add a new \option{scalar\_field} and select \option{scalar\_field (CFLNumber)}. 

Some internal diagnostics contain a \option{\ldots/diagnostic/field\_name}
attribute defining the field used to compute the diagnostic (for example the
field used to compute a gradient). The internal diagnostics do not have a dependency resolution, that is if this source field is itself a diagnostic
field it may happen that the source field is not computed yet. In such a case, one should try and use diagnostic algorithms instead, see next section.

In the following, a description of the internal diagnostics available in \fluidity\ is given. 

Internal \option{scalar\_field} diagnostics:

\begin{description}
\item[CFLNumber:]CFLNumber. See \url{http://amcg.ese.ic.ac.uk/index.php?title=Local:Diagnostics\#CFL\_Number}.
	Adapting to this field is not recommended  
\item[ControlVolumeCFLNumber:]Number as defined on a control volume mesh. Adapting to this field is not recommended  
\item[DG\_CourantNumber:]Number as defined on a DG meshAdapting to this field is not recommended  
\item[DiscontinuityDetector:]Discontinuity detector takes value 1 where detector is triggered.
	Adapting to this field is not recommended  
\item[CVMaterialDensityCFLNumber:]Number as defined on a control volume mesh and incorporating the Material Density. Requires a MaterialDensity field!
Adapting to this field is not recommended  
\item[SolidConcentration:]  
\item[CopyofDensity:]This scalar field is meant to replace DENTRAF. Basically, if you use new options, DENTRAF is no longer needed. No repointing is done from this field to DENTRAF.  
\item[VisualizeSolidFluid:]Add field to be used by Solid\_configuration to Visualize the solids and MaterialVolumeFraction together  
\item[VisualizeSolid:]Add field to be used by Solid\_configuration to Visualize the solid\_Concentration  
\item[ParticleScalar:]Add field to be used by Solid\_configuration to map  the solid\_Concentration from particle mesh to the fluid mesh.  
\item[FunctionalBegin:]Add field to be used by Explicit\_ALE to visualize functional values before iterations start.  
\item[FunctionalIter:]Add field to be used by Explicit\_ALE to visualize functional values at each iteration.  
\item[MaterialVolume:]Add a MaterialVolume scalar\_field to calculate the spatially varying volume of a material (requires a MaterialVolumeFraction)  
\item[MaterialMass:]Add a MaterialMass scalar\_field to calculate the spatially varying mass of a material (requires a MaterialVolumeFraction and a MaterialDensity)  
\item[MaterialEOSDensity:]Calculates the MaterialDensity based on the bulk Pressure(and MaterialInternalEnergy if appropriate) for the equationof state of this material.  
\item[MaterialPressure:]Calculates the MaterialPressure based on the MaterialDensity(and MaterialInternalEnergy if appropriate) for the equationof state of this material.  
\item[BulkMaterialPressure:]Calculates the BulkMaterialPressure based on the MaterialDensityand MaterialVolumeFraction (and MaterialInternalEnergy if appropriate) for the equation of state of all materials.  
\item[GridReynoldsNumber:]Grid Reynolds number  
\item[GridPecletNumber:]The GridPecletNumber is $Pe = U*dx/2*\mbox{diffusivity}$.
	Also see the test case 'grid\_peclet\_number'if you wish to see the effect of changing the diffusivity on a 1D, cg-discretised tracer-field.
	Adapting to this field is not recommended  
\item[HorizontalVelocityDivergence:]Horizontal velocity divergence: ${div}_H \mbox{velocity}$. Uses the gravity field direction to determine the horizontal plane.  
\item[VelocityDivergence:]Velocity divergence: $div \mbox{ velocity}$  
\item[Vorticity2D:]Vorticity for a 2D field: $\frac{du}{dy} \frac{dv}{dx}$  
\item[KineticEnergyDensity:]Kinetic energy density: $\frac{1}{2} \rho_0 |u|^2$ where $\rho_0$ is the (reference) density.
	Limitations: The Density, PerturbationDensity, KineticEnergyDensity and Velocity fields must be on the same mesh.  
\item[GravitationalPotentialEnergyDensity:]Gravitational potential energy density: $\rho_0 (1.0 + \rho') (g \cdot (r - r_0))$ where $\rho_0$ is the (reference) density, $\rho'$ is the perturbation density and $r_0$ is the potential energy zero point.
	Limitations: \begin{itemize}
	\item Requires a constant gravity direction. 
	\item The Density, PerturbationDensity and GravitationalPotentialEnergyDensity fields must be on the same mesh. 
	\end{itemize}
\item[IsopycnalCoordinate:]Isopycnal coordinate
	$z_{star(x,t)} = \frac{1}{A} \int_{V'} H(\rho(x',t)-\rho(x,t)) dV'$ where $\rho$ is the density, $A$ is the width/area of the domain.
	Limitations:
	\begin{itemize}
	\item You need to specify a (fine) mesh to redistribute the PerturbationDensity onto.
	\item Requires a constant gravity direction. 
	\item The Density, PerturbationDensity and GravitationalPotentialEnergyDensity fields must be on the same mesh. 
	\end{itemize} 
\item[BackgroundPotentialEnergyDensity:]Background potential energy density: $PE_b = \rho z_{star}$ where rho is the density, $z_{star}$ is the isopycnal coordinateLimitations: 
	\begin{itemize}
	\item Requires a constant gravity direction. 
	\item The Density, PerturbationDensity and GravitationalPotentialEnergyDensity fields must be on the same mesh. 
	\end{itemize}
\item[PotentialVorticity:]Ertel potential vorticity: $(f + \mbox{curl } u) \cdot \mbox{grad } \rho'$
	Limitations: Requires a geometry dimension of 3.  
\item[RelativePotentialVorticity:]Relative potential vorticity:  $\mbox{curl } u \cdot \mbox{grad } \rho'$  
\item[MeshEdgeLengths:]Local average mesh edge lengths  
\item[HorizontalStreamFunction:]Calculate the horizontal stream function psi where:  $\partial_x \psi = -v$ and  $\partial_y \psi = u$ where $u$ and $v$ are perpendicular to the gravity direction. Applies astrong Dirichlet boundary condition of $0$ on all boundaries.  
\item[Speed:]Speed: $|u|$ 
	Limitations: The Speed and Velocity fields must be on the same mesh.  
\item[SolidPhase:]Volume of the vehicles used in Traffic Modelling  
\item[AbsoluteDifference:]Absolute Difference between two scalar fields. Both fields must be in this material\_phase. Assumes both fields are on the same mesh as the AbsoluteDifference field.  
\item[ScalarAbsoluteDifference:]Absolute Difference between two scalar fields. Both fields must be in this material\_phase. Assumes both fields are on the same mesh as the AbsoluteDifference field.  
\item[GalerkinProjection:]Galerkin projection of one field onto another mesh. The field must be in this material\_phase. 
	NOTE: you need the solver options if the meshof this field is continuous.  
\item[PrimaryProduction:]Primary production of Phytoplankton. This is calculated bythe ocean biology module and will not be calculated unlessocean biology is being simulated.  
\item[PhytoplanktonGrazing:]Grazing of Phytoplankton by Zooplankton. This is calculated bythe ocean biology module and will not be calculated unlessocean biology is being simulated.  
\item[TidalRange]  
\item[MaxFreeSurface]  
\item[MinFreeSurface]  
\item[HarmonicAmplitudeM2]  
\item[HarmonicPhaseM2]  
\item[HarmonicAmplitudeS2]  
\item[HarmonicPhaseS2]  
\item[HarmonicAmplitudeN2]  
\item[HarmonicPhaseN2]  
\item[HarmonicAmplitudeK2]  
\item[HarmonicPhaseK2]  
\item[HarmonicAmplitudeK1]  
\item[HarmonicPhaseK1]  
\item[HarmonicAmplitudeO1]  
\item[HarmonicPhaseO1]  
\item[HarmonicAmplitudeP1]  
\item[HarmonicPhaseP1]  
\item[HarmonicAmplitudeQ1]  
\item[HarmonicPhaseQ1]  
\item[HarmonicAmplitudeMf]  
\item[HarmonicPhaseMf]  
\item[HarmonicAmplitudeMm]  
\item[HarmonicPhaseMm]  
\item[HarmonicAmplitudeSSa]  
\item[UniversalNumber:]Output the universal numbering of the mesh on which this field is based.  
\item[NodeOwner:]Output the processors which own the nodes of the mesh on which this field is based.  
\item[ElementOwner:]Output the processors which own the elements of the mesh on which this field is based.  
\item[HarmonicPhaseSSa:]Primary production of Phytoplankton. This is calculated bythe ocean biology module and will not be calculated unlessocean biology is being simulated.  
\end{description}

The available internal \option{vector\_field}  diagnostics are:

\begin{description}

 \item[TemperatureGradient :] Temperature gradient    
 \item[Gradient :] Gradient of a scalar field. Field must be in this material\_phase.    
 \item[FiniteElementGradient :] Gradient of a scalar field evaluated using the C gradientmatrix constructed using finite elements.Field must be in this material\_phase.    
 \item[FiniteElementDivergenceTransposed :] Gradient of a scalar field evaluated using the transposeof the $C^T$ divergence matrix constructed using finiteelements. Field must be in this material\_phase.    
 \item[Vorticity :] Relative vorticity field - curl of the velocity field    
 \item[PlanetaryVorticity :] Planetary vorticity
	Limitations: Requires geometry dimension of 3.    
 \item[AbsoluteVorticity :] Absolute vorticity:  $f + \mbox{curl } u$ 
	Limitations: Requires a geometry dimension of 3.    
 \item[ControlVolumeDivergenceTransposed :] Gradient of a scalar field evaluated using the transposeof the $C^T$ matrix constructed using control volumes. Field must be in this material\_phase.    
 \item[InnerElementFullVelocity :] Full velocity in an inner element SGS treatment of momentum
	Limitations: \begin{itemize}
	\item Requires a geometry dimension of 3.
	\item Requires inner element active for momentum.
	\end{itemize}    
 \item[InnerElementFullVorticity :] Vorticity of the full velocity in aninner element SGS treatment of momentum
	Limitations: \begin{itemize}
	\item Requires a geometry dimension of 3.
	\item Requires inner element active for momentum.
	\end{itemize}     
 \item[InnerElementVorticity :] Vorticity of the SGS velocity in aninner element SGS treatment of momentum
	Limitations: \begin{itemize}
	\item Requires a geometry dimension of 3.
	\item Requires inner element active for momentum.
	\end{itemize}    
 \item[DgMappedVelocity :] The continuous solution mapped to a discontinuous mesh.
	Limitations: 
	\begin{itemize}
	\item Requires a geometry dimension of 3.
	\item Requires inner element active for momentum.
	\end{itemize}    
 \item[DgMappedVorticity :] Vorticity of the DG mapped Velocity. Note vorticity is actually calculated over a DG field.
	Limitations:
	\begin{itemize}
	\item Requires a geometry dimension of 3.
	\item Requires inner element active for momentum.
	\end{itemize}    
 \item[SolidVelocity :] Solid Velocity field.  Used to generate the momentum source     
 \item[ParticleVector :] Same as Solid Velocity field but it is on the Particle mesh. It is used to map the velocities coming from an external program like FEMDEM or DEM to the fluid mesh.     
 \item[ParticleForce :] Same as Solid Velocity field but it is on the Particle mesh. It is used to map the velocities coming from an external program like FEMDEM or DEM to the fluid mesh.     
 \item[SolidForce :] Same as Solid Velocity field but it is on the Particle mesh. It is used to map the velocities coming from an external program like FEMDEM or DEM to the fluid mesh.     
 \item[VelocityPlotForSolids :]     
 \item[FunctionalGradient :] Same as Solid Velocity field but it is on the Particle mesh. It is used to map the velocities coming from an external program like FEMDEM or DEM to the fluid mesh.     
 \item[LinearMomentum :] LinearMomentum field: $p = \rho u$ (where $p$ is the linear momentum, $\rho$ the density and $u$ the velocity)    
 \item[ControlVolumeAuxiliaryGradient :] Calculate the control volume auxiliary gradient for a particular field. The related field must be a scalar field in this material\_phase.    
 \item[DGAuxiliaryGradient :] Calculate the dg (Bassi Rebay) auxiliary gradient for a particular field. The related field must be a scalar field in this material\_phase.    
 \item[GeostrophicSource :] Experimental geostrophic source field to be used in combination witha free surface. Does not solve a vertical balance yet.    
 \item[AbsoluteDifference :] Absolute Difference between two vector fields. Both fields must be in this material\_phase. Assumes both fields are on the same mesh as the AbsoluteDifference field.    
 \item[VectorAbsoluteDifference :] Absolute Difference between two vector fields. Both fields must be in this material\_phase. Assumes both fields are on the same mesh as the AbsoluteDifference field.    
 \item[BedShearStress :] $\mbox{Bed Shear Stress} = \mbox{density}*\mbox{drag\_coeff}*|u|*u$. At the moment this assumes the density and drag coefficients are constants. This diagnostic vector field is only calculated over surface elements/nodes, interior nodes will have zero value.    
 \item[MaxBedShearStress :] Max Bed Shear Stress. Note that you need BedShearStress turned on for this to work.    
 \item[DiagnosticCoordinate :] Coordinate field remapped to the mesh of your choice.    
 \item[Displacement :] Displacement    
 \item[GalerkinProjection :] Galerkin projection of one field onto another mesh. The field must be in this material\_phase. NOTE: you need the solver options if the meshof this field is continuous.    
 \item[Coriolis :] Projects the Coriolis term onto the mesh of this diagnostic field. Note that multiple projection methods are available (under the algorithm option).    
 \item[ImbalancedVelocity :] Compute the imbalanced component of velocity,that is,$u - u_{bal}$ where $u_{bal}$ is the velocity that puts the state in geostrophicbalance. Note: if your VelocityMesh is continuous, then the solver option is necessary.    
 \item[BalancedVelocity :] Compute the balanced component of velocity,that is, the velocity that puts the state in geostrophicbalance. This diagnostic depends on ImbalancedVelocity.    
\end{description}


And the internal \option{tensor\_field}  diagnostics are:

\begin{description}
\item[FieldTolerance] From a field on a mesh, diagnose the anisotropic interpolation weight that would give the mesh back. It is computed as: $\epsilon = M^{-1} |H|$ 
\end{description}


\subsubsection{Diagnostic algorithms}

The name of each field for each material / phase  in the options tree must be
unique. Hence there can only be one field named ``Gradient'' in a single
material /phase. The concept of a diagnostic algorithm is designed to solve this
issue - multiple fields, each with their own name, can share the same algorithm.
This, for example, allows the gradient of multiple fields to be calculated for a
single material / phase.

To configure a diagnostic field using a diagnostic algorithm, select the
\option{\ldots/diagnostic} option for a generic \option{scalar\_field}, \option{vector\_field}
or \option{tensor\_field}. This contains a \option{\ldots/diagnostic/algorithm}
choice element from which you can select the diagnostic algorithm.

Some diagnostic algorithms contain a \option{\ldots/diagnostic/algorithm/source\_field}
attribute defining the field used to compute the diagnostic (for example the
field used to compute a gradient). If this source field is itself a diagnostic
field defined in terms of a diagnostic algorithm then the source field is
computed first (dependency resolution).
In the majority of cases, a \option{scalar\_source\_field}, \option{vector\_source\_field}, \option{tensor\_source\_field}
or \option{component\_source\_field} attribute is defined. This identifies the
expected type of input field. \option{component\_source\_field} denotes scalar field
input, but for which vector or tensor field components of the form \option{field\_name\%comp}
can be used.
The attribute \option{\ldots/diagnostic/algorithm/material\_phase\_support}, which may take the
value ``single'' or ``multiple'', defines if the diagnostic algorithm may
access fields in other material / phases. For multiple \option{material\_phase\_support}
diagnostic fields, a source field in another material / phase
may be defined by a :: delimited ``state\_name\::field\_name'' string.

\begin{figure}[ht]
  \centering
  \fig[width=0.7\textwidth]{visualisation_and_diagnostics_images/DiagnosticAlgorithms}
  \caption{Configuration of a diagnostic field using a diagnostic algortithm in
           Diamond. Here a pressure gradient diagnostic is defined.}
  \label{fig:diagnostic_algorithm}
\end{figure}

\subsubsection{Python diagnostic algorithms}

A python diagnostic algorithm, chosen via \option{\ldots/diagnostic/algorithm::scalar\_python\_diagnostic}
(or similar equivalents for vector and tensor fields) allows direct access to the
internal \fluidity\ data structures in the computation of a diagnostic field. The
python code entered at \option{\ldots/diagnostic/algorithm::scalar\_python\_diagnostic}
can access three variables: the simulation timestep \lstinline[language = Python]*dt*,
the diagnostic field \lstinline[language = Python]*field*, and the simulation state
\lstinline[language = Python]*state*. \lstinline[language = Python]*field* and
\lstinline[language = Python]*state* are python representations of the internal
\fluidity\ data structures - see \citet{femtools} for more complete documentation.

\begin{example}
\begin{lstlisting}[language = Python]
deltaT = 4.0

t = state.scalar_fields["Temperature"]
assert(t.node_count == field.node_count)

for i in range(field.node_count):
        tMinusT0 = t.node_val(i) * deltaT
        diagVisc = 1.620e-2 * (1.0 - 2.79e-2 * tMinusT0 + 6.73e-4 * tMinusT0 * tMinusT0)
        visc = numpy.zeros((3, 3))
        for j in range(3):
                visc[j][j] = diagVisc
        field.set_val_at_node(i, visc)
\end{lstlisting}
\caption{A tensor python diagnostic algorithm defining a temperature varying
         viscosity used in a baroclinic annulus simulation, configured
         as in \citet{hignett1985} table 1 (main comparison).}
\end{example}

\subsection{The stat file}
\label{sect:diagnostics_stat_file}

\subsection{Overview}
The stat file contains information about the simulation, collected at run time. These diagnostics can be extracted from it using the stat\_parser, section \ref{sect:diagnostic_output} and it can be quickly and easily visualised with statplot, section \ref{sect:statplot}. Note, for parallel runs, unless otherwise stated the values have been calculated in a `parallel-safe' manner.

The diagnostics that are recorded in the stat file for each field are selected by the user.  What is included should be considered carefully as including a lot of information can make a notable increase in the simulation run time. To configure the stat file locate the \option{\ldots/stat} element for a generic \option{scalar\_field}, \option{vector\_field} or \option{tensor\_field}, e.g. Figure \ref{fig:diamond_enable_stat}. This contains further elements that will allow the configuration of the stat file as outlined in table \ref{tab:stat_file_diagnostics}. Diagnostics that are more involved and require a longer description are listed in table \ref{tab:stat_file_diagnostics} and documented in section \ref{sect:stat_diagnostics}.

\begin{figure}[ht]
  \centering
  \fig[width=0.7\textwidth]{visualisation_and_diagnostics_images/diamond_enable_stat}
  \caption{Example configuration of the stat file for \option{\ldots/vector\_field(Velocity)}.}
  \label{fig:diamond_enable_stat}
\end{figure}

The diagnostics will be output at every time step, from the end of the first time step onwards and, where relevant, are output before the mesh is adapted. The following options, regarding when the diagnostics are output, may also be chosen by activating the following elements in the Diamond file (the names are self-explanatory):
\begin{itemize}
\item \option{/io/stat/output\_at\_start}
\item \option{/io/stat/output\_before\_adapts}
\item \option{/io/stat/output\_after\_adapts}
\end{itemize}  

It is also possible to include the values from the previous time step and nonlinear iterations for a vector field by picking the choice elements:
\begin{itemize}
\item \option{\ldots/stat/previous\_time\_step/include\_in\_stat}
\item \option{\ldots/stat/nonlinear\_field/include\_in\_stat}
\end{itemize}

and for a scalar field by activating the elements:

\begin{itemize}
\item \option{\ldots/stat/include\_nonlinear\_field}
\item \option{\ldots/stat/include\_previous\_time\_step}
\end{itemize}

\subsection{Stat file diagnostics}
\label{sect:stat_diagnostics}

\begin{landscape}
\begin{longtable}{|p{0.2\textwidth}|p{0.18\textwidth}|p{0.10\textwidth}|p{0.4\textwidth}|p{0.35\textwidth}|}
\hline
\centering
%-----------------------------------------------------------------------------------------------------------------------------------%
Name			& Statistic		& Material phase name	& Diamond information			& Notes \\
%-----------------------------------------------------------------------------------------------------------------------------------%
\hline \multicolumn{5}{|p{1.25\textwidth}|}{{\bf File format information}} \\ \hline
%-----------------------------------------------------------------------------------------------------------------------------------%
format			& binary or plain\_text	&			& always included			& the .stat file will be in plain text unless \option{/io/detectors/binary\_output} is switched on  \\
real\_size		& real\_size		&			& \option{/io/detectors/binary\_output}	& size of real \\
integer\_size		& integer\_size		&			& \option{/io/detectors/binary\_output}	& size of an integer \\
\hline \multicolumn{5}{|p{1.5\textwidth}|}{{\bf Mesh diagnostics} e.g. \option{/geometry/mesh::CoordinateMesh/from\_file}} \\ \hline
CoordinateMesh		& nodes			& 			& \option{\ldots/stat/include\_in\_stat}	& number of nodes in the mesh \\
CoordinateMesh		& elements		&			& \option{\ldots/stat/include\_in\_stat}	& number of elements in the mesh \\
CoordinateMesh		& surface elements	&			& \option{\ldots/stat/include\_in\_stat}	& number of surface elements in the mesh \\
%-----------------------------------------------------------------------------------------------------------------------------------%
\hline \multicolumn{5}{|p{1.25\textwidth}|}{{\bf Machine statistics}} \\ \hline
%-----------------------------------------------------------------------------------------------------------------------------------%
FluidityVersion		& \fluidity\ version	&			& always included			& \fluidity\ version \\
CompileTime		& date and time		&			& always included			& compile date and time \\
StartTime		& date and time		&			& always included			& simulation start date and time \\
HostName		& hostname		&			& always included			& name of host machine,
default "Unknown" \\ 
%-----------------------------------------------------------------------------------------------------------------------------------%
\hline \multicolumn{5}{|p{1.25\textwidth}|}{{\bf Memory diagnostics} - these are only included if \fluidity\ is configured with either - -enable-debugging or - -enable-memory-stats. For parallel runs they are over all processors/} \\ \hline
%-----------------------------------------------------------------------------------------------------------------------------------%
memory type		& current		& Memory		& n/a					& current memory usage \\
memory type		& min			& Memory		& n/a					& minimum memory usage during the last time step \\
memory type		& max			& Memory		& n/a					& maximum memory usage during the last time step \\
%-----------------------------------------------------------------------------------------------------------------------------------%
\hline \multicolumn{5}{|p{1.25\textwidth}|}{{\bf Time diagnostics}} \\ \hline
%-----------------------------------------------------------------------------------------------------------------------------------%
Elapsed time		& value			&			& always included			& current simulation time \\
dt			& value			&			& always included			& time step used for the previous time step\\
Elapsed wall time	& value			&			& always included			& how long, in real, wall clock time the simulation has been running\\
%-----------------------------------------------------------------------------------------------------------------------------------%
\hline \multicolumn{5}{|p{1.25\textwidth}|}{{\bf Scalar field diagnostics} e.g. for \option{/material\_phase::fluid/scalar\_field::Temperature/prognostic}  } \\ \hline
%-----------------------------------------------------------------------------------------------------------------------------------%
Temperature		& min			& fluid			& \option{\ldots/stat/include\_in\_stat} & minimum of the scalar field \\
Temperature		& max			& fluid			& \option{\ldots/stat/include\_in\_stat} & maximum of the scalar field \\
Temperature		& l2norm		& fluid			& \option{\ldots/stat/include\_in\_stat} & \Ltwo norm of the scalar field over the mesh the scalar field is on\\
Temperature		& integral		& fluid			& \option{\ldots/stat/include\_in\_stat} & integral of the field over the mesh over the mesh the scalar field is on \\
Temperature		& cv\_l2norm		& fluid			& \option{\ldots/stat/include\_cv\_stats} & \Ltwo norm of the scalar field over the control volume dual mesh to the mesh the scalar field is on\\
Temperature		& cv\_integral		& fluid			& \option{\ldots/stat/include\_cv\_stats} & integral of the field over the control volume dual mesh to the mesh the scalar field is on\\
Temperature		& surface\_integral\% name& fluid		& \option{\ldots/stat/surface\_integral[0]} & section \ref{sect:stat_surface_integral} \\
Temperature		& mixing\_bins		& fluid			& \option{\ldots/stat/} \option{include\_mixing\_stats[0]} & section \ref{sect:stat_mixing_stats} \\
%-----------------------------------------------------------------------------------------------------------------------------------%
\hline \multicolumn{5}{|p{1.25\textwidth}|}{{\bf Vector field diagnostics} e.g. for \option{/material\_phase::fluid/vector\_field::Velocity/prognostic}. The values of component will range from 1 to number of dimensions. The force, pressure force and viscous force statistics have not been rigorously tested in parallel} \\ \hline
%-----------------------------------------------------------------------------------------------------------------------------------%
Velocity\%magnitude	& min			& fluid			& \option{\ldots/stat/include\_in\_stat} & minimum of the magnitude of the vector field \\
Velocity\%magnitude	& max			& fluid			& \option{\ldots/stat/include\_in\_stat} & maximum of the magnitude of the vector field \\
Velocity\%magnitude	& l2norm		& fluid			& \option{\ldots/stat/include\_in\_stat} & \Ltwo norm of the magnitude of the vector field \\
Velocity\%component	& min			& fluid			& \option{\ldots/stat/include\_in\_stat} & minimum of component 1 of the vector field  \\
Velocity\%component	& max			& fluid			& \option{\ldots/stat/include\_in\_stat} & maximum of component 1 of the vector field \\
Velocity\%component	& l2norm		& fluid			& \option{\ldots/stat/include\_in\_stat} & \Ltwo norm of component 1 of the vector field \\
Velocity\%component	& integral		& fluid			& \option{\ldots/stat/include\_in\_stat} & integral of component 1 of the vector field over the mesh \\
Velocity		& surface integral\% name& fluid			& \option{\ldots/stat/surface\_integral[0]} & section \ref{sect:stat_surface_integral} \\
Velocity		& force			& fluid			& \option{\ldots/stat/} \option{compute\_body\_forces\_on\_surfaces} &  this requires a list of surface ids overwhich the total force is calculated???\\
Velocity		& pressure force	& fluid			& \option{\ldots/stat/} \option{compute\_body\_forces\_on\_surfaces/} \option{output\_terms}& pressure force over the surfaces with ids given in the attribute for \option{\ldots/stat/compute\_body\_forces\_on\_surfaces} ???  \\
Velocity		& viscous force		& fluid			& \option{\ldots/stat/} \option{compute\_body\_forces\_on\_surfaces/} \option{output\_terms}& outputs the viscous force over the surfaces with ids given in the attribute for \option{\ldots/stat/compute\_body\_forces\_on\_surfaces} ??? \\
%-----------------------------------------------------------------------------------------------------------------------------------%
\hline
\caption[Stat file diagnostics]{Stat file diagnostics  \newline This table contains the `Name' of the diagnostic, `Statistic' and `Material phase name' as they will appear in the stat file, section \ref{sect:diagnostic_output}. `Diamond information' contains the spud path to locate where the option is in Diamond. Finally `Notes' offers information about the diagnostic.}
\label{tab:stat_file_diagnostics}
\end{longtable}
\end{landscape}

\subsubsection{Surface integrals}
\label{sect:stat_surface_integral}

The surface integral diagnostics allow the calculation of surface integrated quantities
for arbitrary scalar or vector fields. The options can be found for any scalar
or vector field at \option{\ldots/stat/surface\_integral}. All surface integral diagnostics are parallelised.

\textbf{Scalar fields}

There are two types of surface integrals for scalar fields, \option{value} and
\option{gradient\_normal}. These are selected in the \option{type} attribute of
\option{\ldots/stat/surface\_integral}.

The surface integral type \option{value} calculates the surface integral of the
scalar field:

\begin{equation}
\int_{\partial \Omega} T,
\end{equation}

and the surface integral type \option{gradient\_normal} calculates surface integral
of the dot product of the gradient of the field with the surface normal:

\begin{equation}
\int_{\partial \Omega} \nabla T \cdot \vec{n}.
\end{equation}

The \option{gradient\_normal} surface integral is calculated using the volume
element shape function derivatives, via:

\begin{equation}
\sum_{i,j} \int_{\partial \Omega} {T_i \frac{\partial \phi_i}{\partial x_j} n_j}.
\end{equation}

\textbf{Vector fields}

There is one type of surface integral for vector fields, \option{value}. This
calculates the surface integral of the dot product of the field with the surface
normal:

\begin{equation}
\int_{\partial \Omega} \vec{v} \cdot \vec{n}.
\end{equation}

\textbf{Surface integral options}

The \option{name} attribute must be set for all diagnostic surface integrals. In
addition to this, surface IDs (see section \ref{sect:surface_ids}) may be specified at
\option{\ldots/stat/surface\_integral/surface\_ids}. If specified, the surface
integral is computed over just these surfaces. If it is disabled the integral is computed
over the whole surface of the
mesh. If the element \option{\ldots/stat/surface\_integral/normalise} is activated
the integral is normalised by dividing by the surface area.

\subsubsection{Mixing stats}
\label{sect:stat_mixing_stats}
Mixing stats calculates the volume fraction of the scalar field in a set of `bins' the bounds of which are specified by the user. 

The mixing stats can be calculated using the control-volume mesh or for \Poo using the mesh that the scalar field is on. This is specified by setting the choice element under \option{\ldots/stat/include\_mixing\_stats[0]} to either \option{continuous\_galerkin} or \option{control\_volumes}. The bin bounds must also be specified.  The element \option{\ldots/stat/include\_mixing\_stats[0]/mixing\_bin\_bounds} requires a list of floats that are the values of the bin bounds e.g. if the list reads 0.0 1.0 2.0 3.0 4.0 then 5 bins will be returned with ($T$: the scalar field):
\begin{itemize}
\item size bin 1 $=$ volume fraction of domain with $0.0\leq T < 1.0$
\item size bin 2 $=$ volume fraction of domain with $1.0\leq T < 2.0$
\item \ldots 
\item size bin 5 $=$ volume fraction of domain with $3.0\leq T$
\end{itemize}

The volume fractions can be normalised by the total volume of the domain by activating the element \option{\ldots/stat/include\_mixing\_stats[0]/control\_volumes/normalise}. The `tolerance' beneath the bin bounds for which the scalar field should be included can be specified by activating the element \option{\ldots/stat/include\_mixing\_stats[0]/control\_volumes/tolerance} (i.e. $1.0-\mathrm{tolerance} \leq T < 2.0-\mathrm{tolerance}$). If not selected it defaults to machine tolerance epsilon(0.0). For an example of using the mixing stats see the test case in the \fluidity\ trunk \lstinline[language = bash]+.../tests/lock_exchange_2d_cg+ or \lstinline[language = bash]+.../tests/cv_mixing_bin_test_serial/+.

\subsection{Detectors}
\label{sect:diagnostics_detectors}
The detectors file contains information about the positions of the detectors which is the same for all time steps in the case of static detectors and change at each time step for the Lagrangian detectors as they are advected by the flow. It also contains information about the values of different flow variables at the positions of the detectors. Both, position and variables information are collected at run time.

The user selects which field variables should be included in the detectors file by setting the corresponding option in Diamond. For example, if interested in the value of Temperature at the different detectors positions and at each time step, the option \option{\ldots/scalar\_field/prognostic/detectors} \option{/include\_in\_detectors} should be set. Only the fields of interest should be included since extra information will make the detectors file very large and more difficult to handle. If many detectors are present and/or information from many flow variables is required, it is recommended to set the \option{/io/detectors/binary\_output} in Diamond, since an ascii file will quickly become very large.

The information of the detectors can be extracted with stat\_parser and it can be visualised with statplot. 

The position of the detectors and value of the selected variables at those positions is saved into the output file at every time step, starting from the end of the first time step and it is done before the mesh is adapted. 

%%%%%%%%%%%%%%%%%%%%%%%%%%%%%%%%%%%%%%%%%%%%%%%%%%%%%%%%%%%%%%%%%%%
%---------------------OFFLINE DIAGNOSTICS-------------------------%
%%%%%%%%%%%%%%%%%%%%%%%%%%%%%%%%%%%%%%%%%%%%%%%%%%%%%%%%%%%%%%%%%%%

\section{Offline diagnostics}
\label{sect:offline_diagnostics}

There are three main types of offline diagnostics:
\begin{itemize}
\item fltools, section \ref{sect:fltools}: programs written in fortran that are compiled by running \lstinline[language = XML]+make fltools+ on the command line in the top directory of the \fluidity source tree. The F90 source files can be found in \lstinline[language = XML]+tools+
\item python scripts, section \ref{sect:diagnostics_useful_scripts}: scripts that can be run from the command line. They are found in \lstinline[language = XML]+scripts+
\item python modules: modules that can be imported e.g. for use in a test case. They are found in \lstinline[language = XML]+python/fluidity/diagnostics+ and are imported with  \\ \lstinline[language = python]+import python.fluidity.diagnostics.modulename+
\end{itemize}

\subsection{vtktools}
\label{sect:diagnostics_vtk_tools}

Vtktools allows you to analyse data from the vtu file using python. Initially, it will probably be easier to use the interactive interface ipython.  The script in Example \ref{ex:vtktools} imports the vtktools module into python and then reads in a  vtu file.  The information from the vtu file is now contained in the vtu object ug. Scalar and vector quantities can now be extracted. The pressure is extracted and put into a python array called p, and the velocity is put into the array uvw. 
\begin{example}
  \begin{lstlisting}[language=Python]
import sys
sys.path.append('<<fluidity_source_path>>/tools')
import vtktools
ug=vtktools.vtu('run_123.vtu')
# To extract information from a list of files use the following line 
# inside a loop
# ug=vtktools.vtu('run_123_'+str(i)+'.vtu')
ug.GetFieldNames()
p=ug.GetScalarField('Pressure')
uvw=ug.GetVectorField('Velocity')
  \end{lstlisting}
  \caption{A Python function to import the information from a vtu file into an object that python can handle. The pressure and valocity fields are converted into python arrays.}
  \label{ex:vtktools}
\end{example}

It is possible to add a field to the vtu object. This is done in Example \ref{ex:vtktools2}. The final line writes over the original vtu file with a new file which also contains the new field PressureSquared. The new vtu file can now be opened in Mayavi and the new field can be viewed in the normal way.  
\begin{example}
  \begin{lstlisting}[language=Python]
psq=p**2
ug.AddScalarField('PressureSquared', psq)
ug.Write()
\end{lstlisting}
  \caption{This follows on from the previous python script. It adds a new field to the vtu filed. }
\label{ex:vtktools2}
\end{example}
For more information about vtktools, see the help, which is obtained as shown in Example \ref{ex:vtktools4}.
\begin{example}
  \begin{lstlisting}[language=Python]
help('vtktools.vtu')  
\end{lstlisting}
\caption{How to get help with vtktools.}
  \label{ex:vtktools4}
\end{example}

Vtktools will output an array with the values of all the fields at each point of the mesh that was used in the simulation. This mesh could be unstructured or adapted so the  points would not be evenly distributed in space. For some diagnostics it may be desirable to have arrays that give the values of the fields at points that are evenly distributed in space. This can be done with the script in Example \ref{ex:vtktools3}.
\begin{example}
  \begin{lstlisting}[language=Python]
import sys
sys.path.append('<<fluidity_source_path>>/tools')
import vtktools
u=vtktools.vtu("run_123.vtu")
Xlist = arange(0.0,4.001,0.01)# x co-ordinates of the desired array shape
Ylist = arange(0.0,2.001,0.01)# y co-ordinates of the desired array shape
[X,Y] = meshgrid(Xlist,Ylist)
Z = 0.0*X # This is 2d so z is an array of zeros.
X = reshape(X,(size(X),))
Y = reshape(Y,(size(Y),))
Z = reshape(Z,(size(Z),))
zip(X,Y,Z)
pts = zip(X,Y,Z)
pts = vtktools.arr(pts)
# create arrays of velocity and temperature values at the desired points
vel = u.ProbeData(pts, 'Velocity') 
temperature_structured = u.ProbeData(pts, 'Temperature') 
  \end{lstlisting}
\caption{This creates an array of the desired shape and then creates arrays of the velocity and temperature fields at the points on that array.}
\label{ex:vtktools3}
\end{example}
 

\subsection{Diagnostic output}
\label{sect:diagnostic_output}

The primary output format for \fluidity\ diagnostics is the .stat file. There
are two file formats used for storing data in .stat files: a plain text format
and a binary format.

\subsubsection{Plain text .stat file format}

A plain text .stat file consists of two sections: an XML header section and a
data section. The header section appears at the start of the .stat file within
\lstinline[language = XML]*<header>* $\ldots$
\lstinline[language = XML]*</header>* tags, and defines all meta data concerning
all statistics contained in the data section. The data section contains a number of lines,
with each line containing a single data point for each of the statistics defined in the header
section.

The header element contains 
\lstinline[language = XML]*<constant>* and
\lstinline[language = XML]*<field>* child elements. The
\lstinline[language = XML]*<constant>* elements contain data relevant to the
entire simulation, such as the \fluidity\ version and simulation start date:

\begin{lstlisting}[language = XML]
  <constant name = "field_name"
    type = "field_type" value = "field_value"/>
\end{lstlisting}

where the \lstinline[language = XML]*type* attribute defines the data type
(one of "string", "integer" or "float"). A plain text .stat file defines one
additional constant element:

\begin{lstlisting}[language = XML]
<constant name="format" type="string" value="plain_text"/>
\end{lstlisting}

The \lstinline[language = XML]*<field>* elements contain meta-data for statistics
contained in the data section:

\begin{lstlisting}[language = XML]
  <field name = "field_name"
    statistic = "statistic_name" column = "field_column"
    [material_phase = "material_phase_name"]
    [components = "field_components"]/>
\end{lstlisting}

For statistics of scalar, vector or tensor fields, 
\lstinline[language = XML]*statistic* defines the statistic name for
a field \lstinline[language = XML]*name* in material / phase
\lstinline[language = XML]*material_phase*. For other objects (such as mesh
statistics) the \lstinline[language = XML]*material_phase* attribute may not
be supplied. The \lstinline[language = XML]*column* attribute is an integer defining the
index of the first component of the field data in each line of the data section.
The optional \lstinline[language = XML]*component* attribute defines the number
of field components (and defaults to one if not supplied). 

\begin{example}
\begin{lstlisting}[language = XML]
<header>
<constant name="FluidityVersion" type="string" value="11780M" />
<constant name="CompileTime" type="string" value="Nov 25 2009 09:38:20" />
<constant name="StartTime" type="string" value="20091125 095448.326+0000" />
<constant name="HostName" type="string" value="Unknown" />
<constant name="format" type="string" value="plain_text" />
<field column="1" name="ElapsedTime" statistic="value"/>
<field column="2" name="dt" statistic="value"/>
<field column="3" name="ElapsedWallTime" statistic="value"/>
<field column="4" name="CoordinateMesh" statistic="nodes"/>
<field column="5" name="CoordinateMesh" statistic="elements"/>
<field column="6" name="CoordinateMesh" statistic="surface_elements"/>
<field column="7" name="Tracer" statistic="min" material_phase="Fluid"/>
<field column="8" name="Tracer" statistic="max" material_phase="Fluid"/>
<field column="9" name="Tracer" statistic="l2norm" material_phase="Fluid"/>
<field column="10" name="Tracer" statistic="integral" material_phase="Fluid"/>
</header>
1.0 1.0 0.108 41 40 2 0.0 0.0 0.0 0.0
2.0 1.0 0.146 41 40 2 0.0 0.0 0.0 0.0
\end{lstlisting}
\caption{A simple plain text .stat file}
\end{example}

\subsubsection{Binary .stat file format}

The binary .stat file format contains an XML header section stored in a plain
text file, and a data section stored in binary in
a separate file. The file name of the binary data file is equal to the
header file name plus an additional \textit{.dat} file extension. The XML
header section is identical to the plain text XML header, except that the
\lstinline[language = XML]*format* constant element is replaced with:

\begin{lstlisting}[language = XML]
<constant name="format" type="string" value="binary"/>
\end{lstlisting}

The binary .stat file format also defines two additional constant elements:

\begin{lstlisting}[language = XML]
<constant name="real_size" type="integer" value="real_byte_size"/>
<constant name="integer_size" type="integer" value="integer_byte_size"/>
\end{lstlisting}

defining the size (in bytes) of a real and an integer in the data section.

At present the data section of binary .stat files contains only floating point
data.

\subsubsection{Reading .stat files in python}
\label{sect:diagnostics_stat_parser}

A .stat file can be read into a dictionary object using the
\lstinline[language = Python]*fluidity_tools* python module via:

\begin{lstlisting}[language = Python]
import fluidity_tools
stat = fluidity_tools.stat_parser(filename)
\end{lstlisting}

\lstinline[language = Python]*fluidity_tools.stat_parser* reads both
plain text and binary .stat file formats. For .stat files using the binary
format, \lstinline[language = Python]*filename* should correspond to the XML
header file.

You can find out which fields are contained in a state Fluid via:

\begin{lstlisting}[language=Python]
stat["Fluid"].keys()
\end{lstlisting}

The ``max'' statistics of a field ``Velocity\%magnitude'' in this state can be
plotted via:

\begin{lstlisting}[language=Python]
from matplotlib import pylab
time = stat["ElapsedTime"]["value"]
max_speed = stat["Fluid"]["Velocity%magnitude"]["max"]
pylab.plot(time, max_speed)
pylab.xlabel("Time")
pylab.ylabel("Max Speed")
pylab.show()
\end{lstlisting}

\subsection{Detectors output}
\label{sect:detectors_output}
Similarly to the .stat file, the detectors file formats are ascii or plain text format and binary format.

In plain text format, the .detectors file contains first an XML header followed by a section with the data. The header is contained within \lstinline[language = XML]*<header>* $\ldots$
\lstinline[language = XML]*</header>* tags and describes the content of each column in the data section. An example of the header is presented in \ref{header_det}.

\begin{example}
\begin{lstlisting}[language = XML]
<header>
<field column="1" name="ElapsedTime" statistic="value"/>
<field column="2" name="dt" statistic="value"/>
<field column="3" name="LAGRANGIAN_DET_000001" statistic="position" 
     components="3"/>
<field column="6" name="LAGRANGIAN_DET_000002" statistic="position" 
     components="3"/>
<field column="9" name="LAGRANGIAN_DET_000003" statistic="position" 
     components="3"/>
...
<field column="819348" name="Temperature" 
    statistic="LAGRANGIAN_DET_000001" material_phase="BoussinesqFluid"/>
<field column="819349" name="Temperature" 
    statistic="LAGRANGIAN_DET_000002" material_phase="BoussinesqFluid"/>
<field column="819350" name="Temperature" 
    statistic="LAGRANGIAN_DET_000003" material_phase="BoussinesqFluid"/>
...
<field column="983217" name="Velocity" statistic="LAGRANGIAN_DET_000001" 
    material_phase="BoussinesqFluid" components="3"/>
<field column="983220" name="Velocity" statistic="LAGRANGIAN_DET_000002" 
    material_phase="BoussinesqFluid" components="3"/>
<field column="983223" name="Velocity" statistic="LAGRANGIAN_DET_000003" 
    material_phase="BoussinesqFluid" components="3"/>
\end{lstlisting}

\caption{An example of the header in .detectors file}
\label{header_det}
\end{example}

when having the \option{/io/detectors/binary\_output} option set in Diamond, the header is stored in a plain text file with the .detectors extension and the data is stored in binary in another file with .detectors.dat extension.

In order to read the .detectors files in python, the \lstinline[language = Python]*fluidity_tools* python module described in \ref{sect:diagnostics_stat_parser} should be used.

\begin{lstlisting}[language = Python]
import fluidity_tools
detectors = fluidity_tools.stat_parser(filename)
\end{lstlisting}

As indicated earlier, this python module reads both
plain text and binary file formats. For .detectors files using the binary
format, \lstinline[language = Python]*filename* should also correspond to the XML header file.

It is possible to find out which fields are contained in a state Water with:

\begin{lstlisting}[language=Python]
stat["Water"].keys()
\end{lstlisting}

The Temperature versus time for a particular detector can be plotted with the following python lines:

\begin{lstlisting}[language=Python]
from matplotlib import pylab
time = detectors["ElapsedTime"]["value"]
Temp = detectors["Water"]["Temperature"]["LAGRANGIAN_DET_000001"]
pylab.plot(time, Temp)
pylab.xlabel("Time")
pylab.ylabel("Temperature")
pylab.show()
\end{lstlisting}

\subsection{fltools}
\label{sect:fltools}
An extended set of \fluidity\ tools exist that supplement the main
\fluidity\ program. Table \ref{tab:fltools} lists them and descriptions can
be found by referring to the relevant section. The tools can be built by
running \lstinline[language = bash]+make fltools+ in the top directory of
the \fluidity\ trunk. The programs generated can then be found in the
\lstinline[language = bash]+bin/+ directory.

\begin{table}
\begin{center}
  \begin{tabular}{| l | l |}
    \hline
	Program					& Section 				\\
    \hline
	create\_climatology\_atlas		& \ref{sect:create_climatology_atlas}	\\
	fladapt					& \ref{sect:fladapt}			\\
	fldiagnostics				& \ref{sect:fldiagnostics}		\\		
	flintegrate				& \\
	petsc\_readnsolve			& \ref{sect:petsc_readnsolve} 		\\
	project\_to\_continuous			& \ref{sect:project_to_continuous} 	\\
	streamfunction\_2d			& \ref{sect:streamfunction_2d} 		\\	
	test\_pressure\_solve			& \ref{sect:test_pressure_solve}	\\
	unifiedmesh				& \ref{sect:unifiedmesh} 		\\	
    	vertical\_integration			& \ref{sect:vertical_integration} 	\\
	vtkdiagnostic				& \ref{sect:vtkdiagnostic}		\\
	vtu\_bins           & \ref{sect:vtu_bins}		\\
	vtk\_projection				& \ref{sect:vtkprojection}		\\
    \hline
  \end{tabular}
\end{center}
\caption[Table of fltools]{Table of fltools. On running \lstinline[language = bash]+make fltools+ they can be found in the \lstinline[language = bash]+bin/+ directory of the \fluidity\ trunk.}
\label{tab:fltools}
\end{table}

%%%%%%%%%%%%%%%%%% CREATE CLIMATOLOGY ATLAS %%%%%%%%%%%%%%%%%%%%%%

\subsubsection{create\_climatology\_atlas}
\label{sect:create_climatology_atlas}

This creates a climatology atlas, for use with ICOM, using "High resolution ($1/4 ^\circ$) Temperature and Salinity Analyses of the World's Oceans. Version 2"

The $1/4 ^\circ$ grid climatological mean fields of in situ temperature (degrees Celsius) and salinity (practical salinity scale) for the annual, seasonal, and monthly time periods were calculated by \cite{boyer2005} using objective analysis techniques. The data and associated metadata was obtained from the NODC, \url{http://www.nodc.noaa.gov/OC5/WOA01/qd_ts01.html}. All the data files are gzipped ASCII 1/4 gridded data files ('Girdded Fields') and contain the DOS end-of-line character (M). There are 12 monthly averages of temperature and 12 monthly averages of salinity. In addition, there 4 seasonal averages of temperature and 4 seasonal averages of salinity corresponding to winter (defined as January, February and March), spring summer and autumn. Further information can be found online at the above address.

For use with ICOM (relaxing to climatology at the boundaries), a NetCDF data was created containing the monthly means, and the additional 9 standard levels from the seasonal means in order to provide information below the 1500m level. In addition, Killworth correction (Killworth 1995) is applied to the data in order to facilitate accutate time interpolation.

To use this just execute it in the directory which contains all the files listed above gunzipped.

%%%%%%%%%%%%%%%%%%%%%%%%% FLADAPT %%%%%%%%%%%%%%%%%%%%%%%%%%%%%%%%

\subsubsection{fladapt}
\label{sect:fladapt}
fladapt performs a single mesh adapt based on the input options file (which may be a checkpoint) and outputs the resulting mesh. It run from the command line:

\begin{lstlisting}[language = Bash]
fladapt [-v] INPUT OUTPUT
\end{lstlisting}
where \lstinline[language = Bash]+INPUT+ is the name of the input options file and \lstinline[language = Bash]+OUPUT+ is the name of the generated mesh. The flag \lstinline[language = Bash]+-v+ flag enables verbose output and the flag \lstinline[language = Bash]+-h+ flag displays the help message.

%%%%%%%%%%%%%%%%%%%%%%%%% FLDIAGNOSTICS %%%%%%%%%%%%%%%%%%%%%%%%%%

\subsubsection{fldiagnostics}
\label{sect:fldiagnostics}
fldiagnostics is an offline diagnostic tool. It is run from the command line:

\begin{lstlisting}[language = Bash]
fldiagnostics ACTION [OPTIONS] INPUT OUTPUT [FIRST] [LAST]
\end{lstlisting}

If \lstinline[language = Bash]+FIRST+ is supplied, treats \lstinline[language = Bash]+INPUT+ and \lstinline[language = Bash]+OUTPUT+ as project names, and processes the specified range of project files. The options are:
\begin{lstlisting}[language = Bash]
add[-diag]  Add a diagnostics field. Options:
              -m NAME   Field from which to extract a mesh to use with the
                        diagnostic field (default "Velocity")
              -o NAME   Diagnostic field name
              -r RANK   Specify the rank of the diagnostic field (will try all
                        ranks if not supplied, but will also suppress useful
                        error messages)
              -s STATE  Name of the state from which to read options in the 
                        options file (defaults to the first state)
              -x FLML   Model options file (not always required)
\end{lstlisting}

and the \lstinline[language = Bash]+-h+ flag displays the help message. To add the diagnostic field GridReynoldsNumber to a set of vtus run:

\begin{lstlisting}[language = Bash]
fldiagnostics add -o GridReynoldsNumber -r 0 ...
			           ... lock_exchange.vtu lock_exchange_out 5 8
\end{lstlisting}


%%%%%%%%%%%%%%%%%% PETSC READNSOLVE %%%%%%%%%%%%%%%%%%%%%%%%%%%%%%

\subsubsection{petsc\_readnsolve}
\label{sect:petsc_readnsolve} 
Whenever in \fluidity\ a linear solve has not completed succesfully the equation is dumped out in a file called matrixdump. This file can be used to analyse the behaviour of this solve without having to rerun the model with petsc\_readnsolve. It reads in the matrixdump and tries to solve it with PETSc options set in the flml file. It is advisable to first reproduce the failing solve with the same options as it happened in \fluidity\. After that the solver options in the flml file can be changed to see if that fixes the problem. The options under \option{\ldots/solver/diagnostics} are particularly useful to diagnose the problem.  petsc\_readnsolve is run from the command line:

\begin{lstlisting}[language = Bash]
petsc_readnsolve FILENAME FIELDNAME [OPTIONS]
\end{lstlisting}
where \lstinline[language = Bash]+FILENAME+ is the relevant flml file and \lstinline[language = Bash]+FIELDNAME+ is the field for which the solve was failing. The \lstinline[language = Bash]+OPTIONS+ available are:
\begin{lstlisting}[language = Bash]
 -l 	 			Write the information that is normally
				printed  in the terminal to a log file 
				called petsc_readnsolve.log-0.
-prns_verbosity N 	 	Determines the amount of information 
				that is printed to the terminal. By default 
				petsc_readnsolve uses the maximum verbosity 
				(3), this can be lowered with this option. 
-prns_filename file		reads from the specified file instead of 
				'matrixdump'
-prns_zero_init_guess 		no initial guess is read from matrixdump 
				instead the initial guess is zeroed
-prns_write_solution file 	writes solution vector to specified file 
				so that it can be used for comparison in 
				next runs of petsc_readnsolve (provided 
				we are sufficiently confident in the 
				accuracy of the obtained solution).
-prns_read_solution file 	reads solution vector from the specified file,
				so that exact errors can be calculated. For 
				small matrices a good approximation of the  
				exact solution can be found using a direct 
				method: select iterative_method "preonly" 
				and preconditioner "lu" in the .flml. Note 
				however that for ill-conditioned matrices 
				direct methods are very sensitive to round 
				off errors
-prns_scipy 			writes out several files that can be read in scipy.
-prns_random_rhs 	 	Instead of the rhs in the matrixdump, use a 
				random rhs vector.
\end{lstlisting} 

Additionally all options that are available from the PETSc library may be added to the command line. Options specified in the flml file always take precedence however. Some PETSc useful options:

\begin{lstlisting}[language = Bash]
-ksp_view 	Information on all the solver settings.
-mat_view_info 	Information on matrix size
-mat_view_draw 	Draw the matrix nonzero structure
-help 		Gives an overview of all PETSc options that can be given for 
		the selected solver/preconditioner combination. 
\end{lstlisting} 

{\bf Parallel}


When a solve fails in a parallel run, a single matrixdump file is written. petsc\_readnsolve can be run in serial on this matrixdump but owing to the usual large size of a parallel run and that the behaviour of a solver in parallel is often different than in serial, it is generally better to run petsc\_readnsolve in parallel as well. This is done by prepending \lstinline[language = Bash]+mpirun -np N+ on the command line (where \lstinline[language = Bash]+N+ is the number of processes).

petsc\_readnsolve in parallel requires the mesh files of the same mesh as the one used by \fluidity\ during the failing solve. Therefore, for adaptive runs, a checkpoint at the point of the failing solve is required and then the checkpoint flml is used for petsc\_readnsolve. In most cases the mesh files are not needed for serial runs of petsc\_readnsolve, even if the \fluidity\ run was parallel.

%%%%%%%%%%%%%%%%%% PROJECT TO CONTINUOUS %%%%%%%%%%%%%%%%%%%%%%%%%

\subsubsection{project\_to\_continuous}
\label{sect:project_to_continuous}
project\_to\_continuous, given a vtu file containing fields on a discontinuous mesh and triangle files for the corresponding continuous mesh, will produce a vtu with its fields projected onto the continuous mesh. It is run from the command line:

\begin{lstlisting}[language = Bash]
project_to_continuous [OPTIONS] vtufile trianglename
\end{lstlisting}

where \lstinline[language = Bash]+vtufile+ is the name of the discontinuous vtu and \lstinline[language = Bash]+trianglename+ is the base name of the triangle files. The flag \lstinline[language = Bash]+-h+ prints out the help message and the flag \lstinline[language = Bash]+-v+ enables verbose output.

%%%%%%%%%%%%%%%%%% STREAMFUNCTION 2D %%%%%%%%%%%%%%%%%%%%%%%%%%%%%

\subsubsection{streamfunction\_2d}
\label{sect:streamfunction_2d}

streamfunction\_2d solves the Poisson equation for the 2D streamfunction $\Psi$:

\begin{equation}
\nabla^2 \Psi = \ppx{u_y} - \ppx[y]{u_x},
\end{equation}

using a continuous Galerkin formulation. It applies the strong Dirichlet
boundary condition of:

\begin{equation}
\Psi = 0 \text{ on } \partial \Omega
\end{equation}

for all surfaces, and hence is only suitable for problems with a no normal flow boundary
condition on all surfaces.

streamfunction\_2d is built as part of the fltools build target (see section \ref{sect:fltools}),
and is used via:

\begin{lstlisting}[language = Bash]
streamfunction_2d [-v] input output
\end{lstlisting}

where \lstinline[language = Bash]+input+ is a vtu file containing a continuous vector
field ``Velocity'', and \lstinline[language = Bash]+output+ is an output vtu filename. The
\lstinline[language = Bash]+-v+ flag enables verbose output.

streamfunction\_2d can only be used in serial.

%%%%%%%%%%%%%%%%%% TEST PRESSURE SOLVE %%%%%%%%%%%%%%%%%%%%%%%%%%%

\subsubsection{test\_pressure\_solve}
\label{sect:test_pressure_solve}
test\_pressure\_solve designed to test the pressure solvers on large aspect ratio domains. SOLVE MINUS LAPLACE EQUATION -- i.e. geometric Laplacian
  !! This is to make a positive definite matrix (instead of negative)

%%%%%%%%%%%%%%%%%% UNIFIED MESH %%%%%%%%%%%%%%%%%%%%%%%%%%%%%%%%%%

\subsubsection{unifiedmesh}
\label{sect:unifiedmesh}
unifiedmesh dumps the supermesh constructed from two input meshes to discontinuous triangle mesh files and a vtu file. It is run from the command line:
\begin{lstlisting}[language = Bash]
unifiedmesh <triangle-file-1> <triangle-file-2> <output-file-name>
\end{lstlisting} 
No file names should have extenstions and both the triangle mesh files and the vtu file will have the name \lstinline[language = Bash]+output-file-name+

%%%%%%%%%%%%%%%%%% VERTICAL INTEGRATION %%%%%%%%%%%%%%%%%%%%%%%%%%

\subsubsection{vertical\_integration}
\label{sect:vertical_integration}

The vertical\_integration tool computes the Galerkin projection of the vertical
integral of some field onto a specified target surface mesh, via supermeshing (see section \ref{sec:supermeshing})
of the source mesh with a vertical extrusion of the target mesh. This can be used to
compute vertically integrated quantities for arbitrary unstructured meshes (with
no columnar vertical structure).

vertical\_integration is built as part of the fltools build target (see section \ref{sect:fltools}),
and is used via:

\begin{lstlisting}[language = Bash]
vertical_integration -b bottom -t top -s sizing [-p degree -v]
  target source field output
\end{lstlisting}

where \lstinline[language = Bash]+target+ is the target surface mesh triangle base name,
\lstinline[language = Bash]+source+ is the source vtu, \lstinline[language = Bash]+field+
is the field in \lstinline[language = Bash]+source+ to integrate, and \lstinline[language = Bash]+output+
is an output vtu filename. Components of vector or tensor fields may be integrated
by specifying a \% delimited ``field\%component'' value for \lstinline[language = Bash]+field+.

The compulsory flags \lstinline[language = Bash]+-b bottom+
and \lstinline[language = Bash]+-t top+ define the lower and upper bounds of
the vertical integral to be \lstinline[language = Bash]+bottom+ and \lstinline[language = Bash]+top+
respectively, and \lstinline[language = Bash]+-s sizing+ sets the thickness of layers
used in the computation of the vertical integral (the thickness of the layers in the
vertical extrusion of the target mesh through the source mesh). The optional
flag \lstinline[language = Bash]+-p degree+ sets the polynomial degree of the
output integral. By default the output field is \Pzero (discontinuous).
If \lstinline[language = Bash]+degree+$> 0$ is specified, then the output
field is \ensuremath{P_{degree}} (continuous). The
\lstinline[language = Bash]+-v+ flag enables verbose output.

vertical\_integration can only be used in serial.

%%%%%%%%%%%%%%%%%%%%%% VTKDIAGNOSTIC %%%%%%%%%%%%%%%%%%%%%%%%%%%%%

\subsubsection{vtkdiagnostic}
\label{sect:vtkdiagnostic}
vtkdiagnostic runs diagnostics on a given vtu file. It is run from the command line:
\begin{lstlisting}[language = Bash]
./vtkdiagnostic -i example.vtu [OPTIONS]
\end{lstlisting}

The \lstinline[language = Bash]+OPTIONS+ are: \\
General options:
\begin{lstlisting}[language = Bash]
-i, --input=FILENAME
	VTU file to use as input
-p, --node-positions
	Print out node XYZ positions
-b, --buff-body-volume
	Volume of buff body
-c, --clip <scalar name>/<scalar value>/<orientation>

-e, --element-volumes
	Print out element volumes
--vtk-arrays=array1[,array2,...,arrayN]
	Print out contents of specified VTK arrays
	e.g. --vtk-arrays=Velocity,Temperature
-g, --debug
	Print debugging information
-o, --offset <scalar name>/<offset value>
-r, --radial-scaling=scale_factor
	 Assume shperical earth geometry and scale the radius
-h, --help
	Print this usage information
\end{lstlisting}
Vorticity integral diagnostic options:
\begin{lstlisting}[language = Bash]
-v, --vorticity-integral
	Perform vorticity integral diagnostic
-2, --2d
	Force treatment as a 2d problem
-d, --dump-vtu=FILENAME
	Dumps a VTU file containing velocity and vorticity to FILENAME
-w, --debug-vorticity
	Imposes artificial (sinusoidal) velocity field
	and dumps debugging mesh to vorticityDebugMesh.vtu
\end{lstlisting}

%%%%%%%%%%%%%%%%%%%%%%%%% VTU_BINS %%%%%%%%%%%%%%%%%%%%%%%%%%%%%%%

\subsubsection{vtu\_bins}
\label{sect:vtu_bins}

vtu\_bins returns the fraction of the domain contained within given values
for a given scalar field. vtu\_bins is built as part of the fltools build target
(see section \ref{sect:fltools}), and is used via:

\begin{lstlisting}[language = Bash]
vtu_bins [-v] input_filename input_fieldname BOUND1 [BOUND2 BOUND3]
\end{lstlisting}

where \lstinline[language = Bash]+input_filename+ is a vtu file containing a scalar
field \lstinline[language = Bash]+input_fieldname+, and
\lstinline[language = Bash]+BOUND1+, \lstinline[language = Bash]+BOUND2+,
\lstinline[language = Bash]+...+ are the boundary values. The output is sent to
standard output.

If negative boundary values are required, add two ``-'' before the boundary
values on the command line:

\begin{lstlisting}[language = Bash]
vtu_bins [-v] input_filename input_fieldname -- BOUND1 [BOUND2 BOUND3]
\end{lstlisting}

vtu\_bins requires a linear tetrahedron mesh for field \lstinline[language = Bash]+input_fieldname+.

\begin{example}
  \begin{lstlisting}[language = Bash]
$ vtu_bins annulus_1564.vtu Temperature 0.0 1.0
                      -inf -   0.00000000000000000E+000 :  0.23071069007104538E-004
  0.00000000000000000E+000 -   0.10000000000000000E+001 :  0.99951353813554100E+000
  0.10000000000000000E+001 -                        inf :  0.46339079545184387E-003
  \end{lstlisting}
\caption{Using vtu\_bins to compute the volume of under- and over-shoot errors in
         a DG annulus simulation.}
\end{example}

%%%%%%%%%%%%%%%%%%%%%% VTKPROJECTION %%%%%%%%%%%%%%%%%%%%%%%%%%%%%

\subsubsection{vtk\_projection}
\label{sect:vtkprojection}
vtk\_projection projects the co-ordinates of an input vtu to a specified type of output co-ordinates. It is run from the command line:
\begin{lstlisting}[language = Bash]
vtk_projection[OPTIONS] -I <in-coordinates> -O <out-coordinates> infile.vtu
\end{lstlisting}

The \lstinline[language = Bash]+OPTIONS+ are:
\begin{lstlisting}[language = Bash]
-h, --help
	Print this usage information

-I, --incoords
	Coordinates of input file. Valid types are:
	 type      | description 
	-------------------------
	 cart      | Cartesian (meters)e
	 spherical | Longitude/Latitude
	 stereo    | Stereographic Projection

-O, --outcoords
	Coordinates of output file. Valid types are:
	 type      | description 
	-------------------------
	 cart      | Cartesian (meters)
	 spherical | Longitude/Latitude
	 stereo    | Stereographic Projection

-o, --output=FILENAME.vtu
	File that the result is outputed to. 
	**The default is to overwrite the input file**

-v, --verbose
	Be verbose

\end{lstlisting}

%%%%%%%%%%%%%%%%%%%%%%%%%%%%%%%%%%%%%%%%%%%%%%%%%%%%%%%%%%%%%%%%%%
%%%%%%%%%%%%%%%%%%%%%% USEFUL SCRIPTS %%%%%%%%%%%%%%%%%%%%%%%%%%%%
%%%%%%%%%%%%%%%%%%%%%%%%%%%%%%%%%%%%%%%%%%%%%%%%%%%%%%%%%%%%%%%%%%

\subsection{Useful scripts}
\label{sect:diagnostics_useful_scripts}

There are many useful scripts in the \fluidity\ trunk that can be found in the \lstinline[language = bash]+scripts/+ directory. Table \ref{tab:useful_diagnostic_scripts} lists them and descriptions can be found by referring to the relevant section.

\begin{table}
\begin{center}
  \begin{tabular}{| l | l |}
    \hline
	Script 					& Section 			\\
    \hline		
    	create\_aligned\_mesh 			& \ref{sect:create_aligned_mesh}\\
	edge\_length\_distribution.py		& \ref{sect:edgelengthdist} 	\\
	genpvtu					& \ref{sect:genpvtu}		\\
        gen\_square\_meshes			& \ref{sect:gen_square_meshes} 	\\		
    	interval				& \ref{sect:interval} 		\\
	linear\_interpolation.py		& \ref{sect:scripts_linear_interpolation} \\	
	mean\_flow				& \ref{sect:mean_flow}		\\
	mms\_tracer\_error.py			& \ref{sect:mms_tracer_error}	\\
	nodecount.py				& \ref{sect:nodecount}	\\
	pod.py					& \ref{sect:scripts_pod}	\\
	pvtu2vtu				& \ref{sect:pvtu2vtu}		\\
	rename\_checkpoint.py			& \ref{sect:rename_checkpoint}	\\
	stat2csv				& \ref{sect:stat2csv}		\\
    	statplot				& \ref{sect:statplot} 		\\
	transform\_mesh				& \ref{sect:transform_mesh}	\\
	vtudecomp				& \ref{sect:vtudecomp}		\\
	vtudiff					& \ref{sect:vtudiff}		\\
    \hline
  \end{tabular}
\end{center}
\caption[Table of useful diagnostic scripts]{Table of useful diagnostic scripts. They can be found in the \lstinline[language = bash]+scripts/+ directory of the \fluidity\ trunk.}
\label{tab:useful_diagnostic_scripts}
\end{table}

%%%%%%%%%%%%%%%%%% BAROTROPIC STREAMFUNCTION %%%%%%%%%%%%%%%%%%%%%%

\subsubsection{barotropic\_streamfunction}
\label{sect:barotropic_streamfunction}

barotropic\_streamfunction is a wrapper script for the streamfunction\_2d tool
(see section \ref{sect:streamfunction_2d}). It takes, as input, two vtu files containing
the x and y component of a velocity field as scalar fields, combines them to
form a vector field, and calls streamfunction\_2d to compute the streamfunction.
It is generally used in conjunction with
vertical\_integration (see section \ref{sect:vertical_integration}).

barotropic\_streamfunction is used via:

\begin{lstlisting}[language = Bash]
scripts/barotropic_streamfunction [-v] ucomp vcomp output
\end{lstlisting}

where \lstinline[language = Bash]+ucomp+ and \lstinline[language = Bash]+vcomp+
are vtu files containing the x and y component velocity fields as scalar fields
named ``Velocity\%1'' and ``Velocity\%2'' respectively, and
\lstinline[language = Bash]+output+ is an output vtu filename. The
\lstinline[language = Bash]+-v+ flag enables verbose output.

Note that if the velocity component fields are \Pzero, they are remapped
to a \Pone field via nodal averaging (using the vtkCellDataToPointData vtk filter -
see \url{http://www.vtk.org/doc/release/5.4/html/a00246.html})
before being supplied to streamfunction\_2d.

%%%%%%%%%%%%%%%%%% CREATE ALIGNED MESH %%%%%%%%%%%%%%%%%%%%%%%%%%%%

\subsubsection{create\_aligned\_mesh}
\label{sect:create_aligned_mesh}

Creates the triangle files for a mesh that lines up in all directions, so that it can be made it into a singly, doubly or triply periodic mesh. It is run from the command line:

\begin{lstlisting}[language = Bash]
scripts/create_aligned_mesh newmesh Lx Ly Lz Nx Ny Nz [Ox Oy Oz]
\end{lstlisting}

This creates a box that is $\mathrm{Lx} \times \mathrm{Ly} \times \mathrm{Lz}$ with Nx, Ny, Nz layers in the $x,y,z$ directions respectively. The optional arguments Ox, Oy, Oz  specify the position of the origin. Information can be found by running \lstinline[language = bash]+create_aligned_mesh --help+ on the command line. Note the mesh will always be 3d.

%%%%%%%%%%%%%%%%%% EDGE LENGTH DISTRIBUTION %%%%%%%%%%%%%%%%%%%%%%%

\subsubsection{edge\_length\_distribution.py}
\label{sect:edgelengthdist}

edge\_length\_distribution.py is a python script that creates histograms of the edge lengths of the elements in the mesh at each time step. It also produces line graphs of the maximum and minimum edge lengths as a function of time. This is useful for analysing adaptive runs to see how the edge lengths compare to a fixed run of the same process.  It is run from the command line:
\begin{lstlisting}[language = Bash]
scripts/edge_length_distribution.py <vtu filename> 
[options]
\end{lstlisting}
The options are available inside the python file or at the command line.

%%%%%%%%%%%%%%%%%%%%%%%%%% GENPVTU %%%%%%%%%%%%%%%%%%%%%%%%%%%%%%%%

\subsubsection{genpvtu}
\label{sect:genpvtu}
genpvtu creates a set of pvtus from a base set of vtus. It is run from the command line:
\begin{lstlisting}[language = Bash]
scripts/genpvtu basename 
\end{lstlisting}
with basename that of the vtu set e.g. $\{$example\_0\_0.vtu, example\_0\_1.vtu, example\_0\_2.vtu; example\_1\_0.vtu, example\_1\_1.vtu, example\_1\_2.vtu$\}$ has basename 'example'. It will produce pvtus with names example\_0.pvtu, example\_1.pvtu.

%%%%%%%%%%%%%%%%%%%%% GEN SQUARE MESHES %%%%%%%%%%%%%%%%%%%%%%%%%%%

\subsubsection{gen\_square\_meshes}
\label{sect:gen_square_meshes}
gen\_square\_meshes will generate triangle files for 2D square meshes. It is run with python from the command line:
\begin{lstlisting}[language = Bash]
python scripts/gen_square_meshes [OPTIONS] NODES MESHES 
\end{lstlisting} 
NODES is the number of nodes in the mesh and MESHES is the number of meshes to be created. [OPTIONS] are -h for help and -v for Verbose mode.

%%%%%%%%%%%%%%%%%%%%%%%%%% INTERVAL %%%%%%%%%%%%%%%%%%%%%%%%%%%%%%%

\subsubsection{interval}
\label{sect:interval}

This is a one-dimensional mesh generator. It is run from the command line:
\begin{lstlisting}[language = Bash]
scripts/interval [options] left right name 
\end{lstlisting}
Left and right are the start and end points for the mesh, and name is the name of the file.

%%%%%%%%%%%%%%%%%%%%%%%%%% LINEAR INTERPOLATION %%%%%%%%%%%%%%%%%%%

\subsubsection{linear\_interpolation.py}
\label{sect:scripts_linear_interpolation}
linear\_interpolation.py linearly interpolates all the fields of a set of vtus on to the mesh of a target vtu. It is run with python from the command line:
\begin{lstlisting}[language = Bash]
python scripts/linear_interpolation.py TARGET VTU1 VTU2 VTU3 
\end{lstlisting}
TARGET is the name of the mesh that the fields will be interpolated on to. VTU1, VTU2, VTU3 are the vtus from which the fields will be interpolated (the number of vtus can be 1 or more, 3 are used here for illustration). An output vtu called \lstinline[language = Bash]+interpolation_output.vtu+ will be generated that will contain all the interpolated fields from all the vtus. 

%%%%%%%%%%%%%%%%%%%%%%%%%% MEAN FLOW %%%%%%%%%%%%%%%%%%%%%%%%%%%%%%
 
\subsubsection{mean\_flow}
\label{sect:mean_flow}

mean\_flow calculates the mean of the fields in a set of vtus. It is run with python from the command line:
\begin{lstlisting}[language = Bash]
python scripts/mean_flow <options> [vtu basename] [first dump id] [last dump id]
\end{lstlisting}
The options include specification of an area of the domain to sample and the number of sampling planes in each direction. The option \lstinline[language = Bash]+-h+ provides further information on how to use these.

%%%%%%%%%%%%%%%%%%%%%%%%%% MMS TRACER ERROR %%%%%%%%%%%%%%%%%%%%%%%

\subsubsection{mms\_tracer\_error.py}
\label{sect:mms_tracer_error}

mms\_tracer\_error.py evaluates the error between two fields using either an $L_2$ or $L_\infty$ control-volume norm. It is called from a python script e.g.:
\begin{lstlisting}[language = python]
import mms\_tracer\_error as error
l2_error = error.l2("MMS.vtu", "field1", "field2")
inf_error = error.inf("MMS.vtu", "field1", "field2")
\end{lstlisting}
where \lstinline[language = python]+MMS.vtu+ is the name of the vtu and \lstinline[language = python]+field1, field2+ are two scalar fields in the vtu to be compared.

%%%%%%%%%%%%%%%%%%%%%%%%%% NO NODES %%%%%%%%%%%%%%%%%%%%%%%%%%%%%%%

\subsection{nodecount.py}
\label{sect:nodecount}
nodecount.py will return the number of nodes in a list of vtus. It is run from the command line:
\begin{lstlisting}[language = Bash]
scripts/nodecount.py [vtulist]
\end{lstlisting}
If \lstinline[language = Bash]+vtulist+ is not supplied it will run on any vtu in the working directory.

%%%%%%%%%%%%%%%%%%%%%%%%%%%%% POD %%%%%%%%%%%%%%%%%%%%%%%%%%%%%%%%%

\subsubsection{pod.py}
\label{sect:scripts_pod}
pod.py computes the Principal Orthogonal Decomposition of a series of snapshots. It is run with python from the command line:
\begin{lstlisting}[language = Bash]
python scripts/pod.py VTU FIELD
\end{lstlisting}
where \lstinline[language = Bash]+VTU+ is the name of a vtu file with fields like Temperature1, Temperature2, Temperature3, etc to be decomposed and \lstinline[language = Bash]+FIELD+ is the name of the field to decompose (here Temperature). 

%%%%%%%%%%%%%%%%%%%%%%%%%% PVTU2VTU %%%%%%%%%%%%%%%%%%%%%%%%%%%%%%%

\subsubsection{pvtu2vtu}
\label{sect:pvtu2vtu}
pvtu2vtu combines pvtus into vtus. It is run from the command line:
\begin{lstlisting}[language = Bash]
python scripts/pvtu2vtu [OPTIONS] PROJECT FIRSTID [LASTID]
\end{lstlisting} 
with \lstinline[language = Bash]+PROJECT+ the basename of the pvtus and \lstinline[language = Bash]+FIRSTID+ and \lstinline[language = Bash]+LASTID+ the first and last id numbers respectively of the pvtus to be included. \lstinline[language = Bash]+LASTID+ is optional and defaults to \lstinline[language = Bash]+FIRSTID+. Running with the option \lstinline[language = Bash]+-h+ will give further information on the other options available. Note, this may not always work with vtus from adaptive runs.

%%%%%%%%%%%%%%%%%%%%%% RENAME CHECKPOINT %%%%%%%%%%%%%%%%%%%%%%%%%%

\subsubsection{rename\_checkpoint.py}
\label{sect:rename_checkpoint}
rename\_checkpoint.py takes a list of vtu files in the working directory produced from a serial checkpointed flml file with names base\_filename\_checkpoint\_i.vtu for all i and renames them as base\_filename\_index+i.vtu. Additionally it may take a list of
vtu and pvtu files in the current directory produced from a checkpointed parallel flml file with names base\_filename\_checkpoint\_i\_j.vtu and base\_filename\_checkpoint\_i.pvtu for all i (index) and j (processor number) and renames them as base\_filename\_index+i\_j.vtu and base\_filename\_index+i.pvtu. It is run from the command line with:
\begin{lstlisting}[language = Bash]
scripts/rename_checkpoint.py [options] <base_filename> <index>
\end{lstlisting}


%%%%%%%%%%%%%%%%%%%%%%%%%% STAT2CSV %%%%%%%%%%%%%%%%%%%%%%%%%%%%%%%%

\subsubsection{stat2csv}
\label{sect:stat2csv}
stat2csv converts a \fluidity\ stat file into a csv file. It is run from the command line:
\begin{lstlisting}[language = Bash]
scripts/stat2csv [OPTIONS] PROJECT
\end{lstlisting} 
with \lstinline[language = Bash]+PROJECT+ the base name of the stat file. The default output is to PROJECT.csv. Running with the option \lstinline[language = Bash]+-h+ will provide further information on the output file name, type and format.

%%%%%%%%%%%%%%%%%%%%%%%%%% STATPLOT %%%%%%%%%%%%%%%%%%%%%%%%%%%%%%%

\subsubsection{statplot}
\label{sect:statplot}

statplot is a graphical program for previewing files in the .stat file format.
statplot can be launched via:

\begin{lstlisting}[language = Bash]
scripts/statplot filename
\end{lstlisting}

This generates a graphical user interface displaying a plot of one statistic in
the .stat file against another. The ordinate and abscissa statistics can be
selected via combo boxes immediately beneath the plot. The plot itself can
be navigated using the pylab navigation toolbar - see
\url{http://matplotlib.sourceforge.net/users/navigation_toolbar.html} for
more complete documentation.

Additional keyboard commands:

\begin{center}
  \begin{tabular}{| l | l |}
    \hline
    Key & Function \\
    \hline
    s   & Switch to scatter plot mode \\
    l   & Switch to line plot mode \\
    r   & Re-load the input file \\
    R   & Re-load the input file without changing the plot bounds \\
    q   & Quit \\
    x   & Toggle x-axis linear / log \\
    y   & Toggle y-axis linear / log \\
    \hline
  \end{tabular}
\end{center}

\begin{figure}[ht]
  \centering
  \fig[width=0.7\textwidth]{visualisation_and_diagnostics_images/Statplot}
  \caption{Visualisation of a heat flux diagnostic in a 2D cavity convection
           simulation using statplot.}
  \label{fig:statplot}
\end{figure}

%%%%%%%%%%%%%%%%%%%%%%% TRANSFORM MESH %%%%%%%%%%%%%%%%%%%%%%%%%%%%

\subsubsection{transform\_mesh}
\label{sect:transform_mesh}

transform\_mesh applies a given coordinate transformation to the given mesh. It is run from the command line:
\begin{lstlisting}[language = Bash]
scripts/transform_mesh <transformation> <mesh>
\end{lstlisting}
with \lstinline[language = Bash]+<mesh>+ the base name of the triangle mesh files. A mesh.node, mesh.face and mesh.ele file are required. \lstinline[language = Bash]+<transformation>+ is a python expression giving the coordinate transformation. For example to rescale the z-dimension by a factor of 1000 run:
\begin{lstlisting}[language = Bash]
scripts/transform_mesh '(x,y,1000*z)' mesh
\end{lstlisting}
It can be used for 2D or 3D meshes.

%%%%%%%%%%%%%%%%%%%%%%%%%  VTUDECOMP %%%%%%%%%%%%%%%%%%%%%%%%%%%%%%

\subsubsection{vtudecomp}
\label{sect:vtudecomp}

vtudecomp decomposes a vtu given a decomposed triangle mesh. It is run from the command line:
\begin{lstlisting}[language = Bash]
scripts/transform_mesh [OPTIONS] MESH VTU
\end{lstlisting}
with \lstinline[language = Bash]+MESH+ the base name of the decomposed triangle mesh and \lstinline[language = Bash]+VTU+ the name of the vtu file to be decomposed. Running with the option \lstinline[language = Bash]+-h+ provides further information on the other options. Note, genpvtu, section \ref{sect:genpvtu} can be used to create a pvtu from the decomposed vtus.

%%%%%%%%%%%%%%%%%%%%%%%%%  VTUDIFF %%%%%%%%%%%%%%%%%%%%%%%%%%%%%%%%

\subsubsection{vtudiff}
\label{sect:vtudiff}

vtudiff generates vtus with fields equal to the difference between the corresponding fields in two input vtus (\lstinline[language = Bash]+INPUT1+ - \lstinline[language = Bash]+INPUT2+). The fields of \lstinline[language = Bash]+INPUT2+ are projected onto the cell points of \lstinline[language = Bash]+INPUT1+. It is run from the command line:
\begin{lstlisting}[language = Bash]
scripts/transform_mesh [OPTIONS] INPUT1 INPUT2 OUTPUT [FIRST] [LAST]
\end{lstlisting}
with \lstinline[language = Bash]+OUTPUT+ the name of the output vtu. If \lstinline[language = Bash]+FIRST+ is supplied, treats INPUT1 and INPUT2 as project names, and generates a different vtu for the specified range of output files \lstinline[language = Bash]+FIRST+ - \lstinline[language = Bash]+LAST+. If not supplied \lstinline[language = Bash]+LAST+ defaults to \lstinline[language = Bash]+FIRST+. The option \lstinline[language = Bash]+-s+ if supplied together with \lstinline[language = Bash]+FIRST+ and \lstinline[language = Bash]+LAST+, only \lstinline[language = Bash]+INPUT1+ is treated as a project name. This allows a range of vtus to be diffed against a single vtu.



