\chapter{Embedded models}\label{chap:embedded}

The parameterisations described in chapter \ref{chap:parameterisations}\ are
used to model fluids processes which cannot be resolved by the model. In
contrast, the models described in this chapter detail models embedded in
\fluidity\ which model non-fluids processes.

\section{Biology}

The Biology model in \fluidity\ is a very simple four-equation model in
which the variables are phytoplankton, zooplankton, nutrient and detritus.
These four variables evolve under an advection-diffusion equation similar to
that for other tracers such as temperature and salinity but modified by the
addition of a source term which contains the interactions between the
biological fields.

Figure \ref{fig:biofluxes} shows the interactions between the four
biological tracers. Nutrients (plus sunlight) are converted into
phytoplankton. Zooplankton feed on phytoplankton and detritus but in doing
so produce some nutrients and detritus so grazing also results in fluxes
from phytoplankton and detritus to nutrient and detritus. Both phytoplankton
and zooplanton die producing detritus and detritus is gradually converted to
nutrient by a remineralisation process.
\begin{figure}[hb]
  \centering
  \onlypdf{\begin{pdfdisplay}}
    \begin{psmatrix}[colsep=4,rowsep=4]
      \psframebox[framearc=.2]{\
        \begin{minipage}[c][7ex]{0.3\linewidth}
          \centering
          Phytoplankton
        \end{minipage}} &
      \psframebox[framearc=.2]{
        \begin{minipage}[c][7ex]{0.3\linewidth}
          \centering
          Zooplankton
        \end{minipage}}\\
      \psframebox[framearc=.2]{
        \begin{minipage}[c][7ex]{0.3\linewidth}
          \centering
          Nutrient
        \end{minipage}}&
      \psframebox[framearc=.2]{
        \begin{minipage}[c][7ex]{0.3\linewidth}
          \centering
          Detritus
        \end{minipage}}\\
      \ncarc[angleB=270]{->}{1,1}{1,2}
      \naput{grazing}
      \ncarc[angleA=0,angleB=180]{->}{2,1}{1,1}
      \naput{fertilisation}
      \ncarc[angleA=180,angleB=0]{->}{1,1}{2,1}
      \naput{grazing}
      \ncarc[angleA=270]{->}{2,2}{2,1}
      \naput{remineralisation}
      \ncarc[angleA=270]{<-}{2,1}{2,2}
      \naput{grazing}
      \ncarc[angleA=180,angleB=0]{->}{1,2}{2,2}
      \naput{death}
      \ncarc[angleA=0,angleB=180]{->}{2,2}{1,2}
      \naput{grazing}
      \ncarc[angleA=315,angleB=135]{<-}{2,2}{1,1}
      \naput{death}
      \ncarc[angleA=135,angleB=315]{->}{1,1}{2,2}
      \naput{grazing}
    \end{psmatrix}
  \onlypdf{\end{pdfdisplay}}
  \caption{The fluxes between the biological tracers. Grazing refers to the
    feeding activity of zooplankton on phytoplankton and detritus.}
  \label{fig:biofluxes}
\end{figure}

\subsection{Biological source terms}
The source terms for phytoplankton ($P$), zooplankton ($Z$), nutrient ($N$)
and detritus ($D$) respectively are given by the following expressions:
\begin{gather}
  S_P=\mathrm{R}_P -  \mathrm{G}_P -\mathrm{De}_P\\
  S_Z=\gamma\beta(\mathrm{G}_P+\mathrm{G}_D) - \mathrm{De}_Z\\
  S_N=-\mathrm{R}_P + \mathrm{De}_D +
  (1-\gamma)\beta(\mathrm{G}_P+\mathrm{G}_D)\\
  S_D=-\mathrm{De}_D + \mathrm{De}_P + \mathrm{De}_Z + (1-\beta)\mathrm{G}_P
  -\beta\mathrm{G}_D)
\end{gather}
The definitions of each of these terms are given below. It is significant
that the right-hand sides of these equations sum to zero. This implies that,
for a closed biological system, the total of the biological constituents is
always conserved. The terms in these equations are given in table
\ref{tab:bioparameters}. The variable terms are explained in more detail below.

\begin{table}[ht]
  \centering
  \begin{tabular}{clll}
    \textbf{Symbol} & \textbf{Meaning} & \textbf{Typical value} & \textbf{Section}\\\hline
    $\mathrm{R}_P$ & phytoplankton growth rate & & \ref{sec:R_P}\\
    $\mathrm{G}_P$ & rate of zooplankton grazing on phytoplankton && \ref{sec:G_P}\\
    $\mathrm{De}_P$ & death rate of phytoplankton && \ref{sec:De_P}\\
    $\mathrm{G}_D$ & rate of zooplankton grazing on detritus && \ref{sec:G_D}\\
    $\mathrm{De}_Z$ & death rate of zooplankton && \ref{sec:De_Z}\\
    $\mathrm{De}_D$ & death rate of detritus && \ref{sec:De_D}\\
    $I$ & photosynthetically active radiation & & \ref{sec:I} \\
    $\alpha$ & sensitivity of phytoplankton to light & \unit[0.015]{m$^2$\,W$^{-1}$\,day$^{-1}$}  & \ref{sec:R_P}\\
    $\beta$ & assimilation efficiency of zooplankton & 0.75 \\
    $\gamma$ & zooplankton excretion parameter & 0.5 \\
    $\mu_P$ & phytoplankton mortality rate & \unit[0.1]{day$^{-1}$} & \ref{sec:De_P}\\
    $\mu_Z$ & zooplankton mortality rate & \unit[0.2]{day$^{-1}$} & \ref{sec:De_Z}\\
    $\mu_D$ & detritus remineralisation rate & \unit[0.05]{day$^{-1}$} & \ref{sec:De_D}\\
    $g$ & zooplankton maximum growth rate & \unit[1]{day$^{-1}$} & \ref{sec:G_P}\\
    $k_N$ & half-saturation constant for nutrient & 0.5  & \ref{sec:G_D}\\
    $k$ & zooplankton grazing parameter & 0.5 & \ref{sec:G_P}\\
    $p_P$ & zooplankton preference for phytoplankton & 0.75 & \ref{sec:G_P}\\
    $v$ & maximum phytoplankton growth rate & \unit[1.5]{day$^{-1}$} & \ref{sec:R_P}
  \end{tabular}
  \caption{Meanings of symbols in the biology model. Typical values are
    provided for externally set parameters.}
  \label{tab:bioparameters}
\end{table}

\subsubsection{$\mathrm{R}_P$, the phytoplankton growth rate}\label{sec:R_P}

$\mathrm{R}_P$ is the growth-rate of phytoplankton which is governed by the
current phytoplankton concentration, the available nutrients and the
available light:
\begin{equation}
  \mathrm{R}_P=J\,P\,Q
\end{equation}
$J$ is the light-limited phytoplankton growth rate which is in turn given
by:
\begin{equation}
  J=\frac{v\alpha I}{(v^2+\alpha^2 I^2)^{0.5}}
\end{equation}
In this expression, $v$ is the maximum phytoplankton growth rate, $\alpha$
controls the sensitivity of growth rate to light and $I$ is the available
photosynthetically active radiation.

$Q$ is the nutrient limiting factor and is given by:
\begin{equation}
  Q=\frac{N}{k_N+N}
\end{equation}
where $k_N$ is the half-saturation constant for nutrient.

\subsubsection{$\mathrm{G}_P$, the rate of phytoplankton grazing by zooplankton}\label{sec:G_P}
The rate at which zooplankton eat phytoplankton is given by:
\begin{equation}
  \mathrm{G}_P=\frac{g p_P P^2 Z}{k^2 + p_P P^2 + (1-p_P) D^2}
\end{equation}
in which $p_P$ is the preference of zooplankton for grazing phytoplankton
over grazing detritus, $g$ is the maximum zooplankton growth rate and $k$ is
a parameter which limits the grazing rate if the concentration of
phytoplankton and detritus becomes very low.

\subsubsection{$\mathrm{G}_D$, the rate of detritus grazing by zooplankton}\label{sec:G_D}
The rate at which zooplankton eat detritus is given by:
\begin{equation}
  \mathrm{G}_D=\frac{g (1-p_P) D^2 Z}{k^2 + p_P P^2 + (1-p_P) D^2}
\end{equation}
in which all of the parameters have the meaning given in the previous
section.

\subsubsection{$\mathrm{De}_P$, the phytoplankton death rate}\label{sec:De_P}

A proportion of the phytoplankton in the system will die in any given time
interval. The dead phytoplankton become detritus:
\begin{equation}
  \mathrm{De}_P=\frac{\mu_P P^2}{P+1}
\end{equation}
in which $\mu_P$ is the phytoplankton mortality rate.

\subsubsection{$\mathrm{De}_Z$, the zooplankton death rate}\label{sec:De_Z}

A proportion of the zooplankton in the system will die in any given time
interval. The dead zooplankton become detritus:
\begin{equation}
  \mathrm{De}_Z=\mu_Z Z^2
\end{equation}

\subsubsection{$\mathrm{De}_D$, the detritus remineralisation rate}\label{sec:De_D}

As the detritus falls through the water column, it gradually converts to
nutrient:
\begin{equation}
  \mathrm{De}_D=\mu_D D
\end{equation}

\subsection{Photosynthetically active radiation}\label{sec:I}

Phytoplankton depends on the levels of light in the water column at the
frequencies useful for photosynthesis. This is referred to as
photosynthetically active radiation. Sunlight falls on the water surface and
is absorbed by both water and phytoplankton. This is represented by the
following equation:
\begin{equation}
  \ppx[z]{I}=(A_{\mathrm{water}}+A_PP)I
\end{equation}
where $A_{\mathrm{water}}$ and $A_P$ are the absorption rates of
photosynthetically active radiation by water and phytoplankton respectively.

This equation has the form of a one-dimensional advection equation and
therefore requires a single boundary condition: the amount of light incident
on the water surface. This is a Dirichlet condition on $I$.

\subsection{Detritus falling velocity}\label{sec:detritus}

Phytoplankton, zooplankton and nutrient are assumed to be neutrally buoyant
and therefore move only under the advecting velocity of the water or by
diffusive mixing. Detritus, on the other hand, is assumed to be denser than
water so it will slowly sink through the water-column. This is modelled by
modifying the advecting velocity in the advection-diffusion equation for
detritus by subtracting a sinking velocity $u_{\mathrm{sink}}$ from the
vertical component of the advecting velocity.