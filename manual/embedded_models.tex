\chapter{Embedded models}\label{chap:embedded}

The parameterisations described in chapter \ref{chap:parameterisations}\ are
used to model fluids processes which cannot be resolved by the model. In
contrast, the models described in this chapter detail models embedded in
\fluidity\ which model non-fluids processes.

\section{Biology}
\label{sec:biology_model}
The Biology model in \fluidity\ contains a number of different submodels. All are
currently population level models where variables evolve under an 
advection-diffusion equation similar to that for other tracers, such as 
temperature and salinity, but modified by the
addition of a source term which contains the interactions between the
biological fields. The fluxes depend on which model is selected and which
tracers are available. 

There are two models currently distributed with \fluidity: a four-component model
and a six-component model.

\subsection{Four component model}

Figure \ref{fig:biofluxes} shows the interactions between the
biological tracers in the four component model. Nutrients (plus sunlight) are converted into
phytoplankton. Zooplankton feed on phytoplankton and detritus but in doing
so produce some nutrients and detritus so grazing also results in fluxes
from phytoplankton and detritus to nutrient and detritus. Both phytoplankton
and zooplankton die producing detritus and detritus is gradually converted to
nutrient by a remineralisation process.
\begin{figure}[hb]
  \centering
  \onlypdf{\begin{pdfdisplay}}
    \begin{psmatrix}[colsep=4,rowsep=4]
      \psframebox[framearc=.2]{\
        \begin{minipage}[c][7ex]{0.3\linewidth}
          \centering
          Phytoplankton
        \end{minipage}} &
      \psframebox[framearc=.2]{
        \begin{minipage}[c][7ex]{0.3\linewidth}
          \centering
          Zooplankton
        \end{minipage}}\\
      \psframebox[framearc=.2]{
        \begin{minipage}[c][7ex]{0.3\linewidth}
          \centering
          Nutrient
        \end{minipage}}&
      \psframebox[framearc=.2]{
        \begin{minipage}[c][7ex]{0.3\linewidth}
          \centering
          Detritus
        \end{minipage}}\\
      \ncarc[angleB=270]{->}{1,1}{1,2}
      \naput{grazing}
      \ncarc[angleA=0,angleB=180]{->}{2,1}{1,1}
      \naput{fertilisation}
      \ncarc[angleA=180,angleB=0]{->}{1,1}{2,1}
      \naput{grazing}
      \ncarc[angleA=270]{->}{2,2}{2,1}
      \naput{remineralisation}
      \ncarc[angleA=270]{<-}{2,1}{2,2}
      \naput{grazing}
      \ncarc[angleA=180,angleB=0]{->}{1,2}{2,2}
      \naput{death}
      \ncarc[angleA=0,angleB=180]{->}{2,2}{1,2}
      \naput{grazing}
      \ncarc[angleA=315,angleB=135]{<-}{2,2}{1,1}
      \naput{death}
      \ncarc[angleA=135,angleB=315]{->}{1,1}{2,2}
      \naput{grazing}
    \end{psmatrix}
  \onlypdf{\end{pdfdisplay}}
  \caption{The fluxes between the biological tracers. Grazing refers to the
    feeding activity of zooplankton on phytoplankton and detritus.}
  \label{fig:biofluxes}
\end{figure}

\subsubsection{Biological source terms}
The source terms for phytoplankton ($P$), zooplankton ($Z$), nutrient ($N$)
and detritus ($D$) respectively are given by the following expressions:
\begin{gather}
  S_P=\mathrm{R}_P -  \mathrm{G}_P -\mathrm{De}_P,\\
  S_Z=\gamma\beta(\mathrm{G}_P+\mathrm{G}_D) - \mathrm{De}_Z,\\
  S_N=-\mathrm{R}_P + \mathrm{De}_D +
  (1-\gamma)\beta(\mathrm{G}_P+\mathrm{G}_D),\\
  S_D=-\mathrm{De}_D + \mathrm{De}_P + \mathrm{De}_Z + (1-\beta)\mathrm{G}_P
  -\beta\mathrm{G}_D).
\end{gather}
The definitions of each of these terms are given below. It is significant
that the right-hand sides of these equations sum to zero. This implies that,
for a closed biological system, the total of the biological constituents is
always conserved. The terms in these equations are given in table
\ref{tab:bioparameters}. The variable terms are explained in more detail below.

\begin{table}[ht]
  \centering
  \begin{tabular}{llll}\hline
    \textbf{Symbol} & \textbf{Meaning} & \textbf{Typical value} & \textbf{Section}\\\hline
    $\mathrm{R}_P$ & phytoplankton growth rate & & \ref{sec:R_P}\\
    $\mathrm{G}_P$ & rate of zooplankton grazing on phytoplankton && \ref{sec:G_P}\\
    $\mathrm{De}_P$ & death rate of phytoplankton && \ref{sec:De_P}\\
    $\mathrm{G}_D$ & rate of zooplankton grazing on detritus && \ref{sec:G_D}\\
    $\mathrm{De}_Z$ & death rate of zooplankton && \ref{sec:De_Z}\\
    $\mathrm{De}_D$ & death rate of detritus && \ref{sec:De_D}\\
    $I$ & photosynthetically active radiation & & \ref{sec:I} \\
    $\alpha$ & sensitivity of phytoplankton to light & \unit[0.015]{m$^2$\,W$^{-1}$\,day$^{-1}$}  & \ref{sec:R_P}\\
    $\beta$ & assimilation efficiency of zooplankton & 0.75 \\
    $\gamma$ & zooplankton excretion parameter & 0.5 \\
    $\mu_P$ & phytoplankton mortality rate & \unit[0.1]{day$^{-1}$} & \ref{sec:De_P}\\
    $\mu_Z$ & zooplankton mortality rate & \unit[0.2]{day$^{-1}$} & \ref{sec:De_Z}\\
    $\mu_D$ & detritus remineralisation rate & \unit[0.05]{day$^{-1}$} & \ref{sec:De_D}\\
    $g$ & zooplankton maximum growth rate & \unit[1]{day$^{-1}$} & \ref{sec:G_P}\\
    $k_N$ & half-saturation constant for nutrient & 0.5  & \ref{sec:G_D}\\
    $k$ & zooplankton grazing parameter & 0.5 & \ref{sec:G_P}\\
    $p_P$ & zooplankton preference for phytoplankton & 0.75 & \ref{sec:G_P}\\
    $v$ & maximum phytoplankton growth rate & \unit[1.5]{day$^{-1}$} & \ref{sec:R_P}\\
    \hline
  \end{tabular}
  \caption{Meanings of symbols in the biology model. Typical values are
    provided for externally set parameters.}
  \label{tab:bioparameters}
\end{table}

\textbf{$\mathrm{R}_P$, the phytoplankton growth rate}\label{sec:R_P}

$\mathrm{R}_P$ is the growth-rate of phytoplankton which is governed by the
current phytoplankton concentration, the available nutrients and the
available light:
\begin{equation}
  \mathrm{R}_P=J\,P\,Q,
\end{equation}
where $J$ is the light-limited phytoplankton growth rate which is in turn given
by:
\begin{equation}
  J=\frac{v\alpha I}{(v^2+\alpha^2 I^2)^{1/2}}.
\end{equation}
In this expression, $v$ is the maximum phytoplankton growth rate, $\alpha$
controls the sensitivity of growth rate to light and $I$ is the available
photosynthetically active radiation.

$Q$ is the nutrient limiting factor and is given by:
\begin{equation}
  Q=\frac{N}{k_N+N},
\end{equation}
where $k_N$ is the half-saturation constant for nutrient.

\textbf{$\mathrm{G}_P$, the rate of phytoplankton grazing by zooplankton}\label{sec:G_P}
The rate at which zooplankton eat phytoplankton is given by:
\begin{equation}
  \mathrm{G}_P=\frac{g p_P P^2 Z}{k^2 + p_P P^2 + (1-p_P) D^2},
\end{equation}
in which $p_P$ is the preference of zooplankton for grazing phytoplankton
over grazing detritus, $g$ is the maximum zooplankton growth rate and $k$ is
a parameter which limits the grazing rate if the concentration of
phytoplankton and detritus becomes very low.

\textbf{$\mathrm{G}_D$, the rate of detritus grazing by zooplankton}\label{sec:G_D}
The rate at which zooplankton eat detritus is given by:
\begin{equation}
  \mathrm{G}_D=\frac{g (1-p_P) D^2 Z}{k^2 + p_P P^2 + (1-p_P) D^2},
\end{equation}
in which all of the parameters have the meaning given in the previous
section.

\textbf{$\mathrm{De}_P$, the phytoplankton death rate}\label{sec:De_P}

A proportion of the phytoplankton in the system will die in any given time
interval. The dead phytoplankton become detritus:
\begin{equation}
  \mathrm{De}_P=\frac{\mu_P P^2}{P+1},
\end{equation}
in which $\mu_P$ is the phytoplankton mortality rate.

\textbf{$\mathrm{De}_Z$, the zooplankton death rate}\label{sec:De_Z}

A proportion of the zooplankton in the system will die in any given time
interval. The dead zooplankton become detritus:
\begin{equation}
  \mathrm{De}_Z=\mu_Z Z^2.
\end{equation}

\textbf{$\mathrm{De}_D$, the detritus remineralisation rate}\label{sec:De_D}

As the detritus falls through the water column, it gradually converts to
nutrient:
\begin{equation}
  \mathrm{De}_D=\mu_D D.
\end{equation}

\subsection{Six-component model}
\label{sec:bio-6-component}

The six-component model is based on the model of \citet{popova2006}
which is designed to be applicably globally. 
Figure \ref{fig:biofluxes6} shows the interactions between the six
biological tracers. Nutrients, either ammonium or nitrate, (plus sunlight) are converted into
phytoplankton. Zooplankton feed on phytoplankton and detritus but in doing
so produce some nutrients and detritus so grazing also results in fluxes
from phytoplankton and detritus to nutrient and detritus. Both phytoplankton
and zooplankton die producing detritus and detritus is gradually converted to
nutrient by a re-mineralisation process. In addition, chlorophyll is present
as a subset of the phytoplankton.

\begin{figure}[ht]
  \centering
  \pdffig[width=0.7\textwidth]{embedded_models_images/bio_model_6}
  \caption{Six-component biology model.}
  \label{fig:biofluxes6}
\end{figure}


The source terms for phytoplankton ($P$), Chlorophyll-a ($\mathrm{Chl}$), zooplankton ($Z$), nitrate ($N$),
ammonium ($A$), and detritus ($D$) respectively are given by the following expressions:
\begin{gather}\label{eq:bio6_sources}
  S_P=PJ(\mathrm{Q}_N+\mathrm{Q}_A) - \mathrm{G}_P - \mathrm{De}_P,\\
  S_{\mathrm{Chl}}=(\mathrm{R}_P*J*(\mathrm{Q}_N+\mathrm{Q}_A)*P + 
      (-\mathrm{G}_P-\mathrm{De}_P))*\theta/\zeta),\\
  S_Z=\delta*(\beta_P*\mathrm{G}_P+\beta_D*\mathrm{G}_D) - \mathrm{De}_Z,\\
  S_N=-J*P*\mathrm{Q}_N+\mathrm{De}_A,\\
  S_A=-J*P*\mathrm{Q}_A + \mathrm{De}_D + (1 - \delta)*(\beta_P*\mathrm{G}_P + \beta_D*\mathrm{G}_D) + (1-\gamma)*\mathrm{De}_Z-\mathrm{De}_A,\\
  S_D=-\mathrm{De}_D + \mathrm{De}_P + \gamma*\mathrm{De}_Z +(1-\beta_P)*\mathrm{G}_P - \beta_D*\mathrm{G}_D
\end{gather}
The terms in these equations are given in table 

\ref{tab:bioparameters6}. The variable terms are explained in more detail below.
\subsubsection{Biological source terms}

\begin{table}[ht]
  \centering
  \begin{tabular}{lp{8cm}p{5cm}l}\hline
    \textbf{Symbol} & \textbf{Meaning} & \textbf{Typical value} & \textbf{Equation}\\\hline
    $\alpha$ & initial slope of $P-I$ curve,\unit[(W m${-2}$)${-1}$ day$^{-1}$] && \eqref{eq:bio6_alpha} \\
    $\alpha_c$ & Chl-a specific initial slope of $P-I$ curve & 2 \unit[(gCgChl$^{-1}$Wm$^{-2}$)${-1}$day$^{-1}$] & \\
    $\beta_P, \beta_D$ & assimilation coefficients of zooplankton & 0.75 & \\
    $\mathrm{De}_D$ & rate of breakdown of detritus to ammonium && \eqref{eq:bio6_dd}\\
    $\mathrm{De}_P$ & rate of phytoplankton natural mortality && \eqref{eq:bio6_dp}\\
    $\mathrm{De}_Z$ & rate of zooplankton natural mortality && \eqref{eq:bio6_dz} \\
    $\mathrm{De}_A$ & ammonium nitrification rate && \eqref{eq:bio6_da} \\    
    $\delta$ & excretion parameter & 0.7 & \\
    $\epsilon$ & grazing parameter relating capture of prey items to prey density & 0.4 & \\
    $\mathrm{G}_P$ & rate of zooplankton grazing on phytoplankton && \eqref{eq:bio6_gp}\\
    $\mathrm{G}_D$ & rate of zooplankton grazing on detritus && \eqref{eq:bio6_gd}\\
    $g$ & zooplankton maximum growth rate & 1.3 \unit[day$^{-1}$] & \\
    $\gamma$ & fraction of zooplankton mortality going to detritus & 0.5 & \\
    $I_0$ & photosynthetically active radiation (PAR) immediately below surface of water. Assumed to be 0.43 of the surface radiation &  \\
    $J$ & light-limited phytoplankton growth rate, \unit[day$^{-1}$] && \eqref{eq:bio6_j}\\
    $k_A$ & half-saturation constant for ammonium uptake & 0.5 \unit[mmol m$^{-3}$]  & \\
    $k_N$ & half-saturation constant for nitrate uptake & 0.5 \unit[mmol m$^{-3}$]  & \\
    $k_P$ & half-saturation constant for phytoplankton mortality & 1 \unit[mmol m$^{-3}$]  & \\
    $k_Z$ & half-saturation constant for zooplankton mortality & 3 \unit[mmol m$^{-3}$]  & \\
    $k_w$ & light attenuation due to water & 0.04 \unit[m$^{-1}$] & \\ 
    $k_c$ & light attenuation due to phytoplankton & 0.03 \unit[m$^2$ mmol$^{-1}$] & \\ 
    $\lambda_{\mathrm{bio}}$ & rate of the phytoplankton and zooplankton transfer into detritus & 0.05 \unit[day$^{-1}$]& \\
    $\lambda_A$ & nitrification rate  & 0.03 \unit[day$^{-1}$]& \\
    $\mu_P$ & phytoplankton mortality rate & 0.05 \unit[day$^{-1}$] & \\
    $\mu_Z$ & zooplankton mortality rate & 0.2 \unit[day$^{-1}$] & \\
    $\mu_D$ & detritus reference mineralisaiton rate & 0.05 \unit[day$^{-1}$] & \\
    $\Psi$ & strength of ammonium inhibition of nitrate uptake & 2.9 \unit[(mmol m$^{-3}$)$^{-1}$] & \\
    $p_P$ & relative grazing preference for phytoplankton & 0.75 & \\
    $p_D$ & relative grazing preference for detritus & 0.25 & \\
    $\mathrm{Q}_N$ & non-dimensional nitrate limiting factor && \eqref{eq:bio6_qn}  \\
    $\mathrm{Q}_A$ & non-dimensional ammonium limiting factor && \eqref{eq:bio6_qa}  \\
    $\mathrm{R}_P$ & Chl growth scaling factor && \eqref{eq:bio6_chlgrowth} \\
    $v$ & Maximum phytoplankton growth rate & 1 \unit[day$^{-1}$] & \\
    $w_g$ & detritus sinking velocity & 10 \unit[m day$^{-1}$]& \\
    $z$ & depth && \\
    $\theta$ & Chl to carbon ratio, \unit[mg Chl mgC$^{-1}$] && \\
    $\theta_m$ & maximum Chl to carbon ratio & 0.05 \unit[mg Chl mgC$^{-1}$] &\\
    $\zeta$ & conversion factor from \unit[gC] to \unit[mmolN] based on C:N ratio of 6.5 & 0.0128 \unit[mmolN ngC$^{-1}$] & \\
    \hline
  \end{tabular}
  \caption{Meanings of symbols in the six-component biology model. Typical values are
    provided for externally set parameters.}
  \label{tab:bioparameters6}
\end{table}

Unlike the model of \citet{popova2006} we use a continuous model, with no
change of equations (bar one exception) above or below the photic zone. For
our purposes, the photic zone is defined as 100m water depth. First we calculate
$\theta$:
\begin{equation}
    \theta = \frac{\mathrm{Chl}}{\mathrm{P}\zeta}
\label{eq:bio6_theta}
\end{equation}

However, at low light levels, $\mathrm{Chl}$ might be zero, therefore
we take the limit that $\theta \rightarrow \zeta$ at low levels ($1e^{-7}$) of
chlorphyll-a or photoplankton.

We then calculate $\alpha$:
\begin{equation}
    alpha = \alpha_c \theta
\label{eq:bio6_alpha}
\end{equation}

Using the PAR available at each vertex of the mesh, we now calculate the
light-limited phytoplankton growth rate, $J$:
\begin{equation}
    J = \frac{v\alpha I_n}{\sqrt{v^2 + \alpha^2 + I_n^2}}
\label{eq:bio6_j}
\end{equation}

The limiting factors on nitrate and ammonium are now calculated:
\begin{equation}
    \mathrm{Q}_N = \frac{N\exp^{-\Psi A}}{K_N+N}\label{eq:bio6_qn}, 
\end{equation}
\begin{equation}
    \mathrm{Q}_A = \frac{A}{K_A+A}\label{eq:bio6_qa}
\end{equation}

From these the diagnostic field, primary production ($\mathrm{X_P}$), can be calculated:
\begin{equation}
    \mathrm{X_P}=J\left(Q_N + Q_A\right)P
\label{eq:bio6_primprod}
\end{equation}

The chlorophyll growth scaling factor is given by:
\begin{equation}
    \mathrm{R}_P = Q_N Q_A \left(\frac{\theta_m}{\theta}\right) \left(\frac{v}{\sqrt{v^2 + \alpha^2 + I_n^2}}\right)
\label{eq:bio6_chlgrowth}
\end{equation}

The zooplankton grazing terms are now calculated:
\begin{equation}
    \mathrm{G}_P = \frac{gp_PP^2Z}{\epsilon + \left(p_PP^2 + p_DD^2\right)}\label{eq:bio6_gp},
\end{equation}
\begin{equation}
    \mathrm{G}_D = \frac{gp_DD^2*Z}{\epsilon+\left(p_PP^2 + p_DD^2\right)}\label{eq:bio6_gd}
\end{equation}

Finally, the four death rates and re-mineralisation rates are calculated:
\begin{equation}
    \mathrm{De}_P = \frac{\mu_pP^2}{P+k_p} + \lambda_{\mathrm{bio}}*P\label{eq:bio6_dp},
\end{equation}
\begin{equation}
    \mathrm{De}_Z = \frac{\mu_zZ^3}{Z+k_z} + \lambda_{\mathrm{bio}}*Z\label{eq:bio6_dz},
\end{equation}
\begin{equation}
    \mathrm{De}_D = \mu_DD+\lambda_{\mathrm{bio}}*P + \lambda_{\mathrm{bio}}*Z\label{eq:bio6_dd},
\end{equation}
\begin{equation}
    \mathrm{De}_A = \lambda_A A \; \mbox{where $z < 100$}\label{eq:bio6_da}
\end{equation}


\subsection{Photosynthetically active radiation (PAR)}\label{sec:I}

Phytoplankton depends on the levels of light in the water column at the
frequencies useful for photosynthesis. This is referred to as
photosynthetically active radiation. Sunlight falls on the water surface and
is absorbed by both water and phytoplankton. This is represented by the
following equation:
\begin{equation}
  \ppx[z]{I}=(A_{\mathrm{water}}+A_PP)I,
\end{equation}
where $A_{\mathrm{water}}$ and $A_P$ are the absorption rates of
photosynthetically active radiation by water and phytoplankton respectively.

This equation has the form of a one-dimensional advection equation and
therefore requires a single boundary condition: the amount of light incident
on the water surface. This is a Dirichlet condition on $I$.

As the PAR equation is relatively trivial to solve, the following options
are recommended:
\begin{itemize}
\item Discontinuous discretisation
\item Solver: gmres iterative method, with no preconditioner.
\end{itemize}

\subsection{Detritus falling velocity}\label{sec:detritus}

Phytoplankton, zooplankton and nutrient are assumed to be neutrally buoyant
and therefore move only under the advecting velocity of the water or by
diffusive mixing. Detritus, on the other hand, is assumed to be denser than
water so it will slowly sink through the water-column. This is modelled by
modifying the advecting velocity in the advection-diffusion equation for
detritus by subtracting a sinking velocity $u_{\mathrm{sink}}$ from the
vertical component of the advecting velocity.

\section{Sediments}

\fluidity\ is capable of simulating an unlimited number of sediment concentration fields.
Each sediment field, with concentration, $c_{i}$, behaves as any other tracer field,
except that it is subject to a settling velocity, $u_{si}$. The equation of conservation
of suspended sediment mass thus takes the form:

\begin{equation}\label{eq:sediment_conc}
  \ppt{c_i} + \nabla \cdot c_i ({\bf u} - \delta_{j3}u_{si}) = \nabla \cdot (\kaptens \nabla c_i)
\end{equation}

Source and absorption terms have been removed from the above equation. These will only be
present on region boundaries.

Each sediment field represents a discrete sediment type with a specific diameter and
density. A distribution of sediment types can be achieved by using multiple sediment
fields.

{\bf Notes on model set up}

Each sediment field must have a sinking velocity. Note that this is not shown as a
required element in the options tree as it is inherited as a standard option for all
scalar fields.

A sediment density, and sediment bedload field must also be defined. The sediment bedload
field stores a record of sediment that has exited the modelled region due to settling of
sediment particles.

To use sediment, a linear equation of state must also be enabled
\option{\ldots/equation\_of\_state/fluids/linear}

\subsection{Hindered Sinking Velocity}

The effect of suspended sediment concentration on the fall velocity can be taken into
account by making the Sinking Velocity field diagnostic. The equation of Richardson and
Zaki [1954] is then used to calculate the hindered sinking velocity, $u_{si}$, based upon
the unhindered sinking velocity, $u_{s0}$, and the total concentration of sediment, $c$.

\begin{equation}\label{eq:hindered_sinking_velocity}
  u_{si} = u_{s0}(1-c)^{2.39}
\end{equation}

\subsection{Deposition and erosion}

A surface can be defined, the sea-bed, which is a sink for sediment. Once sediment fluxes
through this surface it is removed from the system and stored in a separate field: the
Bedload field. Each sediment class has its on bedload field.

Erosion of this bed can be modelled by applying the sediment\_reentrainment boundary
condition. There are several options for the re-entrainment algorithm that is used to
calculate the amount of sediment eroded from the bed.

\noindent
1. Garcia's re-entrainment algorithm

Erosion occurs at a rate based upon the shear velocity of the flow at the bed, $u^*$, the
distribution of particle classes in the bed, and the particle Reynolds number,
$R_{p,i}$. The dimensionless entrainment rate for the i$^{th}$ sediment class, $E_i$, is
given by the following equation:

\begin{equation}
  E_i = F_i \frac{AZ_i^5}{1-AZ_i^5/0.3}
\end{equation}

\begin{equation}
  Z_i = \lambda_m \frac{u^*}{u_{si}} R_{p,i}^{0.6} \left (\frac{d_i}{d_{50}} \right)^{0.2}
\end{equation}

\noindent
Where $F_i$ is the volume fraction of the relevant sediment class in the bed, $d_i$ is the
diameter of the sediment in the i$^{th}$ sediment class and $d_{50}$ is the diameter for
which 50\% of the sediment in the bed is finer. $A$ is a constant of value $1.3 \times
10^7$

\noindent
$u^*$ and $R_{p,i}$ are defined by the following equations:

\begin{equation}
  u^* = \sqrt{\tau_b/\rho}
\end{equation}

\begin{equation}
  R_{p,i} = \sqrt{Rgd^{3}}/\nu
\end{equation}

This is given dimension by multiplying by the sinking velocity, $u_{si}$, such that the
total entrainment flux is:

\begin{equation}
  E_{m} = u_{si}E_i
\end{equation}

\noindent
2. Generic re-entrainment algorithm

Erosion occurs when the bed-shear stress is greater than the critical shear stress. Each
sediment class has a separate shear stress, which can be input or calculated depending on
the options chosen. Erosion flux, $E_m$ is implemented as a Neumann boundary condition on
the bedload/erosion surface.

\begin{equation}\label{eq:sediment_erosion_rate}
  E_m = E_{0m}\left(1-\varphi\right)\frac{\tau_{sf} - \tau_{cm}}{\tau_{cm}}
\end{equation}

\noindent
where $E_{0m}$ is the bed erodibility constant (kgm$^{-1}$s${-1}$) for sediment class $m$,
$\tau_{sf}$ is the bed-shear stress, $\varphi$ is the bed porosity (typically 0.3) and
$\tau_{cm}$ is the critical shear stress for sediment class $m$. The critical shear stress
can be input by the user or automatically calculated using:

\begin{equation}\label{eq:critical_shear_stress}
  \tau_{cm} = 0.041\left(s-1\right)\rho gD
\end{equation}

\noindent
where s is the relative density of the sediment, i.e. $\frac{\rho_{S_{m}}}{\rho}$ and $D$
is the sediment diameter (mm). The SedimentDepositon field effectively mixes the deposited
sediment, so order of bedload is not preserved.

\subsection{Sediment concentration dependent viscosity}

The viscosity is also affected by the concentration of suspended sediment. This can be
taken account for by using the sediment\_concentration\_dependent\_viscosity algorithm on
a diagnostic viscosity field. If using a sub-grid scale parameterisation this must be
applied to the relevant background viscosity field.

The equation used is that suggested by Krieger and Dougherty, 1959, and more recently by
Sequeiros, 2009.  Viscosity, $\nu$, is a function of the zero sediment concentration
viscosity, $\nu_0$, and the total sediment concentration, $c$, as follows.

\begin{equation}\label{eq:sediment_concentration_dependent_viscosity}
  \nu = \nu_{0}(1-c/0.65)^{-1.625}
\end{equation}

Note: a ZeroSedimentConcentrationViscosity tensor field is required.

\newpage

\section{Population balance modelling}

Dispersed objects are found widely in many diverse applications and the need to model their evolution has given rise to the development of the population balance equation (PBE). Dispersed items, which may be particles, cells or molecules, have a distribution that usually changes in time and space. The population balance equation is a conservation equation for the number of dispersed particles that models the evolution of this distribution \citep{ramkrishna2000population}. Fluidity can solve the PBE using the direct quadrature method of moments (DQMOM), which is a quadrature based moment method (QBMM) proposed by \citet{marchisio2005solution}.

\subsection{Population balance equation}
The PBE can be written as \citep{ramkrishna2000population}:
\begin{equation}
\frac{\partial n(\xi,\vec{x},t)}{\partial t} + \nabla \cdot \left( \langle \vec{u}|\xi \rangle \, n \right) - \nabla \cdot \left( \overline{\overline D}_\vec{x} (\xi,\vec{x},t) \nabla n \right) = S_\xi (\xi,\vec{x},t),
\label{eq:pbe}
\end{equation}
where $n(\xi,\vec{x},t)$ is the dispersed phase number density function, with $\xi$ and $\vec{x}$  being the internal and the external coordinates, respectively. 
In many cases, the PBE is formulated in terms of the particle volume as the internal coordinate due to the volume being conserved in breakage and aggregation processes. 
In Fluidity, though, particle size has been taken as the internal coordinate and all conservation terms have been written using a length-based formulation. Since particle size is the internal variable of interest when modelling polydispersed flows, it is convenient to formulate the PBE in terms of the particle length instead of its volume. Also, while implementing the method of moments to solve the PBE in this work, integral moments are used for the length-based NDF instead of fractional moments that are generally needed for the volume-based NDF. This simplifies the calculations mathematically.

The number density function is defined such that $n(\xi,\vec{x},t) \d \xi \d V_{\vec{x}}$ is the average number of dispersed particles contained in an infinitesimal volume $\d \xi \d V_{\vec{x}}$ centered around the particle state ($\xi,\vec{x}$). Therefore the total number of particles in the system is given as:
\begin{equation}
n_{tot}=\int_{\Omega_{\xi}} \int_{\Omega_{\vec{x}}} n(\xi,\vec{x},t) \d \xi \d V_{\vec{x}},
\end{equation}
where $\xi \in \Omega_{\xi}$ and $\vec{x} \in \Omega_{\vec{x}}$.

In Equation~\eqref{eq:pbe}, $\langle \vec{u}|\xi \rangle$ is the mean dispersed phase velocity field conditional to the particle size $\xi$. This velocity field is responsible for convecting the particles in the physical domain. $\overline{\overline D}_\vec{x}$ in the third term on the left hand side of Equation~\eqref{eq:pbe} is the spatial diffusion tensor. Spatial diffusion is useful when modelling the Brownian motion effects in the evolution of fine particle distribution.

$S_\xi$, called the source term in Equation~\eqref{eq:pbe}, includes all terms containing derivatives or integrals with respect to the particle size $\xi$. $S_{\xi}$ includes the growth term, diffusion in the internal (or phase) space, and the birth and death functions due to particle breakage and coalescence.

\subsection{Source terms in the PBE}
\subsubsection{Growth}
Growth can be understood as convection in the particle internal space, smooth movement from one internal coordinate (i.e. particle size) to another. 
The contribution to the source term in Equation~\eqref{eq:pbe} due to dispersed phase growth is given by:
\begin{equation}
S_\xi=-\frac{\partial}{\partial \xi} \left( \dot{\xi}(\xi,\vec{x},t) n \right),
\label{eq:pbe_source_term_growth}
\end{equation}
where $\dot{\xi}$ ($=\frac{\d \xi}{\d t}$) is the growth rate, which is analogous to the velocity in external space.

Some of the examples of growth in the PBE are: an increase in particle size due to deposition from the continuous phase in solutions (e.g. crystal growth) and an increase in bubble size as they rise in a column due to the decrease in hydrostatic pressure. Growth is considered a continuous process even though the actual physical process might not be strictly continuous. For example, in the case of crystal growth, the deposition process is molecular but crystal size increases infinitesimally in an infinitesimal time and hence the process can be considered continuous. The definition of growth, therefore, is time-scale dependent.

\subsubsection{Internal diffusion}
Certain processes can be modelled as diffusive with the diffusion occurring in the internal space in the PBE, e.g. when particles grow at different rates \citep{jones2002using}.
The source term corresponding to this internal diffusion is given by:
\begin{equation}
S_\xi=\frac{\partial^2}{\partial \xi^2} \left(D_{\xi} (\xi,\vec{x},t) n \right),
\label{eq:pbe_source_term_internal_diff}
\end{equation}
where $D_{\xi} (\xi,\vec{x},t)$ is the diffusivity in the internal space.

\subsubsection{Breakage and coalescence} \label{sec:pbe_brk_aggr}
Breakage and coalescence of particles are discontinuous events that lead to the birth and death of particles in a very short time (assuming particle interactions though hard-sphere potential). 
The contribution to the source term of Equation~\eqref{eq:pbe} from the birth and death functions due to particle breakage and coalescence is given as:
\begin{equation}
S_\xi=B_B - D_B + B_C - D_C.
\label{eq:pbe_source_term_br_ag}
\end{equation}
The birth and death functions due to breakage are given as:
\begin{equation} \label{eq:B_B_pbe}
B_B(\xi)=\int_{\xi}^{\infty} \nu(\xi_1) a(\xi_1) b(\xi | \xi_1) n(\xi_1) \d \xi_1
\end{equation}
and 
\begin{equation} \label{eq:D_B_pbe}
D_B(\xi) = a(\xi) n(\xi),
\end{equation}
respectively. Here, $\nu(\xi)$, $a(\xi)$ and $b(\xi|\xi_1)$ are the breakage kernels that define the number of particles formed after breakage, the breakage frequency, and the daughter distribution function, respectively.
These three kernels are determined from physical models.
The breakage frequency (also known as breakage rate) is the fraction of particles breaking per unit time and has the units \SI{}{\per\second}. It is usually modelled probabilistically.
Binary breakage is considered in most applications, i.e. $\nu(\xi)=2$.
The daughter distribution function $b(\xi|\xi_1)$ defines the probability of formation of particles of size $\xi$ due to the breakage of particles from class $\xi_1$. This probability density function satisfies the normalisation condition $\int_0^{\xi_1} b(\xi|\xi_1) \d \xi = 1$. \label{pagelabel:ddf_properies} For binary breakage, the relation $b((\tilde{\xi}_1-\tilde{\xi})|\tilde{\xi}_1)$ = $b(\tilde{\xi}|\tilde{\xi}_1)$ holds. Additionally, the conservation of mass imposes the following condition on the daughter distribution function: $\tilde{\xi}_1=\nu(\tilde{\xi}_1) \int_0^{\tilde{\xi}_1} \tilde{\xi} b(\tilde{\xi}|\tilde{\xi}_1) \d \tilde{\xi}$. $\tilde{\xi}$ in the above equations refers to the particle volume corresponding to the particle size $\xi$. It must be noted that the breakage functions are local, i.e. the fragments share the same spatial location as the parent particle. 

Coalescence is described in terms of the number of particle pairs and a coalescence frequency.
If there is no statistical correlation between the colliding particles, the particle pairs can be defined as the product of two individual number densities.
The birth and death functions due to coalescence are, therefore, given as:
\begin{equation} 
B_C(\xi) = \frac{1}{\tilde{\delta}} \int_{0}^{\xi} \left(\frac{\xi^2}{\xi'^2}\right) \beta (\xi', \xi_1) n(\xi') n(\xi_1) \d \xi_1
\label{eq:birth_coalescence}
\end{equation}
and
\begin{equation} \label{eq:D_C_pbe}
D_C(\xi) =\int_0^{\infty} \beta (\xi, \xi_1) n(\xi) n(\xi_1) \d \xi_1,
\end{equation}
respectively, where $\beta(\xi',\xi_1)$ is the coalescence frequency for particles of sizes $\xi'$ and $\xi_1$, and it has the units \SI{}{\metre\cubed\per\second}.
In general, the coalescence frequency may also depend on the spatial coordinates of the two aggregating particles and time, but for simplicity it is assumed to be a function of particle size only in this implementation.
As for breakage, coalescence frequency is also determined from physical models.
The parameter $\tilde{\delta}$ represents the number of times identical pairs have been considered in the interval of integration and is equal to two for binary coalescence, which is the most common scenario for dilute systems \citep{ramkrishna2000population}.
In Equation~\eqref{eq:birth_coalescence}, $\xi'$ is given as $\xi'^3=\xi^3-\xi_1^3$, i.e. the particles are considered three-dimensional and the volume of the resulting particle class ($\xi$) is the sum of the volumes from the contributing size classes ($\xi'$ and $\xi_1$). See \citet{bhutani2016} and \citet{bhutani2016polydispersed} for more details on the breakage and coalescence terms. 

\subsection{Solution method -- DQMOM}

Fluidity uses the direct quadrature method of moments \citep{marchisio2005solution} to solve the population balance equation.
This method requires fewer equations to be solved in order to get a good estimate of the particle size distribution, i.e. 4--6 equations against 50--200 equations needed in the other popular method---the classes method \citep{marchisio2003quadratureb}. 
\subsubsection{DQMOM approximation}
This method uses a quadrature approximation to the NDF, given by:
\begin{equation}
n \left( \xi, \vec{x}, t \right) = \sum_{j=1}^N w_{j} \left( \vec{x}, t \right) \delta \left[ \xi - \langle \xi \rangle_{j} \left( \vec{x}, t \right) \right],
\label{eq:DQMOM_ndf}
\end{equation}
as opposed to approximating the integrals in some other methods of moments. $\delta$ in the above equation is the Dirac delta function, $N$ is the total number of quadrature points, and $w_j$ and $\langle \xi \rangle_{j}$ are the weights and abscissas in the DQMOM approximation, respectively. This choice for the approximation to the NDF provides a convenient closure for the source term $S_{\xi}$ in Equation~\eqref{eq:pbe}.

The major advantages of using DQMOM over any other method of moments is that each weight and abscissa can be defined as a function of space, which makes it easier to implement this method, and very few abscissas are needed to accurately model the NDF due to the adaptive quadrature approach.
Additionally, each node can have its own velocity field, if required, as stated in \citet{marchisio2005solution}.

\subsubsection{Population balance equation with DQMOM}
By substituting Equation~\eqref{eq:DQMOM_ndf} in the PBE (Equation~\eqref{eq:pbe}) and replacing $w_{j} \langle \xi \rangle_{j} = \varsigma_{j}$ as the weighted-abscissa, the PBE gets transformed to:
\begin{equation} \label{eq:dqmomdiff}
\begin{split}
\sum_{j=1}^N \delta&(\xi - \langle \xi \rangle_{j}) \left[ \frac{\partial w_{j}}{\partial t} + 
\nabla \cdot \left( \langle \vec{u} \rangle_j w_j \right) - 
\nabla \cdot \left( \overline{\overline{D}}_{\vec{x}} \nabla w_j \right)
 \right]\\
&- \sum_{j=1}^N \delta'(\xi - \langle \xi \rangle_{j}) \left\{ \left[ \frac{\partial \varsigma_{j}}{\partial t} + 
\nabla \cdot \left( \langle \vec{u} \rangle_j \varsigma_j \right)
 - 
 \nabla \cdot \left( \overline{\overline{D}}_{\vec{x}} \nabla \varsigma_j \right) \right]
  \right.\\
&- \left.\langle \xi \rangle_{j} \left[ \frac{\partial w_{j}}{\partial t} + 
\nabla \cdot \left( \langle \vec{u} \rangle_j w_j \right)
 - 
 \nabla \cdot \left( \overline{\overline{D}}_{\vec{x}} \nabla w_j \right)
  \right] \right\}\\
&- \sum_{j=1}^N \delta''(\xi - \langle \xi \rangle_{j}) \left[ \left( \nabla \langle \xi \rangle_j \cdot \overline{\overline{D}}_{\vec{x}} \cdot \nabla \langle \xi \rangle_j \right) w_j \right] = S_{\xi},
\end{split}
\end{equation}
where $\delta'(\xi - \langle \xi \rangle_{j})$ and $\delta''(\xi - \langle \xi \rangle_{j})$ are the first and second derivatives of the function $\delta(\xi - \langle \xi \rangle_{j})$.

Although the advection term in Equation~\eqref{eq:pbe} included a velocity field conditional on the abscissa, a common velocity field approximation to the dispersed phase is used in Fluidity without any dependence on the particle size for simplicity. This velocity field is obtained from the solution to the momentum equation.

The DQMOM transport equations are defined as:
\begin{subequations}
\begin{equation} \label{eq:dqmomweight}
\frac{\partial w_{j}}{\partial t} + \nabla \cdot \left( \langle \vec{u} \rangle_j w_j \right) - \nabla \cdot \left( \overline{\overline{D}}_{\vec{x}} \nabla w_j \right) = \widetilde{g}_{j}
\end{equation}
and 
\begin{equation} \label{eq:dqmomweightabscissa}
\frac{\partial \varsigma_{j}}{\partial t} + \nabla \cdot \left( \langle \vec{u} \rangle_j \varsigma_j \right) - \nabla \cdot \left( \overline{\overline{D}}_{\vec{x}} \nabla \varsigma_j \right)  = \widetilde{h}_{j},
\end{equation}
\end{subequations}
where $j=1,2,...,N$. 
The equations above are similar to an advection--diffusion system and can be solved for the scalar fields $w_j$ and $\varsigma_j$ once the source terms $\widetilde{g}_{j}$ and $\widetilde{h}_{j}$ have been calculated. 

Taking the $k^{\textrm{th}}$ moment of Equation~\eqref{eq:dqmomdiff}, it gets transformed to a linear system, given as:
\begin{equation} \label{eq:dqmom_lineq_raw}
(1-k) \sum_{j=1}^N \langle \xi \rangle_{j}^k \widetilde{g}_{j} + k \sum_{j=1}^N \langle \xi \rangle_{j}^{k-1} \widetilde{h}_{j} = \overline{S}_k^{(N)} + \overline{C}_k,
\end{equation}
which can be solved to obtain $\widetilde{g}_{j}$ and $\widetilde{h}_{j}$. In deriving this linear system, either integral or fractional-order moments can be used depending on physical considerations. Since a length-based formulation is used for the NDF in Fluidity, integral moments make more sense and hence $k=1,2,...,2N$ is used.
In the above equation $\overline{S}_k^{(N)}$ is the $k$\textsuperscript{th} moment of the source term of the PBE with the DQMOM quadrature approximation applied, i.e.
\begin{equation}
\overline{S}_k^{(N)} = \int_{0}^{\infty} \xi^k S_{\xi} d\xi.
\label{eq:source_moment}
\end{equation}
Also
\begin{equation}
\overline{C}_k = k(k-1) \sum_{j=1}^N \langle \xi \rangle_{j}^{k-2} C_{j},
\end{equation}
where $C_j$ is defined as:
\begin{equation}
C_j =\left( \nabla \langle \xi \rangle_j \cdot \overline{\overline{D}}_{\vec{x}} \cdot \nabla \langle \xi \rangle_j \right) w_j.
\end{equation}
Due to the inherent nature of the DQMOM quadrature approximation, the moment source term $\overline{S}_k^{(N)}$ is closed for any functional form.
A detailed formulation of the terms that constitute the PBE source term can be found in \citet{bhutani2016}. 

The moments of the NDF can be calculated from the DQMOM weights and abscissas using:
\begin{equation} \label{eq:moments_quad_wts_abs}
m_k = \sum_{j=1}^N w_{j} \langle \xi \rangle_{j}^k.
\end{equation}
The Sauter mean diameter $d_{32}$, defined as the ratio of the third to the second moment of the NDF, is used for the estimation of the particle size and can be calculated using the above equation.

\subsection{Implementation of DQMOM in Fluidity}
\label{DQMOM_implementation_Fluidity}
Equations~\eqref{eq:dqmomweight} and \eqref{eq:dqmomweightabscissa} form a system of combined advection--diffusion equations that are coupled in the source terms. 
These equations are solved for the weights $w_j$ and the weighted-abscissas $\varsigma_j$, which are then used to calculate the moments and eventually the Sauter mean diameter $d_{32}$.

The number of equations that need to be solved depends on the value of the number of quadrature nodes, $N$, chosen to represent the NDF. Solution accuracy improves as the number of nodes are increased but $N=2$ to $N=3$ has been reported to produce sufficiently accurate solution of the PBE  with a small computation cost \citep{marchisio2005solution}. 
DQMOM however has been implemented in Fluidity to solve the PBE for any value for the number of nodes $N$.

\subsubsection{Matrix form of the DQMOM linear system}
The DQMOM linear system (Equation~\eqref{eq:dqmom_lineq_raw}) is implemented in Fluidity in a matrix form as:
\begin{equation} \label{eq:dqmom_matrix_eqn}
\mathbf{A} \pmb{\alpha} = \mathbf{d},
\end{equation}
where the square matrix $\mathbf{A} = \left[ \mathbf{A_1} \ \mathbf{A_2} \right ]$. The matrices  $\mathbf{A_1}$ and $\mathbf{A_2}$ are given as follows:
\begin{equation} \label{eq:dqmom_matrix_A1}
\mathbf{A_1} = 	\begin{bmatrix}
		1 & \dots & 1 \\
		0 & \dots & 0 \\
		-\langle \xi \rangle_1^2 & \dots & -\langle \xi \rangle_N^2 \\
		\vdots & \vdots & \vdots \\
		2 (1-N) \langle \xi \rangle_1^{2N-1} & \dots & 2 (1-N) \langle \xi \rangle_N^{2N-1} \\		
		\end{bmatrix}
\end{equation}
and
\begin{equation} \label{eq:dqmom_matrix_A2}
\mathbf{A_2} = 	\begin{bmatrix}
		0 & \dots & 0 \\
		1 & \dots & 1 \\
		2 \langle \xi \rangle_1 & \dots & 2 \langle \xi \rangle_N \\
		\vdots & \vdots & \vdots \\
		(2N-1) \langle \xi \rangle_1^{2N-2} & \dots & (2N-1) \langle \xi \rangle_N^{2N-2} \\		
		\end{bmatrix}.
\end{equation}
The vector of unknowns $\pmb{\alpha}$ is given as:
\begin{equation}
\pmb{\alpha} = 	\begin{bmatrix}
		\widetilde{g}_{1} & \dots & \widetilde{g}_{N} \ \widetilde{h}_{1} & \dots & \widetilde{h}_{N} \\	
		\end{bmatrix}^\mathrm{T} = 
		\begin{bmatrix}
		\mathbf{\widetilde{g}}\\	
		\mathbf{\widetilde{h}}\\
		\end{bmatrix},
\end{equation}
and the right hand side vector $\mathbf{d}$ is given as:
\begin{equation} \label{eq:vector_d_eqn}
\mathbf{d} = \mathbf{A_3} \mathbf{C} + \pmb{\beta},
\end{equation}
where 
\begin{equation} \label{eq:C}
\mathbf{C} = 	\begin{bmatrix}
		C_1 & \dots & C_N \\	
		\end{bmatrix}^\mathrm{T},
\end{equation}

\begin{equation} \label{eq:A_3}
\mathbf{A_3} = 	\begin{bmatrix}
		0 & \dots & 0 \\
		0 & \dots & 0 \\
		2 & \dots & 2 \\
		6 \langle \xi \rangle_1 & \dots & 6 \langle \xi \rangle_N \\
		\vdots & \vdots & \vdots \\
		2(2N-1)(N-1) \langle \xi \rangle_1^{2N-3} & \dots & 2(2N-1)(N-1) \langle \xi \rangle_N^{2N-3} \\		
		\end{bmatrix}
\end{equation}
and 
\begin{equation} \label{eq:beta_vector}
\pmb{\beta} = 	\begin{bmatrix}
		\overline{S}_0^{(N)} & \dots & \overline{S}_{2N-1}^{(N)} \\	
		\end{bmatrix}^\mathrm{T}.
\end{equation}

\subsubsection{Initial conditions for DQMOM scalars} \label{sec:pbe_initial_conditions}
The population balance solver has been implemented in Fluidity to accept initial conditions for either the initial weights and weighted-abscissas for all nodes, i.e. values for $w_1, \dots, w_N$ and $\varsigma_1, \dots, \varsigma_N$, or initial conditions for the moments $m_0, m_1, \dots, m_{2N-1}$. The product-difference (PD) algorithm \citep{gordon1968error} is used to convert the $2N$ initial moments to weights and abscissas. If the initial number density function is known, moments have to be calculated externally and supplied as an input to the Fluidity PBE solver. See \cite{bhutani2016} for the details of the PD algorithm implemented in Fluidity. 

If the initial distribution is mono-sized, i.e. all dispersed particles have the same size, the left hand side matrix $\mathbf{A}$ in Equation~\eqref{eq:dqmom_matrix_eqn} becomes singular. In the context of DQMOM, this translates to the fact that all the $N$ abscissas may not be needed to completely describe the initial NDF. To overcome this problem of singularity, two techniques have been implemented in Fluidity. One perturbs the abscissa values by a small amount so that it comes out of singularity as time progresses \citep{marchisio2005solution} (only one perturbation in allowed in \fluidity\ for now). Another method forces the source term $S_{\xi}$ equal to zero. Singularity of matrix $\vec{A}$ can also occur somewhere in the middle of the code execution instead of at $t=0$ and the two methods discussed above are applicable to those situations too. 

\subsubsection{Implementation algorithm} \label{sec:pbe_implementation_algo}

\begin{algorithmic}[1]
\Procedure{DQMOM}{}
    \parState{Initialise DQMOM weights ($w_j$) and weighted-abscissas ($\varsigma_j$). Use PD algorithm if initial conditions for moments are specified.} 
    \parState{Calculate abscissas ($\langle\xi\rangle_j$) for all DQMOM quadrature points ($1,2,\dots,N$).}
    \For {$t=0$ to T}
        \For {non-linear iterations = 1 to $N_{NL}$}
            \parState{Calculate LHS matrix $\vec{A}$ at all spatial quadrature points (Equations~\eqref{eq:dqmom_matrix_A1} and \eqref{eq:dqmom_matrix_A2}) using $\langle\xi\rangle_j$ from the previous time step.}
            \parState{Check for the singularity of $\vec{A}$ matrix. Keep perturbing the abscissas by a small value until the matrix $\vec{A}$ becomes non-singular (i.e. its reciprocal condition number exceeds a given minimum value).}
            \parState{Calculate source term vector $\pmb{\beta}$ for all spatial quadrature points. Include source term ($S_{\xi}$) contribution from growth, internal diffusion, breakage and aggregation.}
            \parState{Calculate matrix $\vec{A_3}$, vector $\vec{C}$ and therefore vector $\vec{d}$ using Equation~\eqref{eq:vector_d_eqn}.}
            \parState{Solve Equation~\eqref{eq:dqmom_matrix_eqn} for the vector $\pmb{\alpha}$=$\left[\vec{\widetilde{g}} \ \ \vec{\widetilde{h}} \right]^\mathrm{T}$ at all spatial quadrature points. Interpolate $\pmb{\alpha}$ from spatial quadrature points to mesh nodes using FE shape functions.}
            \parState{Solve the DQMOM transport Equations~\eqref{eq:dqmomweight} and \eqref{eq:dqmomweightabscissa} using Fluidity's FE framework. If needed, for negative source terms $\vec{\widetilde{g}}$ and $\vec{\widetilde{h}}$ apply source terms as absorption terms to ensure positivity.}
            \parState{Apply minimum weight limit, if required.}
            \parState{Calculate new estimate for $\langle\xi\rangle_j$ values.}
            \parState{Calculate moments and the Sauter mean diameter.}
            
        \EndFor
	\EndFor
\EndProcedure
\end{algorithmic}

It can be seen that the DQMOM method is inherently explicit in its approach as the source terms for Equations~\eqref{eq:dqmomweight} and \eqref{eq:dqmomweightabscissa} are calculated at the previous time step. 
In addition, as mentioned in the algorithm above, the matrix $\vec{A}$ and the RHS vector $\vec{d}$ are calculated at the spatial quadrature points instead of the mesh nodes. This is because the spatial derivatives of the abscissas are not defined at the mesh nodes when using DG discretisations.
However, it is possible to evaluate $\vec{A}$ and $\vec{d}$ directly on the mesh nodes in Fluidity (if a continuous spatial discretisation is used or if spatial diffusion is absent). This nodal implementation uses the nodal evaluation of $\widetilde{g}_j$ and $\widetilde{h}_j$ to minimise errors due to spatial interpolation.
Note that the spatial diffusion term can not be included in the nodal version of the PBE solver and only continuous elements can be used with this nodal implementation. In order to compile the population balance code with the nodal implementation, rename the file 
\option{DQMOM\_node\_source}.\option{F90} to \option{DQMOM}.\option{F90} in the \option{population\_balance} directory and recompile Fluidity.
The default quadrature version allows the use of the diffusion term and DG discretisations in the PBE solver.
