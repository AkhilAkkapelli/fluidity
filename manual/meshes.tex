\chapter{Meshing for \fluidity}\label{chap:meshes}
\index{mesh!generation}
\index{grid|see{mesh}}

In each run of \fluidity\ an input computational mesh needs to 
be provided. Even in adaptive mesh runs an initial mesh is needed
to define the initial condition of the fields on. This initial mesh
can be adapted by \fluidity\ at the start of the simulation, 
in order to optimize the accurate representation of the initial 
condition. The input mesh is then used as a starting point for the
mesh adaptivity algorithm.

\section{Supported mesh formats}
\label{sect:supported_mesh_formats}
\fluidity\ supports two mesh file formats:
\begin{enumerate}
\item Gmsh .msh files. Gmsh is a mesh generator freely available on the
web at \url{http://geuz.org/gmsh/}, and is included in Linux distributions 
such as Ubuntu. This is the recommended file format. Instructions on how 
to generate a mesh using gmsh can be found ...
\item Triangle format. This is stored as a set of 3 files: a .node file,
a .ele file and a .face (3D) or .edge (2D) or .bound (1D) file. This file format
is mainly supported for special purposes, like 1D meshes, and some offline 
tools.
\end{enumerate}

Both simplicial meshes (triangles in 2D and tetrahedrals in 3D) and cubical
meshes (quadrilaterals in 2D and hexagons in 3D) are supported. Note however
that the choice between these two types of meshes also has an influence
on the choice of the polynomial space that is used in the 
finite element discretisation. Choosing for instance a polynomial degree 
of 1 (see ...) means that for simplicial meshes linear polynomials are used
(the \Pone discretisation) whereas on cubical meshes this leads to bilinear
polynomials (known as a \Qone discretisation). Therefore to use the well known
\Poo discretisation or the \PoDGPt discretisation (recommended for large scale
ocean applications) a simplicial input mesh is needed. Structured meshes in two
or three dimensions can be generated by stacking two triangles in a rectangle,
or six tetrahedrals in a cube respectively. Such meshes are easily obtained
using gmsh.

For a detailed discussion of supported mesh formats see appendix~\ref{chap:mesh_formats}.

\section{Surface and regions ids}
\index{surface ID}
\index{region ID}
Surface ids are used in \fluidity\ to mark different parts of the boundary of the
computational domain so that different boundary conditions can be associated
with them. Regions ids are used to mark different parts of the domain itself.
Both ids can be defined in gmsh by assigning physical ids to different
geometrical objects. In two dimensions surface ids can be defined by assigning a
physical id to each group of lines that make up a part of the boundary that
needs to be considered seperately. Region ids can be defined by dividing the
domain up in different (groups of) surfaces and assigning different 
physical ids to them. Similarly, in three dimensions surface ids are defined by
assigining physical surface ids in gmsh, and regions ids by assigning physical
volume ids.

It is recommended, and required in parallel, that all parts of the domain
boundary are marked with a surface id. Region ids are optional. They can be used
to instruct the adaptivity library to strictly maintain the interface between 
different regions of the domain. They also come very handy to set different constant 
field values in the regions.

\section{Mesh types}
\subsection{Extruded meshes}
\label{sect:extruded_meshes}
\index{mesh!extruded}

Given a 1D or 2D input mesh, \fluidity\ can extrude this
mesh to create a layered 2D or 3D mesh, on which simulations can be
performed. The extrusion is always downwards (in the direction of gravity), 
and the top of the domain is always flat, corresponding to the $y=0$-level 
in 2 dimension, the $z=0$ level in 3 dimensions, or the equilibrium 
free surface geoid when running on the sphere.

The advantage of this approach is that the user can provide a horizontal mesh,
that has been created in the normal way (usually gmsh), and all the
configuration related to bathymetry, number of layers and layer depths can be
done in the .flml (see \ref{Sect:extruded} for all options available). It also
enables the application of mesh adaptivity in the horizontal and vertical
independently. This means we can choose to apply adaptivity in the horizontal
only and keep a fixed number of layers, or we can choose to keep the horizontal
mesh fixed and dynamically adjust the vertical grid spacing (vertical
adaptivity). The combination of both horizontal and vertical adaptivity is
referred to as ``2+1'' adaptivity, which is described further in ...

\subsection{Periodic meshes}
\index{mesh!periodic} 
\index{periodic domain} 
\label{mesh!mesh types!periodic} 
Periodic meshes are those that are ``virtually'' connected in one or more directions. To make a periodic
mesh you must first create a triangle file where the edges that are periodic
can be mapped exactly by a simple transformation. For example, if the mesh
is periodic in the $x$-direction, the two edges must have nodes at exactly the
same height on each side. This can be easily accomplished using the
\lstinline[language=Bash]+create_aligned_mesh+ script in the scripts folder.

Alternatively, if you require a more complex periodic mesh with some structure between the periodic 
boundaries you can create one using \lstinline[language=Bash]{gmsh}. This can be achieved by 
setting up the periodic boundaries by using extrude and then deleting the 'internal' mesh.

The use of periodic domains requires additional configuration options. See
section \ref{Sect:periodic}.

\section{Meshing tools}
\index{mesh!meshing tools}

There are a number of meshing tools in tools directory. These can be built by running 
\lstinline[language = bash]+make fltools+ in the top directory of the \fluidity\ trunk.
The following binaries will then be created in the \lstinline+bin/+ directory (see section \ref{sect:fltools}).

\subsection{Mesh Verification}
\index{mesh!meshing tools!mesh verification}

The \lstinline[language = Bash]+checkmesh+ tool can be used to form a number of verification tests on a mesh
in triangle mesh format. More information can be found in section~\ref{sect:checkmesh}.

\subsection{Mesh creation}
\index{mesh!meshing tools!mesh creation}
The \lstinline[language = bash]+interval+ tool (section~\ref{sect:interval}) generates a 1D line mesh in triangle format. 

The \lstinline[language = bash]+gen_square_meshes+ tool (section~\ref{sect:gen_square_meshes})  will generate triangle files for multiple 2D square meshes with a specified number of internal nodes.

The \lstinline[language = bash]+create_aligned_mesh+ tool (section~\ref{sect:create_aligned_mesh}) creates the triangle files for a mesh that lines up in all directions, so that it can be made it into a singly, doubly or triply periodic mesh.

\subsection{Mesh conversion}
\index{mesh!meshing tools!mesh conversion}

The \lstinline[language = Bash]+gmsh2triangle+ tool converts ASCII Gmsh mesh files into triangle format. Whilst Fluidity
can read in Gmsh files directly as noted in section~\ref{sect:supported_mesh_formats}, in
cases where native Gmsh support does not work this tool should be used instead. More information can be 
found in section~\ref{sect:gmsh2triangle}.

The \lstinline[language = Bash]+triangle2vtu+ tool (section~\ref{sect:triangle2vtu}) can be used to convert triangle format files into vtu format. 

The \lstinline[language = Bash]+gmsh_mesh_transform+ script (section~\ref{sect:gmsh_mesh_transform}) applies a coordinate transformation to a region of a given mesh. 
The coordinate transformation is specified as a python expression and will be applied over a user-defined mesh region.

\subsection{Decomposing meshes for parallel}
\label{decomp_meshes_parallel}
\index{parallel!mesh decomposition}

\subsubsection{fldecomp}
\index{mesh!meshing tools!fldecomp}
\label{mesh!meshing tools!fldecomp}

For parallel simulations, you must use \lstinline[language=bash]+fldecomp+ to decompose a Gmsh
mesh into sub-meshes for each process. Here, only binary Gmsh files can be
used - see section \ref{sect:fldecomp} for details.

In order to run fldecomp, if your mesh file is \lstinline[language=bash]+foo.msh+
and you want to decompose into four parts, type:
\begin{lstlisting}[language = Bash]
`\fluiditysourcepath'/bin/fldecomp -n 4 -m gmsh foo
\end{lstlisting}

\subsubsection{flredecomp}
\index{mesh!meshing tools!flredecomp}
\label{mesh!meshing tools!flredecomp}
\lstinline[language=bash]+flredecomp+ is a tool similar to \lstinline[language=bash]+fldecomp+ but runs in parallel. 
It performs a re-decomposition of a Fluidity checkpoint.
For example, to decompose the serial file \lstinline+foo.flml+
into four parts, running on 4 processors type:

\begin{lstlisting}[language=bash]
mpiexec -n 4 `\fluiditysourcepath'/bin/flredecomp \
    -i 1 -o 4 foo foo_flredecomp
\end{lstlisting}

The output of running flredecomp is a series of mesh and vtu files as well
as the new flml; in this case \lstinline+foo_flredecomp.flml+. 
Note that \lstinline[language=bash]+flredecomp+ must be run on enough processors (the larger number of processors between input and output).
More information can be found in section~\ref{sect:flredecomp}.

\subsection{Decomposing a periodic mesh}
\index{mesh!meshing tools!periodise}
\label{sect:decomposing_meshes_periodise}

To be able to run \fluidity\ on a periodic mesh in parallel you have to use
two tools:

\begin{itemize}
\item periodise (section \ref{sect:periodise})
\item flredecomp (section \ref{sect:flredecomp})
\end{itemize}

The input to \lstinline+periodise+ is your flml (in this case
\lstinline{foo.flml}). This flml file should already contain the mapping for
the periodic boundary as described in section
\ref{Sect:periodic}. Periodise is run with the command:

\begin{lstlisting}[language=bash]
`\fluiditysourcepath'/bin/periodise foo.flml
\end{lstlisting}

The output is a new flml called \lstinline+foo_periodised.flml+ and the
periodic meshes. Next run flredecomp (section \ref{sect:flredecomp}) to decompose the mesh for the number of processors
required. The flml output by flredecomp is then used to execute the actual simulation:

\begin{lstlisting}[language=bash]
mpiexec -n [number of processors] \
   `\fluiditysourcepath'/bin/fluidity [options] \
   foo_periodised_flredecomp.flml
\end{lstlisting}

\section{Non-\fluidity\ tools}

In addition to the tools and capabilities of \fluidity, there are numerous
tools and software packages available for mesh generation. Here, we describe 
two of the tools commonly used.

\subsection{Terreno}
\index{mesh!meshing tools!Terreno}
\index{Terreno}

Terreno uses a 2D anisotropic mesh optimisation algorithm to explicitly optimise for 
element quality and bathymetric approximation while minimising the number of mesh
elements created. The shoreline used in the mesh generation process is the result 
of a polyline approximation algorithm that where the minimum length of the resulting 
edges is considered as well as the distance an edge is from a vertex on the original 
shoreline segment being approximated. The underlying philosophy is that meshing and 
approximation should be error driven and should minimise user intervention. The 
latter point is two pronged: usability is paramount and the user should not need 
to be an expert in mesh generation to generate high quality meshes for their ocean 
model; fundamentally there must be clearly defined objectives to the mesh generation 
process to ensure reproducibility of results. The result is an unstructured mesh, 
which may be anisotropic, which focuses resolution where it is required to optimally 
approximate the bathymetry of the domain. The criterion to judge the quality of the 
mesh is defined in terms of clearly defined objectives. An important feature of the 
approach is that it facilitates multi-objective mesh optimisation. This allows one to 
simultaneously optimise the approximation to other variables in addition to the 
bathymetry on the same mesh, such as back-scatter data from soundings, material 
properties or climatology data. 

See the \href{http://amcg.ese.ic.ac.uk/terreno}{Terreno website}\ for more information.

\subsection{Gmsh}
\index{mesh!meshing tools!gmsh}
\index{gmsh}
\label{sect:meshing_tools_non_fluidity_gmsh}

Gmsh is a 3D finite element mesh generator with a build-in CAD engine and post-processor.
Its design goal is to provide a fast, light and user-friendly meshing tool with parametric
input and advanced visualisation capabilities. Gmsh is built around four modules: geometry, 
mesh, solver and post-processing. The specification of any input to these modules is done
either interactively using the graphical user interface or in ASCII text files using Gmsh's
own scripting language. 

For more information see the \href{http://geuz.org/gmsh/}{Gmsh website}\ or the \href{http://amcg.ese.ic.ac.uk}{AMCG
website}. An online manual is available at \href{http://geuz.org/gmsh/doc/texinfo/gmsh.html}{geuz.org/gmsh/doc/texinfo/gmsh.html}.

\subsection{Importing contours from bathymetric data into Gmsh}
\index{mesh!generation!gmsh!entering! bathymetry! data! into! fluidity! using! Gmsh}
\index{Entering bathymetry data into fluidity using Gmsh}

Gmsh can be used to create a mesh of a `real' ocean domain for use with \fluidity. An online guide to using Gmsh's built in
GSHHS plug-in is available at \href{http://perso.uclouvain.be/jonathan.lambrechts/gmsh_ocean/}{gmsh\_ocean}.
It is also possible to import contours from arbitrary bathymetry data sources into Gmsh. A guide and sample code detailing this process will
in the future be available on the \href{http://amcg.ese.ic.ac.uk}{AMCG website}.
