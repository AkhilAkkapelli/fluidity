\chapter{Diamond}\label{chap:diamond}

\section{Installation}
Diamond depends on Python (version 2.3 or greater), PyGTK,
the 4Suite XML library, and the lxml library. All of these dependencies
are available in the Debian/Ubuntu repositories with the following command:
\begin{verbatim}
apt-get install python-gtk2 python-lxml python-4suite-xml
\end{verbatim}

Windows versions of these packages are available for download on their respective websites.

\section{Configuration files}

In accordance with normal Unix practice, system wide configuration for
diamond is stored in the \verb+/etc/diamond+ directory while per-user
configuration is stored in a \verb+.diamond+ directory in the user's home
directory.

\subsection{Colour}\label{sec:colour}

Diamond diff view uses colour to display differences between documents.
These colours are configured by a text file in the diamond directory called \verb+settings+
with the following contents:
\begin{verbatim}
[colour]
normal = black
insert = green
delete = red
update = blue
subupdate = cornflowerblue
diffadd = lightgreen
diffsub = indianred
\end{verbatim}
All colours are optional and will be filled in to the above default values if not given.
The colour are specified with X11 colour names, see \href{http://en.wikipedia.org/wiki/X11_color_names}{here}.

\subsection{Schemas}

Diamond needs to know which Spud schemas are installed and available on the
current system. This is specified in the \verb+schemata+ subdirectory of the
\verb+/etc/diamond+ and \verb+~/.diamond+ directories. The file names in the
\verb+schemata+ directory give the filename extension associated with a
particular problem description language. The content of the file is two or
more lines. The first line is the name of the problem description language,
the remaining lines are key value pairs (key = value) of alias names and the 
path of the XML syntax (\verb+.rng+) version of the corresponding schema.
A line with no "key =" part will be interpreted as "default =".

For example, the Fluidity package has a problem description language called
the Fluidity Markup Language which uses the file extension
\verb+.flml+. When installed on a system by the sysadmin, Fluidity might
create the file \verb+/etc/diamond/flml+ with the following contents:
\begin{verbatim}
Fluidity Markup Language
/usr/share/fluidity/fluidity_options.rng
\end{verbatim}
An individual user \verb+jrluser+ might have the current version of the
Fluidity svn tree checked out in their home directory and would need to
point diamond at the (posssibly updated) schema in their source
tree. \verb+jrluser+ would then create the file
\verb+/home/jrluser/.diamond/schemata/flml+ which would contain:
\begin{verbatim}
Fluidity Markup Language
/home/jrluser/fluidity/schemas/fluidity_options.rng
\end{verbatim}
Now Diamond will pick up the version of the schema in \verb+jrluser+'s svn
tree rather than the version the sysadmin installed.
Alternatively they could just add an alias to the existing \verb+//home/jrluser/.diamond/schemata/flml+:
\begin{verbatim}
Fluidity Markup Language
/usr/share/fluidity/fluidity_options.rng
jrl = /home/jrluser/fluidity/schemas/fluidity_options.rng
\end{verbatim}

It is also possible to specify a URL over HTTP for the location of the
schema. For example:
\begin{verbatim}
Fluidity Markup Language
http://amcg.ese.ic.ac.uk/svn/fluidity/trunk/tools/fluidity_options.rng
\end{verbatim}
This is to facilitate centralised deployments.

\section{Starting Diamond}

Diamond can be started in several ways.

When executed as
\begin{verbatim}
diamond
\end{verbatim}
Diamond will offer the user the choice of which registered schema to use.
If only one schema is registered, then it will use that by default.
Diamond will open a blank document of the language specified by the schema to be edited.

When executed as
\begin{verbatim}
diamond -s /path/to/schema.rng
\end{verbatim}
Diamond will open a blank document using the schema specified on the command line.

When executed as
\begin{verbatim}
diamond filename.suffix
\end{verbatim}
Diamond will inspect the registered schemas for the suffix specified
and use that schema.

\section{Using Diamond}

Diamond provides three views.
\begin{description}
\item[Main view] Is how Diamond starts up.
\item[Slice view] Shows same type elements together.
\item[Diff view] Shows the differences between the open document and another.
\end{description}

\subsection{Main view}
The main view shows the tree structure of the document on the left hand side and
data and documentation for the currently selected element on the right hand side.
The treeview can be traversed with either the mouse by clicking to open, close, add and 
destroy elements or with the keyboard, using the arrows keys, enter and delete.
Elements can be copied (puts the XML representation onto the clipboard) or XML can be pasted into 
an element.
Finally elements of the same type can be grouped together, this will bring all elements of the selected 
type to the top level and hide other elements. Ungroup will then return the view to normal.

\subsection{Slice view}
Slice view is opened by right clicking an element and selecting \verb+Slice+ or 
selecting an element and then selecting \verb+Slice+ in the \verb+Edit+ menu. 
The slice view shows a list of a given type of element. Each item shows the path, data and attributes for that element.

\subsection{Diff view}
To open diff view select \verb+Diff+ or \verb+Diff against last save+ in the \verb+Tools+ menu.
The diff view shows the differences between the currently open document (understood to be the newer version)
and a user selected document (understood to be the older version). The diff view does NOT understand some conventions
used in the main view such as hiding certain elements (for example real\_value or comment).
The colours used in the diff view can be set in the settings file see \ref{sec:colour}.

\subsection{Dynamic validation}
Diamond uses the schema to guide the editing of valid documents that the schema 
permits. When a blank document is opened, information necessary to make a valid
document is missing and the document is thus invalid\footnote{This is true for all
but a trivial schema which permits only one document.}. The invalidity of the document
is reflected in the colouration of the root node: valid nodes are coloured in black,
while invalid nodes are coloured in blue. As invalid nodes have their attributes
and data specified, their validity is dynamically updated and they are coloured appropriately.
The document as a whole is valid if and only if the root node is coloured black.

\subsection{Finding unused schema elements}
Diamond has a built in tool to test multiple document files against one schema to see what parts of the schema are not used.
In the \verb+Tools+ menu select \verb+Find schema usage+, it will prompt for an input directory. It will find every file with
a matching suffix (.flml for Fluidity Markup Langauge) in the directory tree and check the schema against them.
